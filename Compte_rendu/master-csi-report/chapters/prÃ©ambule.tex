\chapter*{Projets supports}
\label{chap:supports}

Nous présentons succinctement les deux projets servant de fondations à la réalisation des travaux présentés dans ce mémoire.

\subsection*{\indexed{HACL*}}
Acronyme pour "High assurance cryptography library"\footnote{\url{https://hacl-star.github.io/}}, lire \textit{"HACL star"}. Il s'agit d'une bibliothèque cryptographique développée au sein du \textbf{\indexed{Projet Everest}}\footnote{\url{https://project-everest.github.io/}}. Initié en 2016, ce projet porté par des chercheurs de l'Inria (équipe PROSECCO\footnote{Équipe de recherche rattachée au centre Inria de Paris, focalisée sur les méthodes formelles et la recherche en protocoles cryptologiques. Pour atteindre ces objectifs, l'équipe développe des langages de programmation, des outils de vérification\dots}), du \indexed{Centre de Recherche Microsoft} et de l'\indexed{Université Carnegie Mellon} a pour but de concevoir des systèmes informatiques formellement sécurisés, appliqués à l'environnement HTTPS. Cette bibliothèque, écrite en \indexed{F*} ("F star"), implémente tous les algorithmes de cryptographie modernes et est prouvée mathématiquement sûre. Elle est ensuite transcrite en C pour être directement employée dans n'importe quel projet. HACL* est notamment utilisé dans plusieurs systèmes de production, notamment Mozilla Firefox, le noyau Linux, le VPN WireGuard,\etc.


\subsection*{\indexed{Binsec}}
Binsec (contraction de \textit{Binary Security})\footnote{\url{https://binsec.github.io/}} est une plateforme open source développé pour évaluer la sécurité des logiciels au niveau binaire. Ce logiciel est développé et maintenu par une équipe du CEA List de l'\indexed{Université Paris-Saclay}, et accompagné de chercheurs de \indexed{Verimag}\footnote{Verimag est un laboratoire spécialisé dans les méthodes formelles pour une informatique sûre, avec des applications aux systèmes cyber-physiques. Fondé en 1993 au sein de l'Université Grenoble Alpes, puis rejoint par le CNRS, il a pour objectif la sécurité informatique dans les domaines des transports et de la santé.} et du \indexed{LORIA}\footnote{Laboratoire lorrain de recherche en informatique et ses applications; créé en 1997, c'est un centre de recherche commun au CNRS, à l'Université de Lorraine, à CentraleSupélec et à l'Inria.}. Il notamment est utilisé pour la recherche de vulnérabilités, la désobfuscation de logiciels malveillants et la vérification formelle de code binaire. Grâce à l'exécution symbolique Binsec peut explorer et modéliser le comportement d'un programme pour détecter des erreurs; détection réalisée en association avec des outils de fuzzing et/ou des solveurs SMT.
