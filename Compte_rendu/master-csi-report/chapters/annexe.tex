
\begin{table}[!ht]
  \caption{Liste des options de compilations et leurs effets (non exhaustive), \url{https://gcc.gnu.org/onlinedocs/gcc/Optimize-Options.html}}
  \label{tab:compile_option}
  \small
  \begin{adjustwidth}{-2.5cm}{-2cm}
  \begin{center}
    \begin{tabular}{ll}
    \hlineB{2}
    \textbf{Option de compilation} & \textbf{Effet} \\
    \rowcolor{lightgray}
    -O0 & Compile le plus vite possible \\
    -O1 / -O & Compile en optimisant la taille et le temps d'exécution \\
    \rowcolor{lightgray}
    -O2 & Comme -O1 mais en plus fort, temps de compilation plus élevé mais exécution plus rapide\\
    -O3 & Comme -O2, avec encore plus d'options, optimisation du binaire\\
    \rowcolor{lightgray}
    -Os & Comme -O2 avec des options en plus, réduction de la taille du binaire au détriment du temps d'exécution  \\
    -Ofast & optimisations de la vitesse de compilation\\
    \rowcolor{lightgray}
    -Oz & optimisation agressive  sur la taille du binaire\\
    \hlineB{2}
    \end{tabular}
  \end{center}
  \end{adjustwidth}
\end{table}


\begin{figure}
  \caption{Structure finale de Érysichthon, depuis l'usager}
  \label{fig:erysichthon_structure}
  \centering
  Les flèches grises indiquent tous les éléments actionnables individuellement.
  \begin{tikzpicture}[scale= 1.3, transform shape]

    % Styles
    \tikzstyle{startstop} = [rectangle, rounded corners, minimum width=2cm, minimum height=1cm, text centered, draw=black, fill=green!30]
    \tikzstyle{process} = [rectangle, minimum width=2cm, minimum height=1cm, text centered, draw=black, fill=orange!30]
    \tikzstyle{arrow} = [thick,->,>=stealth]
    \tikzset{zone1/.style={rectangle, rounded corners, draw=red, dashed, fill=red!10, inner sep=0.3cm}}
    \tikzset{zone2/.style={rectangle, rounded corners, draw=blue, dashed, fill=blue!10, inner sep=0.5cm, text width=3cm}}
    \tikzset{zone3/.style={rectangle, rounded corners, draw=green, dashed, fill=green!10, inner sep=0.3cm}}
    \tikzset{zone4/.style={rectangle, rounded corners, draw=green, dashed, fill=green!30!blue!5, inner sep=0.3cm}}
    
    % Noeuds
    \node (make_test) [startstop] {ANDHRÍMNIR};
    \node (test) [below of = make_test] {Génère les tests};
    \node (test1) [below of=test, xshift=0.5cm, yshift=0.6cm] {\small{test1}};
    \node (test2) [below of=test1, yshift=0.6cm] {\small{test2}};
    \node (test3) [below of=test2, yshift=0.6cm] {\small{test3}};
    \node (test4) [below of=test3, yshift=0.6cm] {\small{test4}};
    \node (dots) [below of=test4, yshift=0.6cm] {\small{$\dots$}};
    
    \foreach \node in {test1, test2, test3, test4, dots} {
      \draw (test.south) |- (\node.west);
      }
      
    \node (start) [startstop, right of=make_test, yshift=2cm, xshift = 4cm] {Érysichthon};
    \node (blanc) [right of=make_test,yshift=-1cm, xshift = 2cm] {};
    \node (compilateur) [below of = start] {Compilateur};
    \node (hacl) [below of = compilateur, xshift=2cm] {Compile Hacl*};
    \node (tests) [below of = hacl, yshift = 0.3cm] {Compile les tests};
    \node (binsec) [below of = tests, xshift=1.8cm] {Exécute Binsec};
    \node (analyse) [below of = binsec, yshift = 0.3cm] {Étude des résultats};
    % Flèches
    \draw [arrow] (compilateur) |- (hacl.west);
    \draw [arrow] (compilateur) |- (tests.west);
    \draw [arrow] (tests) |- (binsec.west);
    \draw [arrow] (tests) |- (analyse.west);

    % Zones
    \begin{scope}[on background layer]
        \node (zone2node) [zone2, fit=(start) (compilateur) (hacl) (tests) (binsec) (analyse) (make_test) (test) (test1) (test2) (test3) (test4) (dots)] {};
        \node (title) [anchor=north west] at (zone2node.north west) {\parbox{3.5cm}{\centering \Huge{\textbf{Érysichthon}}\\\scriptsize{\textit{control panel}}}};
        \node (zone_tests) [zone1, fit=(make_test) (test) (test1) (test2) (test3) (test4) (dots)] {};
        \node (zone_compilation) [zone3, fit=(start) (compilateur) (hacl) (tests) (binsec) (analyse)] {};
        \node (zone_binsec) [zone4, fit=(binsec) (analyse)] {};
    \end{scope}

    % Flèches 2
    
    \draw [arrow, dashed, opacity=0.5] (title) -- (zone_tests);
    \draw [arrow, dashed, opacity=0.5] (title) -- (zone_compilation.west);
    \draw [thick,>=stealth, dashed, opacity=0.4] (title) -- (blanc.north);
    \draw [arrow, dashed, opacity=0.5] (blanc.north) -- (zone_binsec);
    \draw [arrow, dashed, opacity=0.2] (zone_tests.north) -- (zone_compilation.west);
    \end{tikzpicture}
\end{figure}