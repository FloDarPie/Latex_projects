Ce travail de recherche s'inscrit dans le cadre du développement de la sécurité face aux attaques par canaux auxiliaires, plus précisément une sous-classe : les attaques temporelles. Cette classe d'attaque nécessite seulement l'usage d'un chronomètre pour être mise en œuvre. Cette facilité d'utilisation a rendu les concepteurs de systèmes sécurisés, notamment de bibliothèques cryptographiques, sensibles à ce type de menace et a permis le développement de contre-mesures. De récents travaux ont malheureusement mis en évidence qu'un code source résistant à ce type d'attaque pouvait redevenir vulnérable si compilé avec un mauvais compilateur. Nous allons présenter un outil permettant d'automatiser la vérification de binaires et ainsi d'attester de la sécurité concrète d'une bibliothèque cryptographique. Pour atteindre cet objectif, nous avons conçu un outil de génération de tests permettant une analyse complète de la bibliothèque étudiée. Nous avons aussi développé un ensemble de protocoles permettant de détailler notre méthodologie et d'effectuer une analyse généralisée à plusieurs architectures et à différents compilateurs.

\bigbreak

\textbf{Mots-clés :} Attaque par canal auxiliaire, temps constant, analyse symbolique, vérification formelle, compilateurs
