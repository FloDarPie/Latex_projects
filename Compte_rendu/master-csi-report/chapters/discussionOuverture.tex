\chapter*{Discussion et Ouverture}

\section*{Discussion}


Les résulats obtenues grâce à l'analyse d'Érysichthon ne permettent pas encore de conclure sur la sécurité globale de la bibliothèque Hacl*, il reste encore trop de fonctions non analysées ($\sim 39\%$). En revanche les premiers résultats sont encourageants et montre que l'utilisation des contre-mesures développé au chapitre \ref{chap:constantTimeSolution} est effectivement une bonne pratique. Compléter Érysichthon pour avoir une analyse complète est en tête de liste de la liste des tâche du projet.\medbreak

Actuellement, un seul compilateur a permis de produire ces résultats, il faut absolument étendre l'utilisation à d'autres compilateurs pour pouvoir croiser les résultats et avoir une étude plus complète quant à la sécurité de cette bibliothèque.\smallbreak

\subsection*{Retour sur les résultats}

Nous savons que les optimisations de compilateur modifient le binaire et peuvent insérer des fuites parce que les modifications changent la structure du programme. Avec cette outil, au lieu d'appeler frontalement \texttt{-O2}, nous pouvons plutôt appeler nominativement les options qui se cachent derrière : \texttt{-falign-functions}, \texttt{-falign-jumps}, \etc. Cette solution permettrait d'identifier les passes de compilateurs qui sont déterminantes pour l'apparition de failles. Il faut en revanche être attentif avec cette solution car \texttt{GCC},utilisé ici, fonctionne différemment de \texttt{LLVM+Clang}. Ce dernier emploie une représentation interne pour effectuer la transformation entre un programme source et un programme binaire.Cette représentation interne évolue entre les différentes étapes de compilations. Cela induit que certaines étapes, même si elle ne nous plaisent pas car elles ajoutent des failles, sont nécessaires pour les suivantes. Nous avons notamment pu croiser ce cas de figures durant notre période de test et à moins d'avoir un code adapté à chaque compilateur, la solution retenu fu de désactiver les optimisations de compilateurs pour le bloc de code concerné.\medbreak

En poursuivant dans cette voie, il nous est aussi possible d'identifier précisément les fonctions qui ont besoins d'être sécurisée et d'adapter leur compilation. Nous pourrions avoir une liste \texttt{sécurisées} qui identifie les fonctions dont la compilation induit trop de dégâts et les autres seraient compilées avec niveau d'optimisation plus élevé. Par exemple, si \textit{A.c} est compilé avec \texttt{-O0}, \textit{A.o} sera généré sans optimisation. De même, si \textit{B.c} est compilé \texttt{-O3}, \textit{B.o} contiendra des fonctions optimisées. Ainsi, dans le binaire final \textit{C.o}, les appels vers une fonction de \textit{A.o} sont des instructions de saut vers le code compilé avec \texttt{-O0}, et les appels vers \textit{B.o} des sauts vers du code compilé avec \texttt{-O3}. Nous obtenons un mélange de fonctions optimisées et non optimisées dans le même exécutable. Cette solution réduira les performances globales et mais préservera un niveau de sécurité plus élevé. Cette solution pave la voie vers une nouvelle étude.\medbreak

\subsection*{Implémentation dans le cadre d'une CI}

Utiliser Érysichthon c'est retrouver des vulnérabilités documentés, il nous a aussi permis d'identifier une vingtaine de fonctions présentant des fuites (même si la sécurité n'est pas engagée). Donc ajouter Érysichthon à une chaîne de tests, \textit{CI} ou Intégration Continue en langue de molière, est l'objectif principal que nous nous fixons. \citeauthor{schneider2024breakingbadcompilersbreak}, notre référence, énumèrais une liste de solution permettant de garantir la sécurité temps constant d'une librairie. Les solutions notables étaient de mettre en place un "\textit{distributeur}" de binaire, permettant au développeur de piocher selon ses besoins avec bien entendu tous les codes sources publiques, ou d'effectuer massivement des tests. La voie prise pour assurer la sécurité d'HACL* s'aligne avec cette dernière solution. Et ajouter cet outil à un mécanisme d'intégration continue permettra d'avoir des rapports réguliers dès la publication d'une nouvelle version de HACL* (lors d'ajout de nouvelles primitives cryptographiques ou de modification du code source).\smallbreak

L'avantage indéniable de cette méthode est que les failles seront découverte au fur et à mesure des tests effectués sur des compilateurs de plus en plus récents, limitant le risque d'avoir une faille se révélant dans la nature.


\section*{Ouverture}

La sécurité des binaires face aux attaques temporelles n'est pas chose aisé. Nous nous sommes concentrés avec ce projet sur la vérification de binaire et la conception d'une certification de sécurité vers un compilateur. Nous considérons le compilateur comme une boîte noire. Cette approche se justifie par la non main mise sur le développement de tels projets. Historiquement les ordinateurs étaient lents et la conception du logiciel devait être réalisé avec précision pour optimiser ou guider la conception des binaires. Les compilateurs et leurs optimisations ont permis d'améliorer globalement la production de binaire et réduire les coûts de compilation. Aujourd'hui les ordinateurs ont des composants très rapide. La lecture des instructions est plus rapide que l'exécution desdites instructions.\smallbreak

\subsection*{Prompt état de l'art des processeurs}

Le rapport de \citeauthor{constantTimePornin} \cite{constantTimePornin} décrit très bien les mécanismes employé par les processeurs pour accélérer l'exécution de programmes. Une modélisation d'un processus peut être présenté ainsi :
\begin{enumerate}
    \item Charge l'instruction pointé par le compteur de programme, incrémente ce dernier.
    \item Décode l'instruction.
    \item Exécute l'instruction.
\end{enumerate}

La dernière étape peut induire une modification de la mémoire, des chargements ou modifications de valeurs de registre ou résoudre des calculs. Tandis que chaque étape peut se résoudre selon un même rythme, cette troisième opération, à cause de sa variance, peut entraîner des ralentissements. Les constructeurs de matériels ont donc développé des techniques accélérer le temps total d'exécution et permettre ainsi des gains de performances. \smallbreak

Voici les principales techniques employées :

\begin{itemize}
    \item[\textbf{Pipeline}] Permet l'exécution parallèle de différentes instructions. Les instructions se résolvent toujours dans l'ordre abstrait, mais l’exécution de la suivante commence alors que la précédente (et éventuellement d’autres encore en cours) n’est pas terminée.
    \item[\textbf{Prédiction de branche}] Pré-chargement de branchement. Grâce à la méthode précédente, le processeur peut avancer dans son préchargement d'instructions en explorant les deux côtés d'un branchement mémoire puis défausser la branche inutile.
    \item[\textbf{Renommage de registres}] Modification des registres employés pour éviter les dépendances. Grâce au Pipeline, il peut y avoir des ralentissements dans la prédiction à cause de valeurs dépendantes entre les instructions. Cette méthode emploie un autre registre temporairement pour simuler le comportement attendu.
    \item[\textbf{Micro-opérations}] Utilisation d'un encodge interne pour découper les instructions binaires en unité plus petites. Cette opération invisible pour le développeur permet d'optimiser le binaire (\eg combiner une comparaison et un branchement en une seule opération).
    \item[\textbf{Exécution désordonné}\footnote{Plus connus sous le terme technique de référence anglais : Out-of-order Execution}] Certaines opération demande malgré tout plus de temps d'exécution, grâce à sa capacité à lire et traduire en vance les opérations à réaliser, le processeur peut réordonner la file d'exécution.
\end{itemize}\medbreak

Ces résultats impliquent une rupture de la politique temps constant et sont invisible aux yeux du développeur. Une connaissance approfondi sur le fonctionnement des processeurs est nécessaires pour avoir connaissance de ces éléments et des possiblitié de les contre-carrés. Intel et ARM ont mis à disposition des variables à spécifier pour permettre la désactivation de tels optimisations pour certaines opérations (multiplication et division d'entier). Ces solutions partielles ne permettent pas d'avoir des garanties sur l'ensemble du binaire que l'on exécute.

\subsection*{Travaux de recherche au niveau du processeur}

Les architectures portées par Intel et ARM sont propriétaires et donc nous offre peu de possiblités d'évolution. RISC-V de l'autre côté, grâce à son enregistrement dans l'espace commun peut acceuillir des travaux d'universitaires. Cela a permis à des équipes Inria de développer des mécanismes au niveau du processeur qui permet de garantir une exécution sans fuite par canal auxiliaire. Les travaux de \citeauthor{twartingCT} \cite{twartingCT} proposent l'ajout d'instructions \textit{lock} et \textit{unlock} dans l'ISA de l'architectures pour encapsuler des opérations qui doivent être réalisées avec des accès au cache mémoire. Cette protection vient avec l'ajout d'une nouvelle structure microarchitecturale à ajouter sur une carte mère.\smallbreak

Plus récemment, \citeauthor{cryptoeprint:2025/1331} \cite{cryptoeprint:2025/1331} propose un circuit logique qui réalise une opération de masquage et permet de conserver les propriétés temps constant.

