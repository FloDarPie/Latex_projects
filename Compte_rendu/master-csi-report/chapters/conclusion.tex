\chapter*{Conclusion}
\addcontentsline{toc}{chapter}{Conclusion}

À travers ce mémoire, nous avons vu la mise au point d'Érysichthon, un outil d'analyse automatique de bibliothèques cryptographiques. Cette conception a demandé l'implémentation d'un ensemble de modules et notamment d'Andhrímnir, un outil de génération de tests. Ce module permet d'avoir au minimum un test par fonction présente dans la bibliothèque analysée. Érysichthon est un outil utilisant Binsec, notamment les extensions permettant la vérification formelle des fuites temporelles par l'analyse symbolique non-interférente.\smallbreak

Cette conception s'est réalisée grâce à l'étalonnage des réponses à nos questions de recherche. Les travaux de \citeauthor{schneider2024breakingbadcompilersbreak} ont mis en lumière (\textbf{QR1}) que la propagation des garanties de sécurité n'est pas nécessairement conservée lors de la compilation. Vouloir conserver cette propagation demande d'utiliser un compilateur spécialisé et d'ajouter des spécifications dans le code source pour préserver le niveau de sécurité attendu. Nous avons aussi pu voir les contraintes inhérentes à cette solution, ce qui nous a décidés à explorer d'autres solutions. Celle retenue nous a guidés vers l'étude et l'analyse de binaires après compilation. Cette solution consiste à vérifier si les éléments de sécurité ont été préservés ou si des fuites sont apparues. Le choix judicieux (\textbf{QR2}) d'utiliser Binsec nous a permis de mettre en place une méthodologie et l'automatisation de l'analyse de binaires à travers un éventail d'architectures et de compilateurs. Avec ces briques et de nouveaux protocoles, nous avons pu dresser un cahier des charges (\textbf{QR3}) pour effectuer un changement d'échelle et réussir à concevoir un outil effectuant la vérification de bibliothèques cryptographiques.\smallbreak

La première bibliothèque cryptographique vérifiée par Érysichthon est HACL*. Cette bibliothèque a la particularité d'être formellement vérifiée et présente des garanties de sécurité à propos du code source de ses fonctions. L'étude \cite{schneider2024breakingbadcompilersbreak} avait mis en lumière des fuites en fonction des compilateurs employés. Nous avons reproduit les analyses réalisées et cela nous a permis d'identifier des bogues présents dans Binsec.\smallbreak

Cette étape nous a permis de concevoir des méthodes pour automatiser les analyses manuelles que nous venions de réaliser. La production de protocoles précis et détaillés nous a permis d'accélérer le processus d'implémentation d'Érysichthon et d'effectuer notre première analyse complète de HACL* sur x86\_64 avec \texttt{GCC 12.02}.\smallbreak

La discussion des résultats obtenus nous a permis d'établir une liste de correctifs nécessaires pour avoir une analyse complète d'une bibliothèque cryptographique. Nous avons pu, avec cette analyse de HACL*, concevoir une pré-certification de la sécurité de cette bibliothèque. Nous attestons d'une sécurité de bout en bout, initiée formellement dans le code source et conservée au niveau assembleur dans le binaire compilé.
