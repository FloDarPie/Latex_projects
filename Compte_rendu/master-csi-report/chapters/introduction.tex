\chapter*{Introduction}
\label{chap:introduction}

Le développement sécurisé est une tâche ardue. Si on porte notre regard vers le langage de programmation C, un guide \cite{progC_guide}\footnote{Développé par Anne Canteaut, chercheuse de l'équipe COSMIQ, récemment entrée à l'Académie des Sciences} porté par l'\indexed{INRIA}\footnote{Institut National de Recherche en Informatique et Automatisme} est complet en 133 pages tandis qu'un guide pour du développement sécurisé\cite{anssi_guideForSecureprogramming} produit par l'\indexed{ANSSI}\footnote{Agence nationale de la Sécurité des Systèmes d'Information} comprends 182 pages. Cette comparaison met en évidence la discipline requise par le développeur pour faire de la programmation sécurisée; en sus des connaissances, pour améliorer son efficacité, en cryptologie, en architecture matérielle et en programmation bas niveau .\medbreak

Malheureusement, malgré ces compétences, des erreurs peuvent être produites puis exploiter pour réaliser des attaques sur ces sytèmes sécurisés. Il existe de nombreuses classe d'attaques, certaines exploitant les défauts de conception (type A) tandis que d'autres utilisent les caractéristiques matériels (type B). Pour limiter ces effets de bords, la pratique de la programmation formelle permet de contraindre le développeur et empêcher l'apparitions de ces erreurs. La production de preuve formelle du code à l'issu de cet exercice permet d'avoir des garanties contres les attaques de type A.

En revanche, pour se défendre d'attaques de type B (ou attaques par canal auxiliaire) dépendantes du matériel support du programme, il est plus difficile d'avoir une méthode miracle. Actuellement, la solution la plus courante est d'identifier les attaques existantes pour ajouter les contre-mesures adéquates permettant d'avoir un système sécurisé. Une sous-classe d'attaque continue malgré tout de résister à cette méthode : les attaques temporelles.\medbreak

Découverte par Paul Kocher en 1996 \cite{crypto-1996-1469}, il les décrit comme <<\textit{une mesure précise du temps requis par des opérations sur les clées secrètes, permettrait à un attaquant de casser le cryptosystème}>>. Face à cette menace, l'enjeu d'avoir un code \textit{\indexed{achrognostique}}\footnote{Néologisme de Thomas Pornin dans son article \textit{Constant-Time Code: The Pessimist Case} \cite{constantTimePornin} pour désigner un code sans connaissance de temps} vient se rajouter aux pratiques de programmations sécurisées. Et pourtant, si contre les attaques de type A on arrive à concevoir des preuves mathématiques de sécurité associées à nos systèmes sécurisés, les garanties contre les attaques de type B sont plus faible ou inexistente.\medbreak

En 2024, les travaux de \citeauthor{schneider2024breakingbadcompilersbreak} \cite{schneider2024breakingbadcompilersbreak} prouvent qu'un usage inadapté de compilateur sur un système sécurisé introdui des failles exploitables. Ces résultats, observables partiellement avec des travaux antérieurs (par exemple \cite{binsecRel2019}), montrent qu'un usage inadéquat d'options fournies au compilateur optimise un code prouvé sécurisé et retire les protections indiquées par le développeur. Cela nous amène à plusieurs questions de recherche (QR) que nous tenterons de répondre à travers ce document.
\begin{enumerate}
    \item[\textbf{QR1}] Est-il possible de détecter les failles qui permettent une attaque temporelles ?
    \item[\textbf{QR2}] Est-il possible d'automatiser la détection de ces failles ?
    \item[\textbf{QR3}] Est-il possible d'étendre ce méchanisme entre différentes architectures ?
\end{enumerate}

Les réponses à ces questions permettraient de développer des systèmes sécurisés, communs entre différents supports et d'avoir des garanties de sécurité.

\begin{Acorriger}{Fin d'introduction - à finir}
    Dans la première section nous reviendrons sur les attaques temporelles, leurs impacts et comment s'en protéger. Puis, Dans la deuxième section nous présenterons les outils disponibles à l'analyse et pour la détection de failles. Nous continuerons, dans la troisième section, avec la présentation de nos contributions. \textit{Enfin, dans la quatrième section nous présenterons les mécanismes présent au plus bas niveau de l'informatique pour se protéger des attaques temporelles.}\medbreak
    Ce travail a été réalisé au sein du centre INRIA de Paris dans le cadre du projet \textit{Everest} concernant la mise au point de Hacl*.
    
\end{Acorriger}
