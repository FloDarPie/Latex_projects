\chapter*{Introduction}

Le développement sécurisé est une tâche ardue. Si nous portons notre regard vers le langage de programmation C, un guide \cite{progC_guide}\footnote{Développé par Anne Canteaut, chercheuse de l'équipe COSMIQ, récemment entrée à l'Académie des Sciences} porté par l'\indexed{INRIA}\footnote{Institut National de Recherche en Informatique et Automatisme} est complet en 133 pages tandis qu'un guide pour du développement sécurisé\cite{anssi_guideForSecureprogramming} produit par l'\indexed{ANSSI}\footnote{Agence nationale de la Sécurité des Systèmes d'Information} comprend 182 pages. Cette comparaison met en évidence la discipline requise par le développeur pour faire de la programmation sécurisée; en plus des connaissances en cryptologie, en architecture matérielle et en programmation bas niveau nécessaires pour améliorer son efficacité. \medbreak

Malheureusement, malgré ces compétences, des erreurs peuvent être produites puis exploitées pour réaliser des attaques sur ces systèmes sécurisés. Il existe de nombreuses classes d'attaques, certaines exploitant les défauts de conception (type A) tandis que d'autres utilisent les caractéristiques matérielles (type B). Pour limiter ces effets de bord, la pratique de la programmation formelle permet de contraindre le développeur et empêcher l'apparition de ces erreurs. La production de preuve formelle du code à l'issue de cet exercice permet d'avoir des garanties contre les attaques de type A.

En revanche, pour se défendre d'attaques de type B dépendantes du matériel supportant le programme, il est plus difficile d'avoir une méthode générique. Actuellement, la solution la plus courante consiste à identifier les attaques existantes afin d'ajouter les contre-mesures adéquates permettant d'obtenir un système sécurisé. Une sous-classe d'attaques continue malgré tout à résister à cette méthode : les attaques temporelles.\medbreak

Découverte par Paul Kocher en 1996 \cite{crypto-1996-1469}, il les décrit comme <<\textit{une mesure précise du temps requis par des opérations sur les clés secrètes, permettrait à un attaquant de casser le cryptosystème}>>. Face à cette menace, l'enjeu d'avoir un code \textit{\indexed{achrognostique}}\footnote{Néologisme de Thomas Pornin dans son article \textit{Constant-Time Code: The Pessimist Case} \cite{constantTimePornin} pour désigner un code sans connaissance de temps} vient s'ajouter aux pratiques de programmations sécurisées. Et pourtant, si contre les attaques de type A nous arrivons à concevoir des preuves mathématiques de sécurité associées à nos systèmes sécurisés, les garanties contre les attaques de type B sont plus faibles ou inexistentes.\medbreak

En 2024, les travaux de \citeauthor{schneider2024breakingbadcompilersbreak} \cite{schneider2024breakingbadcompilersbreak} prouvent qu'un usage inadapté de compilateur sur un système sécurisé introduit des failles exploitables. Ces résultats, observables partiellement avec des travaux antérieurs (par exemple \cite{binsecRel2019}), montrent qu'un usage inadéquat d'options fournies au compilateur optimise un code prouvé sécurisé et retire les protections indiquées par le développeur. Cela nous amène à plusieurs questions de recherche (QR) auxquelles nous tenterons de répondre dans ce document.
\begin{enumerate}
    \item[\textbf{QR1}] Est-il possible de détecter les failles qui permettent une attaque temporelles ?
    \item[\textbf{QR2}] Est-il possible d'automatiser la détection de ces failles ?
    \item[\textbf{QR3}] Est-il possible d'intégrer ce mécanisme à un processus d'intégration continue ?
\end{enumerate}

Les réponses à ces questions permettraient de développer des systèmes sécurisés, communs entre différents supports et d'avoir des garanties de sécurité.

\begin{Acorriger}{Fin d'introduction - à finir}
    Dans la première section nous reviendrons sur les attaques temporelles, leur impact et comment s'en protéger. Puis, dans la deuxième section nous présenterons les outils disponibles à l'analyse et à la détection de failles. Nous continuerons, dans la troisième section, avec la présentation de nos contributions.\medbreak
    Ce travail a été réalisé au sein du centre INRIA de Paris.
\end{Acorriger}
