\chapter*{Préambule}
\label{chap:prelude}


\subsection*{\indexed{HACL*} \footnote{\url{https://hacl-star.github.io/}}}
Acronyme pour "High assurance cryptography library", lire \textit{"HACL star"}. Il s'agit d'une bibliothèque cryptographique développée au sein du \textbf{\indexed{Projet Everest}}\footnote{\url{https://project-everest.github.io/}}. Initié en 2016, ce projet porté par des chercheurs de l'INRIA (équipe PROSECCO\footnote{Équipe de recherche rattaché au centre INRIA de Paris, focalisé sur les méthodes formelles et la recherche en protocoles cryptologiques. Pour ces objectifs, l'équipe développe des langages de programmation, des outils de vérification\dots}), du \indexed{Centre de Recherche Microsoft} et de l'\indexed{Université Carnégie Mellon} a pour but de concevoir des systèmes informatiques formellement sécurisés appliqués à l'environnement HTTPS. Cette bibliothèque écrite en \indexed{F*} ("F star") implémente tous les algorithmes de cryptographie modernes et est prouvé mathématiquement sûre. Elle est ensuite transcrite en C pour être directement employée dans n'importe quel projet. HACL* est notamment utilisé dans plusieurs systèmes de production, notamment Mozilla Firefox, le noyau Linux, le VPN WireGuard, et bien d'autres \etc.


\subsection*{\indexed{Binsec} \footnote{\url{https://binsec.github.io/}}}
\textit{\textbf{Bin}ary \textbf{Sec}urity} est un ensemble d'outils open source développé pour améliorer la sécurité des logiciels au niveau binaire. Ce logiciel est développé et maintenu par une équipe du CEA List de l'\indexed{Université Paris-Saclay}, et accompagné par des chercheurs de \indexed{Verimag}\footnote{Verimag est un laboratoire spécialisé dans les méthodes formelles pour une informatique sûre, avec des applications aux systèmes cyber-physiques. Fondé en 1993 au sein de l'Université Grenoble Alpes, puis rejoins par le CNRS, il a pour objectif la sécurité dans les domaines des transports et de la santé.} et de \indexed{LORIA}\footnote{Laboratoire lorrain de recherche en informatique et ses applications; crée en 1997, c'est un centre de recherche commun au CNRS, l'Université de Lorraine, CentraleSupélec et l'Inria.}. Il est utilisé pour la recherche de vulnérabilités, la désobfuscation de logiciels malveillants et la vérification formelle de code assembleur. Grâce à l'exécution symbolique et l'interprétation abstraite, Binsec peut explorer et modéliser le comportement d'un programmes pour détecter des erreurs; détection réalisée avec des outils de fuzzing et des solveurs SMT.
