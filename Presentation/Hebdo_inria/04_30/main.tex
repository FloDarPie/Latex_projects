%%%%%%%%%%%%%%%%%%%%%%%%%%%%%%%%%%%%%%%%%%%%%%%%%%%%%%%%%%%%%%%%%%%%%%
%
%         Copyright (c) 2023, gitlabci_gallery / latex
%         All rights reserved.
%
%%%%%%%%%%%%%%%%%%%%%%%%%%%%%%%%%%%%%%%%%%%%%%%%%%%%%%%%%%%%%%%%%%%%%%

\documentclass[A4,svgnames,9pt,aspectratio=169]{beamer}
%% document options:
%% - aspectratio = { 43, 169, 1610 }
%% - utf8
%%


\setlength{\footskip}{300pt}
\usepackage[french]{babel}

\hypersetup{
   allcolors   = rouge_inria,
   pdfauthor   = {Duzés Florian},
   pdftitle    = {\@title},
   pdfsubject  = {Point hebdomadaire, bi-mensuel du stage},
   pdfkeywords = {entretien, observation du travail}
}

%%%%%%%%%%%%%%%%%%%%%%%%%%%%%%%%%%%%%%%%%%%%%%%%%%%%%%%
%%
%%%%%%%%%%%%%%%%%%%%%%%%%%%%%%%%%%%%%%%%%%%%%%%%%%%%%%%

\title[titrecourt]{Réunion flash}
\subtitle{Point hebdomadaire}
\date[30/04/2025]{date long}
\author[Duzes Florian]{Duzeé Florian}

\usetheme{inria}

\begin{document}

%%%%%%%%%%%%%%%%%%%%%%%%%%%%%%%%%%%%%%%%%%%%%%%%%%%%%%%
%%
%%%%%%%%%%%%%%%%%%%%%%%%%%%%%%%%%%%%%%%%%%%%%%%%%%%%%%%

\frame{\titlepage}

%%%%%%%%%%%%%%%%%%%%%%%%%%%%%%%%%%%%%%%%%%%%%%%%%%%%%%%

% Le titre des planches de sommaire est \contentsname, sa valeur
% est fixée ici à "Sommaire" par défaut.
\renewcommand{\contentsname}{Sommaire}

\frame{\tocpage}


%%--%%--%%--%%--%%--%%--%%--%%--%%--%%--%%--%%--%%--%%--%%
 
\section{Un peu de hauteur}
\frame{\sectionpage}

\begin{frame}{État des missions}
  \begin{tikzpicture}[
    node distance=2cm,
    box/.style={rectangle, draw=black, thick, minimum width=3cm, minimum height=1cm, text centered, rounded corners, font=\bfseries},
    arrow/.style={->, >=Stealth, thick, draw=arrowColor},
    contentBoxDone/.style={rectangle, draw=black, thick, fill=board_lightGray!40, rounded corners, minimum width=3cm, minimum height=2cm, text width=3cm, align=center},
    contentBox/.style={rectangle, draw=black, thick, rounded corners, minimum width=3cm, minimum height=2cm, text width=3cm, align=center}
]

    % DESSINS
    \node[box, fill=faitColor] (fait) {Fait};
    \node[box, fill=enCoursColor, right=of fait] (en_cours) {En cours};
    \node[box, fill=aFaireColor, right=of en_cours] (a_faire) {\parbox{3cm}{\centering Prévus \\ \tiny{Pour le 06/05}}};
    \draw[arrow] (fait) -- (en_cours);
    \draw[arrow] (en_cours) -- (a_faire);

    % Choses réalisées
    \node[contentBoxDone, below=0.5cm of fait] {
        \begin{itemize}
          \item Implémentation à l'inria
          \item Configurations des outils  
          \item Tutoriels de Binsec
        \end{itemize}
    };

    % En cours
    \node[contentBox, below=0.5cm of en_cours, fill=enCoursColor!15] {
        \begin{itemize}
            \item Reproduction des premiers résultats en x86\_64
        \end{itemize}
    };

    % Point fixer
    \node[contentBox, below=0.5cm of a_faire, fill=aFaireColor!15] {
        \begin{itemize}
            \item Scripts d'analyse pour le dossier \textit{/tests} en x86\_64 
            \item Basculer sur l'architecture ARM
        \end{itemize}
    };

      \end{tikzpicture}

\end{frame}

%  .  .  .  .  .  .  .  .  .  .  .  .  .  .  .  .  .  .  %

\begin{frame}{Travail réalisé}
  \begin{tikzpicture}[
        node distance=2cm,
        box/.style={rectangle, draw=black, thick, minimum width=3cm, minimum height=1cm, text centered, rounded corners, font=\bfseries},
        arrow/.style={->, >=Stealth, thick, draw=arrowColor},
        contentBox/.style={rectangle, draw=black, thick, rounded corners, minimum width=3cm, minimum height=2cm, text width=3cm, align=center}
    ]

    % DESSINS
    \node[box, fill=faitColor] (fait) {Fait};
    \node[box, fill=enCoursColor, right=of fait] (en_cours) {En cours};
    \node[box, fill=aFaireColor, right=of en_cours] (a_faire) {Prévus};
    \draw[arrow] (fait) -- (en_cours);
    \draw[arrow] (en_cours) -- (a_faire);

    % Choses réalisées
    \node[contentBox, below=0.5cm of fait,fill=enCoursColor!15] (un) {
        \begin{itemize}
          \item Reproduction des premiers résultats en x86\_64
        \end{itemize}
    };

    \node[contentBox, below=0.05cm of un, fill=aFaireColor!15] {
      \begin{itemize}
        \item Scripts d'analyse pour le dossier \textit{/tests} en x86\_64 
        \item Basculer sur l'architecture ARM
      \end{itemize}
  };

    % En cours
    \node[contentBox, below=0.5cm of en_cours, fill=enCoursColor!5] {
        \begin{itemize}
            \item Étendre les scripts pour le dossier \textit{/tests} en ARM
            \item Vérification de la méthode
        \end{itemize}
    };

    % Objectif court
    \node[below=0.5cm of a_faire, text width=3cm, align=center] {
        \begin{itemize}
            \item 
        \end{itemize}
    };

      \end{tikzpicture}

\end{frame}


%%--%%--%%--%%--%%--%%--%%--%%--%%--%%--%%--%%--%%--%%--%%

\section{Protocole x86\_64}
\frame{\sectionpage}

%  .  .  .  .  .  .  .  .  .  .  .  .  .  .  .  .  .  .  %

\begin{frame}[fragile]{En amont}

  \begin{columns}
  \begin{column}{0.3\textwidth}
    \begin{block}{Avant la compilation}
      \begin{itemize}
        \item Cibler la fonction à analyser
      \end{itemize}
    \end{block}      
  \end{column}


  \begin{column}{0.7\textwidth}
  \begin{lstlisting}[style=CStyle, caption={chacha20Poly1305-128-binsec-test.c}, gobble=4]
    int main (int argc, char *argv[])
    {
      Hacl_AEAD_Chacha20Poly1305_Simd128_encrypt
        (cipher, tag, plain, BUF_SIZE, aead_aad, AAD_SIZE, aead_key, aead_nonce);
      exit(0);
    }
  \end{lstlisting}
  \end{column}
  \end{columns}
\end{frame}
\begin{frame}[fragile, noframenumbering]{En amont}

  \begin{columns}
  \begin{column}{0.3\textwidth}
    \begin{block}{Avant la compilation}
      \begin{itemize}
        \item Cibler la fonction à analyser
        \item Dresser la structure des paramètres
      \end{itemize}
    \end{block}      
  \end{column}


  \begin{column}{0.7\textwidth}
    \begin{lstlisting}[style=CStyle, caption={Hacl\_AEAD\_Chacha20Poly1305\_Simd128\_encrypt.h}, gobble=6]
      void
      Hacl_AEAD_Chacha20Poly1305_Simd128_encrypt(
        uint8_t *output,
        uint8_t *tag,
        uint8_t *input,
        uint32_t input_len,
        uint8_t *data,
        uint32_t data_len,
        uint8_t *key,
        uint8_t *nonce
      );
    \end{lstlisting}
  \end{column}
  \end{columns}
\end{frame}
\begin{frame}[fragile, noframenumbering]{En amont}

  \begin{columns}
  \begin{column}{0.3\textwidth}
    \begin{block}{Avant la compilation}
      \begin{itemize}
        \item Cibler la fonction à analyser
        \item Dresser la structure des paramètres
        \item Identifier nos paramètres
      \end{itemize}
    \end{block}      
  \end{column}


  \begin{column}{0.7\textwidth}
    \begin{lstlisting}[style=CStyle, caption={Hacl\_AEAD\_Chacha20Poly1305\_Simd128\_encrypt.h},, gobble=6]
      void
      Hacl_AEAD_Chacha20Poly1305_Simd128_encrypt(
        uint8_t *output,
        uint8_t *tag,         
        uint8_t *input,       //secret
        uint32_t input_len,   //secret
        uint8_t *data,        //secret
        uint32_t data_len,    //secret
        uint8_t *key,         //secret
        uint8_t *nonce        //secret
      );
    \end{lstlisting}
  \end{column}
  \end{columns}
\end{frame}

%  .  .  .  .  .  .  .  .  .  .  .  .  .  .  .  .  .  .  %

\begin{frame}[fragile]{En aval}

  \begin{columns}
    \begin{column}{0.3\textwidth}
      \begin{block}{Après la compilation}
        \begin{itemize}
          \item Réalisation du coredump
        \end{itemize}
      \end{block}      
    \end{column}
  
  
    \begin{column}{0.7\textwidth}
    \begin{lstlisting}[style=GDBStyle, caption={script\_dump}, gobble=6]
      break Hacl_AEAD_Chacha20Poly1305_Simd128_encrypt
      run
      generate-core-file core.snapshot
      kill
      quit
    \end{lstlisting}
    \end{column}
    \end{columns}
  
\end{frame}
\begin{frame}[fragile, noframenumbering]{En aval}

  \begin{columns}
    \begin{column}{0.3\textwidth}
      \begin{block}{Après la compilation}
        \begin{itemize}
          \item Réalisation du coredump
          \item Réalisation du script.ini
        \end{itemize}
      \end{block}      
    \end{column}
  
  
    \begin{column}{0.7\textwidth}
      \begin{lstlisting}[style=INIStyle, caption={study.ini}, label={lst:ini}, gobble=8]
        starting from core

        halt at @[rsp, 8]

        hook <Hacl_AEAD_Chacha20Poly1305_Simd128_encrypt>
            (cipher, mac, msg, len, aad, aad_len, key, nonce) with
          @[len, 16384] := secret
          for i<64> in 0 to len - 1 do
            @[msg + i] := secret
          end
          @[aad_len, 12] := secret
          for i<64> in 0 to aad_len - 1 do
            @[aad + i] := secret
          end
          @[key, 32] := secret
          @[nonce, 12] := secret
        end

        explore all
    \end{lstlisting}
    \end{column}
    \end{columns}
  
\end{frame}
\begin{frame}[fragile, noframenumbering]{En aval}

  \begin{columns}
    \begin{column}{0.3\textwidth}
      \begin{block}{Après la compilation}
        \begin{itemize}
          \item Réalisation du coredump
          \item Réalisation du script.ini
          \item Exécution de Binsec
        \end{itemize}
      \end{block}      
    \end{column}
  
  
    \begin{column}{0.7\textwidth}
    \begin{lstlisting}[style=MakefileStyle, caption={Makefile}, gobble=6]
      TARGET = $(wildcard *.exe)
      SNAP=core.snapshot
      SCRIPT = study.ini

      all: core-dump binsec

      core-dump: 
        gdb -x script_coredump $(TARGET)

      binsec:
        binsec -sse -sse-depth 1000000 -sse-script $(SCRIPT) -checkct $(SNAP)
    \end{lstlisting}
    \end{column}
    \end{columns}
  
\end{frame}

%%--%%--%%--%%--%%--%%--%%--%%--%%--%%--%%--%%--%%--%%--%%

\section{Protocole ARM}
\frame{\sectionpage}

%  .  .  .  .  .  .  .  .  .  .  .  .  .  .  .  .  .  .  %

\begin{frame}{Point de départ}

  \begin{columns}
    \begin{column}{0.5\textwidth}
      \begin{block}{Objectif fort}
        Ne pas utiliser de \textit{coredump}.
    \end{block}
  \end{column}

  \begin{column}{0.5\textwidth}
  \end{column}
  
  
  \end{columns}
\end{frame}
\begin{frame}[noframenumbering]{Point de départ}

  \begin{columns}
    \begin{column}{0.5\textwidth}
      \begin{block}{Objectif fort}
        Ne pas utiliser de \textit{coredump}.
    \end{block}
  \end{column}

  \begin{column}{0.5\textwidth}
    \begin{block}{Conclusion}
      Analyser le binaire.
    \end{block}
  \end{column}
  
  
  \end{columns}
\end{frame}
\begin{frame}[noframenumbering]{Point de départ}

  \begin{columns}
    \begin{column}{0.5\textwidth}
      \begin{block}{Objectif fort}
        Ne pas utiliser de \textit{coredump}.
    \end{block}
  \end{column}

  \begin{column}{0.5\textwidth}
    \begin{block}{Conclusion}
      Analyser le binaire.
      \begin{itemize}
        \item Adresse de la fonction cible.
        \item Adresse des paramètres
        \item Adresse de fin de fonction
      \end{itemize}
    \end{block}
  \end{column}
  
  
  \end{columns}
\end{frame}

%  .  .  .  .  .  .  .  .  .  .  .  .  .  .  .  .  .  .  %

\begin{frame}[fragile]{Premier essai}
  \begin{columns}
    \begin{column}{0.5\textwidth}
      \begin{lstlisting}[style=INIStyle, gobble=8]
        sp<64> := 0x400fc0

        # End of disas binsec
        @[sp, 64] := 0x401120 as return_address

        # Information from the objdump -x
        @[<.plt>, 22] from file
        @[<.text>, 348] from file
        @[<.fini>, 2] from file
        @[<.rodata>, 30] from file
        @[<.eh_frame_hdr>, 17] from file
        @[<.eh_frame>, 84] from file
        @[<.init_array>, 1] from file
        @[<.fini_array>, 1] from file
        @[<.dynamic>, 60] from file
        @[<.got>, 2] from file
        @[<.got.plt>, 12] from file
        @[<.data>, 42] from file
        @[<.bss>, 0] from file
        \end{lstlisting}      
    \end{column}
    \begin{column}{0.5\textwidth}
      \begin{lstlisting}[style=INIStyle, caption={something\_like\_that.ini}, gobble=8]

        # Looking in binsec disas
        @[0x400fc0, 64] := 0x4294967184 as Hacl_Chacha20_chacha20_encrypt
        hook <Hacl_Chacha20_chacha20_encrypt> (len, out, text, key, n, ctr) with
          for i<64> in 0 to len - 1 do
            @[out + i] := secret
          end
          for i<64> in 0 to len - 1 do
            @[text + i] := secret
          end
          @[key, 32] := secret
          @[n, 12] := secret
        end

        starting from <main>
        explore all
      \end{lstlisting}
      
    \end{column}
  \end{columns}
  
\end{frame}

%  .  .  .  .  .  .  .  .  .  .  .  .  .  .  .  .  .  .  %

\begin{frame}[fragile]{Essai n°?}
  \begin{lstlisting}[style=INIStyle, caption={script.ini}, gobble=4]
    #fin de la zone de .text
    #@[sp, 8] := 0x00404860 as return_address
    
    #fin arbitraire : fin de calculs
    @[sp, 8] := 0x004005c0 as return_address
    
    # load common sections from ELF file
    load sections .plt, .text, .rodata, .data, .got, .got.plt, .bss from file
    
    secret global plain, aead_aad, aead_key, aead_nonce
    
    starting from <main>
    with concrete stack pointer
    
    halt at return_address
    explore all
  \end{lstlisting}
\end{frame}

%  .  .  .  .  .  .  .  .  .  .  .  .  .  .  .  .  .  .  %

\begin{frame}{Ça marche !?}

  \begin{block}{Exécution Binsec ne lève pas d'erreur}
        \begin{itemize}[<+->]
          \item \textit{\textbf{Uknwon}} avec l'adresse de fin de la section \textit{.text} - fin d'exécution de Binsec.
          \item \textit{\textbf{Secure}} avec l'adresse de fin d'exécution (section \textit{.plt}).
        \end{itemize}
  \vspace{1cm}
  \begin{center}
  \end{center}
  \end{block}
\end{frame}

\begin{frame}[noframenumbering]{Ça marche !?}

  \begin{block}{Exécution Binsec ne lève pas d'erreur}
        \begin{itemize}
          \item \textit{\textbf{Uknwon}} avec l'adresse de fin de la section \textit{.text} - fin d'exécution de Binsec.
          \item \textit{\textbf{Secure}} avec l'adresse de fin d'exécution (section \textit{.plt}).
        \end{itemize}
  \vspace{1cm}
  \begin{center}
    Est-ce que la fonction est entièrement analysée ?
  \end{center}
  \end{block}
\end{frame}

%  .  .  .  .  .  .  .  .  .  .  .  .  .  .  .  .  .  .  %

\begin{frame}[fragile]{Vérification}

  
  \begin{block}{Tentatives sur cas simple}
    \begin{columns}
      \begin{column}{0.5\textwidth}
        \begin{lstlisting}[style=CStyle, caption={good\_ct.c}, gobble=10]
          void f(uint8_t msg[BUF_SIZE], uint8_t key[KEY_SIZE])
          {
              for (int i=0; i<BUF_SIZE; i++)
              {
                  msg[i] = msg[i] ^ key[i%KEY_SIZE];
              }
          }
        \end{lstlisting}
      \end{column}
      \begin{column}{0.5\textwidth}
        \begin{lstlisting}[style=CStyle, caption={bad\_ct.c}, gobble=10]
          void f(uint8_t msg[BUF_SIZE], uint8_t key[KEY_SIZE])
          {
                  if( key[0] > key[1] )
                  {
                      msg[0] = msg[0] ^ key[0];    
                  }
          }
        \end{lstlisting}
      \end{column}
    \end{columns}
  \end{block}


\end{frame}

\begin{frame}[fragile, ,noframenumbering]{Vérification}

  
  \begin{block}{Tentatives sur cas simple}
    \begin{columns}
      \begin{column}{0.5\textwidth}
        \begin{lstlisting}[style=CStyle, caption={good\_ct.c}, gobble=10]
          void f(uint8_t msg[BUF_SIZE], uint8_t key[KEY_SIZE])
          {
              for (int i=0; i<BUF_SIZE; i++)
              {
                  msg[i] = msg[i] ^ key[i%KEY_SIZE];
              }
          }
        \end{lstlisting}
      \end{column}
      \begin{column}{0.5\textwidth}
        \begin{lstlisting}[style=CStyle, caption={bad\_ct.c}, gobble=10]
          void f(uint8_t msg[BUF_SIZE], uint8_t key[KEY_SIZE])
          {
                  if( key[0] > key[1] )
                  {
                      msg[0] = msg[0] ^ key[0];    
                  }
          }
        \end{lstlisting}
      \end{column}
    \end{columns}
  \end{block}
  \vspace{0.5cm}
  \setbeamercolor{block title}{bg=board_lightGray, fg=white}
  \setbeamercolor{block body}{bg=board_lightGray!20, fg=black}
  \begin{block}{Binsec :}
    \begin{columns}
      \begin{column}{0.5\textwidth}
        \begin{center}
          \textit{\textbf{Secure}}
        \end{center}
        \vfill
      \end{column}
      \begin{column}{0.5\textwidth}
        \begin{center}
          \textit{\textbf{Insecure}}
        \end{center}
        \vfill
      \end{column}
    \end{columns}
  \end{block}
\end{frame}

%  .  .  .  .  .  .  .  .  .  .  .  .  .  .  .  .  .  .  %

\begin{frame}{Comment vérifier ?}
  \begin{block}{Idées de vérifications}
    Dans l'objectif d'automatisation:
    \begin{itemize}[<+->]
      \item Détection fine des adresses de fin.
      \item Application d'une fonction "\textit{fin\_test}".
    \end{itemize}
    
  \end{block}
\end{frame}

%%--%%--%%--%%--%%--%%--%%--%%--%%--%%--%%--%%--%%--%%--%%

\section{Conclusion}
\frame{\sectionpage}

\begin{frame}{Conclusion}
  \vspace{2cm}
  \begin{columns}
    \begin{column}{0.5\textwidth}
      \begin{block}{Protocole x86\_64}
        \begin{itemize}
          \item terminé
        \end{itemize}        
      \end{block}
    \end{column}
    \begin{column}{0.5\textwidth}
      \vspace{0.5cm}
      \begin{block}{Protocole ARM}
        \begin{itemize}
          \item target=aarch64-none-linux-gnu
          \item à éclaircir/terminé
        \end{itemize}        
      \end{block}
       
    \end{column}
  \end{columns}

  \vspace{2cm}
  \small{\textit{Lancement vers d'autres architectures ou avancement dans l'automatisation ?}}

\end{frame}

%%%%%%%%%%%%%%%%%%%%%%%%%%%%%%%%%%%%%%%%%%%%%%%%%%%%%%%

%% Le texte est modifiable en changeant \thankyou
%% \renewcommand{\thankyou}{Thank You.}
\frame{\merci}


\end{document}


