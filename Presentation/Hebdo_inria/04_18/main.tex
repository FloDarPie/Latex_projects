%%%%%%%%%%%%%%%%%%%%%%%%%%%%%%%%%%%%%%%%%%%%%%%%%%%%%%%%%%%%%%%%%%%%%%
%
%         Copyright (c) 2023, gitlabci_gallery / latex
%         All rights reserved.
%
%%%%%%%%%%%%%%%%%%%%%%%%%%%%%%%%%%%%%%%%%%%%%%%%%%%%%%%%%%%%%%%%%%%%%%

\documentclass[A4,svgnames,9pt,aspectratio=169]{beamer}
%% document options:
%% - aspectratio = { 43, 169, 1610 }
%% - utf8
%%

%%
%% insert list of packages
%%

\setlength{\footskip}{300pt}
\usepackage[french]{babel}

\hypersetup{
   allcolors   = rouge_inria,
   pdfauthor   = {Duzés Florian},
   pdftitle    = {\@title},
   pdfsubject  = {Point hebdomadaire, bi-mensuel du stage},
   pdfkeywords = {entretien, observation du travail}
}

%%%%%%%%%%%%%%%%%%%%%%%%%%%%%%%%%%%%%%%%%%%%%%%%%%%%%%%
%%
%%%%%%%%%%%%%%%%%%%%%%%%%%%%%%%%%%%%%%%%%%%%%%%%%%%%%%%

\title[titrecourt]{Réunion flash}
\subtitle{Point hebdomadaire}
\date[18/04/2025]{date long}
\author[Duzes Florian]{Duzés Florian}

\usetheme{inria}

\begin{document}

%%%%%%%%%%%%%%%%%%%%%%%%%%%%%%%%%%%%%%%%%%%%%%%%%%%%%%%
%%
%%%%%%%%%%%%%%%%%%%%%%%%%%%%%%%%%%%%%%%%%%%%%%%%%%%%%%%

\frame{\titlepage}

%%%%%%%%%%%%%%%%%%%%%%%%%%%%%%%%%%%%%%%%%%%%%%%%%%%%%%%

% Le titre des planches de sommaire est \contentsname, sa valeur
% est fixée ici à "Sommaire" par défaut.
\renewcommand{\contentsname}{Sommaire}

\frame{\tocpage}


%%--%%--%%--%%--%%--%%--%%--%%--%%--%%--%%--%%--%%--%%--%%
 
\section{Bienvenue à l'INRIA}
\frame{\sectionpage}

\begin{frame}{Installation}
      \begin{enumerate}
         \item Prise en main du Linux
         \item Mise en place du travailleur INRIA (accés internet, profil en ligne, boite mail, accès restauration)
         \item Configuration des outils de travails (VS Code, LaTeX)
         \item Installation des librairies requises pour F*, Hacl*, Binsec
      \end{enumerate}
\end{frame}



\begin{frame}{Réunion de mardi}
    \begin{block}{Premier contact avec mes référents}
      Retour sur mon arrivée, et installation générale.
      \begin{itemize}
        \item Mail de référence.
      \end{itemize}

      \tiny{\begin{itemize}
        \item[*] Installation de BINSEC (https://github.com/binsec/binsec) et de HACL* (https://github.com/hacl-star/hacl-star). Pour HACL*, pas besoin d'installer la toolchain F* entière (en tout cas pour l'instant), mais assure toi de réussir à compiler le code dans dist/gcc-compatible, ainsi que dans tests
        \item[*] Familiarisation avec BINSEC: le tutoriel est disponible ici (https://github.com/binsec/binsec/tree/master/doc/sse). Il peut également être utile de parcourir certains articles de recherche, particulièrement https://binsec.github.io/assets/publications/papers/2020-sp.pdf
        \item[*] Application de BINSEC à HACL*: Je te mets en pièce jointe un (court) document qui montre comment se servir de BINSEC pour analyser une des implémentations d'HACL* (ChachaPoly avec instructions AVX). Je te conseille avant tout d'essayer de reproduire sur ta machine le processus. Une fois que ce sera fait, essaie de l'adapter à d'autres primitives cryptographiques.
      \end{itemize}
      }
      
    \end{block}

    \begin{block}{Missions claires}
      \begin{itemize}
        \item Tutoriel Binsec à finir.
      \end{itemize}
    \end{block}

\end{frame}

%%--%%--%%--%%--%%--%%--%%--%%--%%--%%--%%--%%--%%--%%--%%

\section{À l'assaut du tutoriel}
\frame{\sectionpage}

%  .  .  .  .  .  .  .  .  .  .  .  .  .  .  .  .  .  .  %

\begin{frame}{Assimilation}
  \begin{block}{Travail réalisé en deux temps}
    \begin{itemize}
      \item Temps de lecture et compréhension des mécanismes
      \item Temps de reproduction et compilation des travaux
    \end{itemize}
  \end{block}
\end{frame}

%  .  .  .  .  .  .  .  .  .  .  .  .  .  .  .  .  .  .  %

\begin{frame}{Difficultés et ralentissement}

  \begin{itemize}
    \item Compilation - manquement de librairies
    \item Prise en main, lecture de documentation (radare2)
  \end{itemize}
  
\end{frame}

%  .  .  .  .  .  .  .  .  .  .  .  .  .  .  .  .  .  .  %

\begin{frame}{Fin du tutoriel}
  Tutoriel terminé jeudi \footnote{Je me suis appuyé sur le mail et j'ai lu les deux articles données. Constant-Time:The Pessimist Case}. J'ai des fichiers récapitulatif/généraux qui font office de mémo.
  \begin{itemize}
    \item protocole.md
    \item binsec\_ref.md
  \end{itemize}

  Dans la branche \textit{git/binsec} installée en locale, j'ai modifié le tutoriel pour passer outre des difficultés rencontrés.

  
\end{frame}

%%--%%--%%--%%--%%--%%--%%--%%--%%--%%--%%--%%--%%--%%--%%

\section{Calendrier prévisionnel}
\frame{\sectionpage}

%  .  .  .  .  .  .  .  .  .  .  .  .  .  .  .  .  .  .  %

\begin{frame}{Missions enregistrées}
  \begin{itemize}
    \item TUTO BINSEC -> a finir d'ici vendredi
    \item => 1er résult à reproduire
    \begin{itemize}
      \item étendre à d'autre plateforme, exploration de noyaux
      \item étendre à d'autres algo
    \end{itemize}
    
    \item Core dump à protocoler
  \end{itemize}
\end{frame}

%%--%%--%%--%%--%%--%%--%%--%%--%%--%%--%%--%%--%%--%%--%%

\section{Conclusion}
\begin{frame}{Conclusion}

    \begin{block}{Cap fixé, à l'aventure}
      Prochain résultats :
      \begin{enumerate}
        \item Reproduction du travail de mes aînés.
        \item Avancement dans les architectures
        \begin{enumerate}
          \item x86\_64
          \item ARM 
        \end{enumerate}
      \end{enumerate}
    Objectif pour le 6 mai (reste 10 jours).
    \end{block}

\end{frame}

%%%%%%%%%%%%%%%%%%%%%%%%%%%%%%%%%%%%%%%%%%%%%%%%%%%%%%%

%% Le texte est modifiable en changeant \thankyou
%% \renewcommand{\thankyou}{Thank You.}
\frame{\merci}


\end{document}


