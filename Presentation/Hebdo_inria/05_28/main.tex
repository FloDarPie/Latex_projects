%%%%%%%%%%%%%%%%%%%%%%%%%%%%%%%%%%%%%%%%%%%%%%%%%%%%%%%%%%%%%%%%%%%%%%
%
%         Copyright (c) 2023, gitlabci_gallery / latex
%         All rights reserved.
%
%%%%%%%%%%%%%%%%%%%%%%%%%%%%%%%%%%%%%%%%%%%%%%%%%%%%%%%%%%%%%%%%%%%%%%

\documentclass[A4,svgnames,9pt,aspectratio=169]{beamer}
%% document options:
%% - aspectratio = { 43, 169, 1610 }
%% - utf8
%%


\setlength{\footskip}{300pt}
\usepackage[french]{babel}

\hypersetup{
   allcolors   = rouge_inria,
   pdfauthor   = {Duzés Florian},
   pdftitle    = {\@title},
   pdfsubject  = {Point hebdomadaire, bi-mensuel du stage},
   pdfkeywords = {entretien, observation du travail}
}

%%%%%%%%%%%%%%%%%%%%%%%%%%%%%%%%%%%%%%%%%%%%%%%%%%%%%%%
%%
%%%%%%%%%%%%%%%%%%%%%%%%%%%%%%%%%%%%%%%%%%%%%%%%%%%%%%%

\title[titrecourt]{Réunion flash}
\subtitle{Point hebdomadaire}
\date[28/05/2025]{date long}
\author[Duzes Florian]{Duzés Florian}

\usetheme{inria}

\begin{document}

%%%%%%%%%%%%%%%%%%%%%%%%%%%%%%%%%%%%%%%%%%%%%%%%%%%%%%%
%%
%%%%%%%%%%%%%%%%%%%%%%%%%%%%%%%%%%%%%%%%%%%%%%%%%%%%%%%

\frame{\titlepage}

%%%%%%%%%%%%%%%%%%%%%%%%%%%%%%%%%%%%%%%%%%%%%%%%%%%%%%%

% Le titre des planches de sommaire est \contentsname, sa valeur
% est fixée ici à "Sommaire" par défaut.
\renewcommand{\contentsname}{Sommaire}

\frame{\tocpage}


%%--%%--%%--%%--%%--%%--%%--%%--%%--%%--%%--%%--%%--%%--%%
 
\section{État des lieux}
\frame{\sectionpage}

\begin{frame}{Point actuel}
  \begin{tikzpicture}[
    node distance=2cm,
    box/.style={rectangle, draw=black, thick, minimum width=3cm, minimum height=1cm, text centered, rounded corners, font=\bfseries},
    arrow/.style={->, >=Stealth, thick, draw=arrowColor},
    contentBoxDone/.style={rectangle, draw=black, thick, fill=board_lightGray!40, rounded corners, minimum width=3cm, minimum height=2cm, text width=3cm, align=center},
    contentBox/.style={rectangle, draw=black, thick, rounded corners, minimum width=3cm, minimum height=2cm, text width=3cm, align=center}
]

    % DESSINS
    \node[box, fill=faitColor] (fait) {Fait};
    \node[box, fill=enCoursColor, right=of fait] (en_cours) {En cours};
    \node[box, fill=aFaireColor, right=of en_cours] (a_faire) {\parbox{3cm}{\centering Prévus \\ \tiny{Pour le 15/06}}};
    \draw[arrow] (fait) -- (en_cours);
    \draw[arrow] (en_cours) -- (a_faire);

    % Choses réalisées
    \node[contentBoxDone, below=0.5cm of fait] {
        \begin{itemize}
          \item Chaîne de compilation
          \item Identifications d'erreur de \textbf{p256}
        \end{itemize}
    };

    % En cours
    \node[contentBox, below=0.5cm of en_cours, fill=enCoursColor!15] {
        \begin{itemize}
          \item Automatisation des \textit{-test.c}
          \item Préparation d'une config pour la génération des tests
        \end{itemize}
    };

    % Point fixer
    \node[contentBox, below=0.5cm of a_faire, fill=aFaireColor!15] {
        \begin{itemize}
            \item Automatisation des .ini
            \item Couvrir les tests
            \item Couvrir les primitives Hacl*
            \item RiskV - compil à la main de tests
        \end{itemize}
    };




      \end{tikzpicture}

\end{frame}

%  .  .  .  .  .  .  .  .  .  .  .  .  .  .  .  .  .  .  %

\begin{frame}{Réalisation}
  \begin{tikzpicture}[
    node distance=2cm,
    box/.style={rectangle, draw=black, thick, minimum width=3cm, minimum height=1cm, text centered, rounded corners, font=\bfseries},
    arrow/.style={->, >=Stealth, thick, draw=arrowColor},
    contentBoxDone/.style={rectangle, draw=black, thick, fill=board_lightGray!40, rounded corners, minimum width=3cm, minimum height=2cm, text width=3cm, align=center},
    contentBox/.style={rectangle, draw=black, thick, rounded corners, minimum width=3cm, minimum height=2cm, text width=3cm, align=center}
]

    % DESSINS
    \node[box, fill=faitColor] (fait) {Fait};
    \node[box, fill=enCoursColor, right=of fait] (en_cours) {En cours};
    \node[box, fill=aFaireColor, right=of en_cours] (a_faire) {\parbox{3cm}{\centering Prévus \\ \tiny{Pour le 15/06}}};
    \draw[arrow] (fait) -- (en_cours);
    \draw[arrow] (en_cours) -- (a_faire);

    % Choses réalisées
    \node[contentBoxDone, below=0.5cm of fait] {
        \begin{itemize}
          \item \textbf{p256} et analyse à la main
          \item Compilator (Compiler vers différentes architectures)
          \item Toolchain et compilation vers Risc-V
          \item Automatisation des .ini
        \end{itemize}
    };

    % En cours
    \node[contentBox, below=0.5cm of en_cours, fill=enCoursColor!15] {
        \begin{itemize}
          \item Automatisation des \textit{-test.c}
          \item Préparation d'une config pour la génération des tests
          \item Planification du mémoire
          \item Rédaction de documentation, protocoles et méthode
        \end{itemize}
    };

    % Point fixer
    \node[contentBox, below=0.5cm of a_faire, fill=aFaireColor!15] {
        \begin{itemize}
            \item Couvrir les tests
            \item Couvrir les primitives Hacl*
            \item RiscV - compil à la main de tests
        \end{itemize}
    };

      \end{tikzpicture}

\end{frame}

%%--%%--%%--%%--%%--%%--%%--%%--%%--%%--%%--%%--%%--%%--%%

\section{Automatique, pas facile}
\frame{\sectionpage}

\begin{frame}[fragile]{Automatique et facile}

  \begin{tikzpicture}[auto, remember picture, overlay]
    \tikzstyle{startstop} = [rectangle, minimum width=2cm, minimum height=1cm, text centered, draw=black, fill=green!60]
    \tikzstyle{arrow} = [thick,->,>=stealth, line width=1.5pt]

    % Noeuds
    \node (raw) at (3,-2.9)  {};
    \node (matching) at (6.8,-3.8) {};
    \node (matching2) at (6.8,-4.2) {};
    \node (final) at (3,-5.5) {};

    \draw [arrow, bend right, color=blue!40] (raw) to (matching);
    \draw [arrow, bend right, color=blue!60] (matching2) to (final);

  \end{tikzpicture}

  \begin{columns}
    \begin{column}{0.5\textwidth}
      \begin{lstlisting}[style=global, caption={Hacl\_AEAD\_Chacha20Poly1305\_decrypt}, gobble=8]
        uint32_t
          Hacl_AEAD_Chacha20Poly1305_decrypt
          uint8_t *output,   uint8_t *input,
          uint32_t input_len,   uint8_t *data,
          uint32_t data_len,   uint8_t *key,
          uint8_t *nonce,   uint8_t *tag
      \end{lstlisting}
      \vspace{2cm}
      Hacl\_AEAD\_Chacha20Poly1305\_decrypt.c
    \end{column}


    \begin{column}{0.5\textwidth}

      \begin{lstlisting}[style=CStyle, caption={matching.json}]
{
  "input":"BUF_SIZE",
  "input_len":"BUF_SIZE",
  "output":"BUF_SIZE",
  "output_len":"BUF_SIZE",
  "key":"KEY_SIZE",
  "nonce":"NONCE_SIZE",
  "tag":"TAG_SIZE",
  "data":"AAD_SIZE",
  "data_len":"AAD_SIZE",
  "BUF_SIZE":16384,
  "KEY_SIZE":32,
  "NONCE_SIZE":12,
  "AAD_SIZE":12,
  "TAG_SIZE":16
}
      \end{lstlisting}

    \end{column}
  \end{columns}

\end{frame}

%  .  .  .  .  .  .  .  .  .  .  .  .  .  .  .  .  .  .  %

\begin{frame}[fragile]{Coquille dans l'engrenage}

      \begin{lstlisting}[style=CStyle, caption={counting.json}]
{
    "*output": 41,
    "*tag": 12,
    "*input": 35,
    "input_len": 33,
    "*data": 14,
    "data_len": 14,
    "*key": 44,
    "*nonce": 15,
    "*expanded_key": 2,
    "*cipher": 6,
    "*plain": 16,
    "*a": 112,
    "*b": 107,
    "*res": 97,
    "*n": 70,
    "bBits": 28,
    "*k": 79,
    "len": 64,
}
      \end{lstlisting}

\end{frame}

%  .  .  .  .  .  .  .  .  .  .  .  .  .  .  .  .  .  .  %

\begin{frame}[fragile]{Besoin de plus d'information}
  \begin{columns}
    \begin{column}{0.5\textwidth}
      \begin{lstlisting}[style=CStyle, caption={Hacl\_Bignum256.h}, gobble=6]
      /**
      Write `a + b mod 2^256` in `res`.

        This functions returns the carry.

        The arguments a, b and res are meant to be 256-bit bignums, i.e. uint64_t[4]
      */
      uint64_t Hacl_Bignum256_add(uint64_t *a, uint64_t *b, uint64_t *res);
    \end{lstlisting}
    \end{column}
    \begin{column}{0.5\textwidth}
    \begin{lstlisting}[style=CStyle, caption={Hacl\_Bignum256.h}, gobble=6]
      /**
      Write `a * b` in `res`.

        The arguments a and b are meant to be 256-bit bignums, i.e. uint64_t[4].
        The outparam res is meant to be a 512-bit bignum, i.e. uint64_t[8].
      */
      void Hacl_Bignum256_mul(uint64_t *a, uint64_t *b, uint64_t *res);
    \end{lstlisting}
    \end{column}
  \end{columns}
\end{frame}

%  .  .  .  .  .  .  .  .  .  .  .  .  .  .  .  .  .  .  %

\begin{frame}[fragile]{Affectation de signature}
  \begin{tikzpicture}[auto, remember picture, overlay]
    \tikzstyle{startstop} = [rectangle, minimum width=2cm, minimum height=1cm, text centered, draw=black, fill=green!60]
    \tikzstyle{arrow} = [thick,->,>=stealth, line width=1.5pt]

    % Noeuds
    \node (raw) at (2.8,-0.6)  {};
    \node (twin) at (6.8,-1.5) {};
    \node (twin2) at (6.8,-3.2) {};
    \node (matching) at (4.8,-4) {};
    \node (matching2) at (3.3,-5.5) {};
    \node (final) at (6.8,-6.2) {};

    \draw [arrow, bend left, color=blue!35] (raw) to (twin);
    \draw [arrow, color=blue!55] (twin2) to (matching);
    \draw [arrow, bend right, color=blue!75] (matching2) to (final);

  \end{tikzpicture}

  \begin{columns}
    \begin{column}{0.4\textwidth}
        \textit{Fonction extraite d'un .h}
      \vspace{1cm} 
      \begin{lstlisting}[style=CStyle, caption={matching.json}]
{
  "input_8_encrypt":"BUF_SIZE"
  ,"input_len_32_encrypt":"BUF_SIZE"
  ,"output_8_encrypt":"BUF_SIZE"
  ,"key_8_encrypt":"KEY_SIZE"

  "BUF_SIZE":16384,
  "KEY_SIZE":32,
}
      \end{lstlisting}
    \end{column}
    \begin{column}{0.1\textwidth}
      
    \end{column}


    \begin{column}{0.5\textwidth}

      \begin{lstlisting}[style=CStyle, caption={twin.json}]
{
  "Chacha20Poly1305_encrypt":"encrypt"
    ,"Chacha20Poly1305_decrypt":"encrypt"
    ,"32_add":"32_add"
    ,"32_sub":"32_add"
    ,"32_add_mod":"32_add"
    ,"32_sub_mod":"32_add"
    ,"32_mul":"32_mul"
    ,"32_sqr":"32_mul"
    ,"32_mod":"32_mod"
}
      \end{lstlisting}
      \vspace{1cm}
      \large{\textit{function-tested}.c}
    \end{column}
  \end{columns}
\end{frame}

%  .  .  .  .  .  .  .  .  .  .  .  .  .  .  .  .  .  .  %

\begin{frame}[fragile]{Création du tampon}
\begin{columns}
  \begin{column}{0.4\textwidth}
    \begin{minipage}[t][8cm][t]{\linewidth}
      \begin{block}{Besoin d'identification}
        Format des données :
        \begin{itemize}
          \item Nom de la fonction
          \begin{itemize}
            \item \textit{Hacl\_family\_}[\textit{op}]
          \end{itemize}
          \item Type de données
          \begin{itemize}
            \item \textit{uint}[\textit{X}]\textit{\_f}
          \end{itemize}
        \end{itemize}
      \end{block}
      \pause
      \begin{block}{Tampon unique}
        parametre\_X\_op  
      \end{block}
    \end{minipage}
  \end{column}
  \begin{column}{0.6\textwidth}
    \begin{minipage}[t][10cm][t]{\linewidth}
      \begin{lstlisting}[style=CStyle, caption={matching.json}]
    "a_32_32_add":"BUF_SIZE_8"
    ,"b_32_32_add":"BUF_SIZE_8"
    ,"res_32_32_add":"BUF_SIZE_8"
    ,"n_32_32_add":"BUF_SIZE_8"
    ,"BUF_SIZE_8":8
    ,"a_32_32_mul":"BUF_SIZE_8"
    ,"b_32_32_mul":"BUF_SIZE_8"
    ,"res_32_32_mul":"BUF_SIZE_16"
    ,"BUF_SIZE_16":16
      \end{lstlisting}

      \begin{lstlisting}[style=CStyle, caption={twin.json}]
    "32_add":"32_add"
    ,"32_sub":"32_add"
    ,"32_add_mod":"32_add"
    ,"32_sub_mod":"32_add"
    ,"32_mul":"32_mul"
    ,"32_sqr":"32_mul"
      \end{lstlisting}
    \end{minipage}
  \end{column}
\end{columns}
  
\end{frame}

%%--%%--%%--%%--%%--%%--%%--%%--%%--%%--%%--%%--%%--%%--%%

\section{Risc-V me voilà}
\frame{\sectionpage}

\begin{frame}{À partir de rien}
  \begin{block}{Conception du compilateur croisé}
    Depuis la source officielle:
    \begin{itemize}
      \item \textit{https://github.com/riscv-collab/riscv-gnu-toolchain}
      \pause
      \item Après 3h et des poussières, ajout du compilateur : \textbf{riscv64-unknown-linux-gnu-gcc}
    \end{itemize}
  \end{block}
\end{frame}

%  .  .  .  .  .  .  .  .  .  .  .  .  .  .  .  .  .  .  %

\begin{frame}[fragile]{Analyse de Binsec}
  \begin{block}{Le fameux p256}
    Modifications du script \textit{.ini}:
    \begin{enumerate}
      \item Section "\textit{.data}" -> "\textit{.sdata}"
      \item Ajout de l'adresse de fin
    \end{enumerate}
  \end{block}

  \begin{lstlisting}[style=global, caption={study.ini}, gobble=2]
  [sse:warning] Enumeration of jump targets @ 0x000104ae hit the limit 3 and may be incomplete
  [sse:warning] Cut path 3 (non executable) @ 0x00000000
  [sse:warning] Cut path 2 (non executable) @ 0xffffffffffffffff
  [sse:warning] Cut path 1 (non executable) @ 0xfffffffffffffffe
  \end{lstlisting}

  \textit{Analyse corrigé en direct}

\end{frame}

%  .  .  .  .  .  .  .  .  .  .  .  .  .  .  .  .  .  .  %

\begin{frame}{Plan pour Risc-V}
  \begin{block}{Objectifs fixés}
    \begin{itemize}
      \item Étendre l'analyse Binsec sur les tests natifs de Hacl*
      \item Intégrer la chaîne de compilation à Érysicthon
      \item Ajouter une compilation vers l'architecture 32 bits
      \item Ajouter la compilation via Clang et LLVM
    \end{itemize}
  \end{block}
\end{frame}


%%--%%--%%--%%--%%--%%--%%--%%--%%--%%--%%--%%--%%--%%--%%



\section{Conclusion}
\frame{\sectionpage}

\begin{frame}{Conclusion}

  \begin{columns}
    \begin{column}{0.3\textwidth}
      \begin{minipage}[t][2cm][t]{\linewidth}
        \begin{block}{Automatisation}
          \begin{itemize}
            \item Continuer la génération des fichiers \textit{-test.c}
            \item Activer la chaîne de compilation
          \end{itemize}
        \end{block}
      \end{minipage}
    \end{column}
    \pause
    \begin{column}{0.3\textwidth}
      \begin{minipage}[t][2cm][t]{\linewidth}
        \begin{block}{Toolchain Risc-V}
          \begin{itemize}
            \item Compilation d'autres toolchain : 32 bits
            \item Compiler plus de -test.c
            \item Effectuer plus d'analyse Binsec
          \end{itemize}
        \end{block}
      \end{minipage}
    \end{column}
    \pause
    \begin{column}{0.3\textwidth}
      \begin{minipage}[t][2cm][t]{\linewidth}
        \begin{block}{Rédaction du mémoire}
          \begin{itemize}
            \item Poser les idées
            \item Concevoir le puit
            \item Plan du plan le 10/06
          \end{itemize}
        \end{block}
      \end{minipage}
    \end{column}
  \end{columns}

\end{frame}

%%%%%%%%%%%%%%%%%%%%%%%%%%%%%%%%%%%%%%%%%%%%%%%%%%%%%%%

%% Le texte est modifiable en changeant \thankyou
%% \renewcommand{\thankyou}{Thank You.}
\frame{\merci}


\end{document}

