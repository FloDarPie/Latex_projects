%%%%%%%%%%%%%%%%%%%%%%%%%%%%%%%%%%%%%%%%%%%%%%%%%%%%%%%%%%%%%%%%%%%%%%
%
%         Copyright (c) 2023, gitlabci_gallery / latex
%         All rights reserved.
%
%%%%%%%%%%%%%%%%%%%%%%%%%%%%%%%%%%%%%%%%%%%%%%%%%%%%%%%%%%%%%%%%%%%%%%

\documentclass[A4,svgnames,9pt,aspectratio=169]{beamer}
%% document options:
%% - aspectratio = { 43, 169, 1610 }
%% - utf8
%%


\setlength{\footskip}{300pt}
\usepackage[french]{babel}

\hypersetup{
   allcolors   = rouge_inria,
   pdfauthor   = {Duzés Florian},
   pdftitle    = {\@title},
   pdfsubject  = {Point hebdomadaire, bi-mensuel du stage},
   pdfkeywords = {entretien, observation du travail}
}

%%%%%%%%%%%%%%%%%%%%%%%%%%%%%%%%%%%%%%%%%%%%%%%%%%%%%%%
%%
%%%%%%%%%%%%%%%%%%%%%%%%%%%%%%%%%%%%%%%%%%%%%%%%%%%%%%%

\title[titrecourt]{Réunion flash}
\subtitle{Point hebdomadaire}
\date[25/06/2025]{date long}
\author[Duzes Florian]{Duzés Florian}

\usetheme{inria}

\begin{document}

%%%%%%%%%%%%%%%%%%%%%%%%%%%%%%%%%%%%%%%%%%%%%%%%%%%%%%%
%%
%%%%%%%%%%%%%%%%%%%%%%%%%%%%%%%%%%%%%%%%%%%%%%%%%%%%%%%

\frame{\titlepage}

%%%%%%%%%%%%%%%%%%%%%%%%%%%%%%%%%%%%%%%%%%%%%%%%%%%%%%%

% Le titre des planches de sommaire est \contentsname, sa valeur
% est fixée ici à "Sommaire" par défaut.
\renewcommand{\contentsname}{Sommaire}

\frame{\tocpage}


%%--%%--%%--%%--%%--%%--%%--%%--%%--%%--%%--%%--%%--%%--%%

\section{Informations supplémentaires}
\frame{\sectionpage}


\begin{frame}{Point d'informations}

  \begin{itemize}
    \item Semaine du 30 juin 
    \item Semaine du 14 juillet 
  \end{itemize}
  
\end{frame}

%%--%%--%%--%%--%%--%%--%%--%%--%%--%%--%%--%%--%%--%%--%%
 
\section{État des lieux}
\frame{\sectionpage}


\begin{frame}{Point actuel}
  \begin{tikzpicture}[
    node distance=2cm,
    box/.style={rectangle, draw=black, thick, minimum width=3cm, minimum height=1cm, text centered, rounded corners, font=\bfseries},
    arrow/.style={->, >=Stealth, thick, draw=arrowColor},
    contentBoxDone/.style={rectangle, draw=black, thick, fill=board_lightGray!40, rounded corners, minimum width=3cm, minimum height=2cm, text width=3cm, align=center},
    contentBox/.style={rectangle, draw=black, thick, rounded corners, minimum width=3cm, minimum height=2cm, text width=3cm, align=center}
]

    % DESSINS
    \node[box, fill=faitColor] (fait) {Fait};
    \node[box, fill=enCoursColor, right=of fait] (en_cours) {En cours};
    \node[box, fill=aFaireColor, right=of en_cours] (a_faire) {Prévus};
    \draw[arrow] (fait) -- (en_cours);
    \draw[arrow] (en_cours) -- (a_faire);

    % Choses réalisées
    \node[contentBoxDone, below=0.5cm of fait] {
        \begin{itemize}
          \item Étendu test de p256 avec LLVM / 32 / 64
          \item -test.json / generic json
          \item Hacl* -> RiscV
        \end{itemize}
        };

        % En cours
        \node[contentBox, below=0.5cm of en_cours, fill=enCoursColor!15] {
          \begin{itemize}
            \item Compiler Hacl* vers RiscV
            \item Test Binsec sur RiscV
            \item -test.json
          \end{itemize}
          };

          % Point fixer
          \node[contentBox, below=0.5cm of a_faire, fill=aFaireColor!15] {
            \begin{itemize}
            \item Chaîne de bout en bout 
            \item Couverture des architectures différentes
            \item Couverture des compilateurs
          \end{itemize}
          };
        \end{tikzpicture}
        
      \end{frame}

%  .  .  .  .  .  .  .  .  .  .  .  .  .  .  .  .  .  .  %

\begin{frame}{Réalisation}
  \begin{tikzpicture}[
    node distance=2cm,
    box/.style={rectangle, draw=black, thick, minimum width=3cm, minimum height=1cm, text centered, rounded corners, font=\bfseries},
    arrow/.style={->, >=Stealth, thick, draw=arrowColor},
    contentBoxDone/.style={rectangle, draw=black, thick, fill=board_lightGray!40, rounded corners, minimum width=3cm, minimum height=2cm, text width=3cm, align=center},
    contentBox/.style={rectangle, draw=black, thick, rounded corners, minimum width=3cm, minimum height=2cm, text width=3cm, align=center}
]

    % DESSINS
    \node[box, fill=faitColor] (fait) {Fait};
    \node[box, fill=enCoursColor, right=of fait] (en_cours) {En cours};
    \node[box, fill=aFaireColor, right=of en_cours] (a_faire) {Prévus};
    \draw[arrow] (fait) -- (en_cours);
    \draw[arrow] (en_cours) -- (a_faire);

    % Choses réalisées
    \node[contentBoxDone, below=0.5cm of fait] {
      \begin{itemize}
        \item Mise à jour de Binsec
      \end{itemize}
    };

    % En cours
    \node[contentBox, below=0.5cm of en_cours, fill=enCoursColor!15] {
        \begin{itemize}
          \item Remplir les config
          \item Gérer les dépendances
          \item Activer chaîne de compilation
          \item Compilation en x86\_64
        \end{itemize}
    };

    % Point fixer
    \node[contentBox, below=0.5cm of a_faire, fill=aFaireColor!15] {
        \begin{itemize}
            \item Chaîne de bout en bout 
            \item Couverture des architectures différentes
            \item Couverture des compilateurs
        \end{itemize}
    };
      \end{tikzpicture}

\end{frame}

%%--%%--%%--%%--%%--%%--%%--%%--%%--%%--%%--%%--%%--%%--%%

\section{Écrire les config et les \textit{Typedef}}
\frame{\sectionpage}

\begin{frame}[fragile]{Au commencement...}

  \begin{columns}
    \begin{column}{0.5\textwidth}
      \begin{lstlisting}[style=CStyle, language=Python, gobble=8]
        /* MIT License*/
    
        #ifndef __Hacl_Chacha20_H
        #define __Hacl_Chacha20_H
    
        #if defined(__cplusplus)
        extern "C" {
        #endif
    
        #include <string.h>
        #include "krml/internal/types.h"
        #include "krml/lowstar_endianness.h"
        #include "krml/internal/target.h"
    
        void
        Hacl_Chacha20_chacha20_encrypt(
          uint32_t len,
          uint8_t *out,
          uint8_t *text,
          uint8_t *key,
          uint8_t *n,
        \end{lstlisting}
        
      \end{column}
      
      \begin{column}{0.5\textwidth}
        \begin{lstlisting}[style=CStyle, language=Python, caption={Hacl\_AEAD\_Chacha20Poly1305.h}, gobble=8]   
          uint32_t ctr
        );

        void
        Hacl_Chacha20_chacha20_decrypt(
          uint32_t len,
          uint8_t *out,
          uint8_t *cipher,
          uint8_t *key,
          uint8_t *n,
          uint32_t ctr
        );
    
        #if defined(__cplusplus)
        }
        #endif
    
        #define __Hacl_Chacha20_H_DEFINED
        #endif
      \end{lstlisting}
      
    \end{column}
  \end{columns}




\end{frame}

%  .  .  .  .  .  .  .  .  .  .  .  .  .  .  .  .  .  .  %

\begin{frame}[fragile]{... le néant}

  \begin{lstlisting}[style=global, language=Python, caption={get\_data.py}, gobble=4]
    scheme_C = r'#if defined\(__cplusplus\)\s*\n(?:extern "C" \{\s*\n|}\s*\n)#endif\s*\n'
    scheme_define = r'#.*?(?<!\\)\n'
    scheme_comment = r'/\*.*?\*/'
    scheme_debug = r'//'
    scheme_type = r'typedef\s+(?:[^{}]|{(?:[^{}]|{[^{}]*})*})*?;'
    scheme_krml = r'KRML_DEPRECATED'
    

    #subduction
    try:
        content = re.sub(scheme_C, '', content)
        content = re.sub(scheme_define, '', content, flags=re.DOTALL)
        content = re.sub(scheme_comment, '', content, flags=re.DOTALL)
        content = re.sub(scheme_type, '', content, flags=re.DOTALL)
        content = re.sub(rf'^{scheme_debug}.*;?', '', content, flags=re.MULTILINE)
        content = re.sub(rf'^{scheme_krml}.*;?', '', content, flags=re.MULTILINE)
  \end{lstlisting}

\end{frame}

%  .  .  .  .  .  .  .  .  .  .  .  .  .  .  .  .  .  .  %

\begin{frame}[fragile]{La simplicité - \textit{28/05}}
  \begin{tikzpicture}[auto, remember picture, overlay]
    \tikzstyle{startstop} = [rectangle, minimum width=2cm, minimum height=1cm, text centered, draw=black, fill=green!60]
    \tikzstyle{arrow} = [thick,->,>=stealth, line width=1.5pt]

    % Noeuds
    \node (raw) at (2.8,-0.6)  {};
    \node (twin) at (6.8,-1.5) {};
    \node (twin2) at (6.8,-3.2) {};
    \node (matching) at (4.8,-4) {};
    \node (matching2) at (3.3,-5.5) {};
    \node (final) at (6.8,-6.2) {};

    \draw [arrow, bend left, color=blue!35] (raw) to (twin);
    \draw [arrow, color=blue!55] (twin2) to (matching);
    \draw [arrow, bend right, color=blue!75] (matching2) to (final);

  \end{tikzpicture}

  \begin{columns}
    \begin{column}{0.4\textwidth}
        \textit{Fonction extraite d'un .h}
      \vspace{1cm} 
      \begin{lstlisting}[style=CStyle, caption={matching.json}]
{
  "input_8_encrypt":"BUF_SIZE"
  ,"input_len_32_encrypt":"BUF_SIZE"
  ,"output_8_encrypt":"BUF_SIZE"
  ,"key_8_encrypt":"KEY_SIZE"

  "BUF_SIZE":16384,
  "KEY_SIZE":32,
}
      \end{lstlisting}
    \end{column}
    \begin{column}{0.1\textwidth}
      
    \end{column}


    \begin{column}{0.5\textwidth}

      \begin{lstlisting}[style=CStyle, caption={twin.json}]
{
  "Chacha20Poly1305_encrypt":"encrypt"
    ,"Chacha20Poly1305_decrypt":"encrypt"
    ,"32_add":"32_add"
    ,"32_sub":"32_add"
    ,"32_add_mod":"32_add"
    ,"32_sub_mod":"32_add"
    ,"32_mul":"32_mul"
    ,"32_sqr":"32_mul"
    ,"32_mod":"32_mod"
}
      \end{lstlisting}
      \vspace{1cm}
      \large{\textit{function-tested}.c}
    \end{column}
  \end{columns}
\end{frame}

%  .  .  .  .  .  .  .  .  .  .  .  .  .  .  .  .  .  .  %

\begin{frame}[fragile]{Les problèmes}
  \begin{columns}
    \begin{column}{0.5\textwidth}
      \begin{lstlisting}[style=CStyle, gobble=6]
      typedef struct Hacl_Hash_Blake2b_blake2_params_s
      {
        uint8_t digest_length;
        uint8_t key_length;
        uint8_t fanout;
        uint8_t depth;
        uint32_t leaf_length;
        uint64_t node_offset;
        uint8_t node_depth;
        uint8_t inner_length;
        uint8_t *salt;
        uint8_t *personal;
      }
      Hacl_Hash_Blake2b_blake2_params;
    
      typedef struct Hacl_Hash_Blake2b_index_s
      {
        uint8_t key_length;
      \end{lstlisting}
      
    \end{column}
    \begin{column}{0.5\textwidth}
      \begin{lstlisting}[style=CStyle, caption={Hacl\_Hash\_Blake2b.h}, gobble=6]
        uint8_t digest_length;
        bool last_node;
      }
      Hacl_Hash_Blake2b_index;
    
      Hacl_Hash_Blake2b_state_t
      *H_H_Blake2b_malloc_with_params_and_key(
        Hacl_Hash_Blake2b_blake2_params *p,
        bool last_node,
        uint8_t *k
      );
    \end{lstlisting}
    \end{column}
  \end{columns}

\end{frame}
%  .  .  .  .  .  .  .  .  .  .  .  .  .  .  .  .  .  .  %

\begin{frame}{Solution}
  \begin{block}{Remodéliser la génération des tests}
    \begin{itemize}
      \item Identifier les \textit{typedef}
      \item Ajouter dans les fichiers config
      \item Concevoir les tests en "pointant" vers ces définitions
      \item \textit{Construire un compilateur C -> C}
    \end{itemize}
  \end{block}
\end{frame}

%  .  .  .  .  .  .  .  .  .  .  .  .  .  .  .  .  .  .  %

\begin{frame}[fragile]{Aperçu}
  \begin{lstlisting}[style=CStyle, language=Python, gobble=4, caption={Hacl\_Bignum.json}]
    {"Meta_data":{
        "build" : "19-06-2025",
        "version" : "0.2.3"
    }
    ,"Hacl_Bignum_MontArithmetic_bn_mont_ctx_u32": {
        "len":"BUFFER_SIZE"
      ,"*n":"BUFFER_SIZE"
      ,"mu":"BUFFER_SIZE"
      ,"*r2":"BUFFER_SIZE"
      ,"BUFFER_SIZE":8
    }
    ,"Hacl_Bignum_MontArithmetic_bn_mont_ctx_u64": {
        "len":"BUFFER_SIZE"
      ,"*n":"BUFFER_SIZE"
      ,"mu":"BUFFER_SIZE"
      ,"*r2":"BUFFER_SIZE"
      ,"BUFFER_SIZE":8
    }}
  \end{lstlisting}
\end{frame}

%%%%%%%%%%%%%%%%%%%%%%%%%%%%%%%%%%%%%%%%%%%%%%%%%%%%%%%

\section{Compilation avec Érysichton}
\frame{\sectionpage}

\begin{frame}{Structure}
  \centering
  \begin{tikzpicture}[auto]

    % Styles
    \tikzstyle{startstop} = [rectangle, rounded corners, minimum width=2cm, minimum height=1cm, text centered, draw=black, fill=green!30]
    \tikzstyle{process} = [rectangle, minimum width=2cm, minimum height=1cm, text centered, draw=black, fill=orange!30]
    \tikzstyle{arrow} = [thick,->,>=stealth]
    \tikzset{zone1/.style={rectangle, rounded corners, draw=red, dashed, fill=red!10, inner sep=0.3cm}}
    \tikzset{zone2/.style={rectangle, rounded corners, draw=blue, dashed, fill=blue!10, inner sep=0.3cm}}
    \tikzset{zone22/.style={rectangle, rounded corners, draw=none, fill=blue!10, inner sep=0.3cm}}
    \tikzset{zone3/.style={rectangle, rounded corners, draw=green, dashed, fill=green!10, inner sep=0.3cm}}
    
    % Noeuds
    \node (hacl) [startstop] {Hacl*};
    \node (c) [below of=hacl] {.h};
    \node (ini) [below of=c, xshift=2cm] {.ini};
    \node (test) [below of=c, xshift=-2cm] {-test.c};
    \node (script) [below of=c] {.script};
    \node (compilateur) [process, below of=test] {Compilateur};
    \node (exe) [below of=compilateur] {-test.exe};
    \node (blanc1) [below of=script] {};
    \node (blanc2) [below of=blanc1] {};
    \node (gdb) [process, below of=blanc2] {GDB};
    \node (snap) [right of=gdb, xshift=2cm] {.snapshot};
    \node (binsec) [startstop, right of=snap, xshift=1.5cm] {Binsec};
    
    % Flèches
    \draw [arrow] (hacl) -- (c);
    \draw [arrow] (c) -- (ini);
    \draw [arrow] (c) -- (test);
    \draw [arrow] (c) -- (script);
    \draw [arrow] (test) -- (compilateur);
    \draw [arrow] (compilateur) -- (exe);
    \draw [arrow] (exe) -- (gdb);
    \draw [arrow] (script) -- (gdb);
    \draw [arrow] (gdb) -- (snap);
    \draw [arrow] (snap) -- (binsec);
    \draw [arrow] (ini) -- (binsec);

    % Zones
    \begin{scope}[on background layer]
        \onslide<1>{\node [zone1, fit=(c) (ini) (test) (script)] {};}
        \onslide<1>{\node [zone2, fit=(script) (gdb)] {};}
        \onslide<1>{\node [zone2, fit=(gdb) (snap) (binsec)] {};}
        \onslide<1>{\draw [zone22]
        ([xshift=-9pt, yshift=10pt]gdb.north west) --
        ([xshift=9pt, yshift=10pt]gdb.north east) -- 
        ([xshift=1pt, yshift=-1pt]gdb.south east) -- 
        ([xshift=-1pt, yshift=-1pt]gdb.south west) --
        cycle; }
        \onslide<1,2>{\node [zone3, fit=(compilateur) (exe)] {};}
    \end{scope}
    \end{tikzpicture}

\end{frame}

%  .  .  .  .  .  .  .  .  .  .  .  .  .  .  .  .  .  .  %

\begin{frame}[fragile]{Modèle de compilation}

  \begin{columns}
    \begin{column}{0.5\textwidth}
      \begin{block}{Réplication de la compilation de Hacl*}
        \large{
                  \dirtree{%
                    .1 Érysichton.
                    .2 Makefile.
                    .2 build.sh.
                    .2 x86\_64.
                    .3 Makefile{.}config.
                    .3 builds/.
                    .2 ARMV7.
                    .2 ARMV8.
                    .2 riscv-32.
                    .2 riscv-64.
                  }
                }
      \end{block}
    \end{column}
    \begin{column}{0.5\textwidth}
      \begin{lstlisting}[style=MakefileStyle, gobble=10, caption={Makefile}]
        ################################
        #        x86_64 section        #
        ################################

        x86_64:
          @./build.sh x86_64
          @make --no-print-directory -f .Makefile.generated all
      \end{lstlisting}
    \end{column}
  \end{columns}
\end{frame}

%  .  .  .  .  .  .  .  .  .  .  .  .  .  .  .  .  .  .  %

\begin{frame}[fragile]{Modèle de compilation}

  \begin{columns}
    \begin{column}{0.5\textwidth}
      \begin{block}{Réplication de la compilation de Hacl*}
        \large{
                  \dirtree{%
                    .1 Érysichton.
                    .2 Makefile.
                    .2 build.sh.
                    .2 x86\_64.
                    .3 Makefile{.}config.
                    .3 builds/.
                    .2 ARMV7.
                    .2 ARMV8.
                    .2 riscv-32.
                    .2 riscv-64.
                  }
                }
      \end{block}
    \end{column}
    \begin{column}{0.5\textwidth}
        \begin{lstlisting}[style=MakefileStyle, gobble=10, caption={Makefile.config}]
          DATA=/home/florian/Documents/recoules-hacl-star/hacl-star
          ARCHI=x86_64

          LDFLAGS="-Xlinker -z -Xlinker noexecstack -Xlinker --unresolved-symbols=report-all"

          HELP=gcc
          FORCE=Os
        \end{lstlisting}
    \end{column}
  \end{columns}
\end{frame}
      
%  .  .  .  .  .  .  .  .  .  .  .  .  .  .  .  .  .  .  %

\begin{frame}[fragile]{Modèle de compilation}

  \begin{columns}
    \begin{column}{0.4\textwidth}
      \begin{block}{Réplication de la compilation de Hacl*}
        \large{
                  \dirtree{%
                    .1 Érysichton.
                    .2 Makefile.
                    .2 build.sh.
                    .2 x86\_64.
                    .3 Makefile{.}config.
                    .3 builds/.
                    .2 ARMV7.
                    .2 ARMV8.
                    .2 riscv-32.
                    .2 riscv-64.
                  }
                }
      \end{block}
    \end{column}
    \begin{column}{0.6\textwidth}
        \begin{lstlisting}[style=MakefileStyle, gobble=10, caption={build.sh}]
          # Makefile automatically generated
          ARCHI := --target=x86_64
          FORCE := -Os \ CC := gcc
          CFLAGS := ... \ LDFLAGS := ...
          BUILD_DIR := ... \ COMPIL_DIR := ...
          SRC_DIR := make_tests/source/
          SRC_FILES := Hacl_AEAD_Chacha20Poly1305_decrypt.c Hacl_AEAD_Chacha20Poly1305_encrypt.c

          SOURCES := $(wildcard $(SRC_DIR)/*.c)
          OBJECTS := $(patsubst $(SRC_DIR)/%.c,$(BUILD_DIR)/%,$(SOURCES))
          EXECUTABLES := $(OBJECTS)

          all:
            echo "$(SOURCES)"

          %.exe: %.o
	          $(CC) $(CFLAGS) $(LDFLAGS) $^ ../dist/gcc-compatible/libevercrypt.a -static -o $@
        \end{lstlisting}
    \end{column}
  \end{columns}
\end{frame}

\begin{frame}{Reconceptualiser les intéractions}
  \begin{block}{Communication entre Modules}
    \begin{itemize}
      \item Fonctionnement pertinent
      \item Précision manuelle pour la compilation
      \item Activation de Binsec
    \end{itemize}
  \end{block}
\end{frame}

%%%%%%%%%%%%%%%%%%%%%%%%%%%%%%%%%%%%%%%%%%%%%%%%%%%%%%%

\section{Conclusion}
\frame{\sectionpage}

\begin{frame}{Conclusion}
  \begin{block}{Objectif}
    Finir le module x86\_64.
  \end{block}

  \begin{enumerate}
    \item Remplir les configurations
    \item Générer les tests
    \item Compiler les tests
    \item Analyser les tests
  \end{enumerate}
  

\end{frame}

%%%%%%%%%%%%%%%%%%%%%%%%%%%%%%%%%%%%%%%%%%%%%%%%%%%%%%%

%% Le texte est modifiable en changeant \thankyou
%% \renewcommand{\thankyou}{Thank You.}
\frame{\merci}


\end{document}

