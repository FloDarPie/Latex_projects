%%%%%%%%%%%%%%%%%%%%%%%%%%%%%%%%%%%%%%%%%%%%%%%%%%%%%%%%%%%%%%%%%%%%%%
%
%         Copyright (c) 2023, gitlabci_gallery / latex
%         All rights reserved.
%
%%%%%%%%%%%%%%%%%%%%%%%%%%%%%%%%%%%%%%%%%%%%%%%%%%%%%%%%%%%%%%%%%%%%%%

\documentclass[A4,svgnames,9pt,aspectratio=169]{beamer}
%% document options:
%% - aspectratio = { 43, 169, 1610 }
%% - utf8
%%


\setlength{\footskip}{300pt}
\usepackage[french]{babel}

\hypersetup{
   allcolors   = rouge_inria,
   pdfauthor   = {Duzés Florian},
   pdftitle    = {\@title},
   pdfsubject  = {Point hebdomadaire, bi-mensuel du stage},
   pdfkeywords = {entretien, observation du travail}
}

%%%%%%%%%%%%%%%%%%%%%%%%%%%%%%%%%%%%%%%%%%%%%%%%%%%%%%%
%%
%%%%%%%%%%%%%%%%%%%%%%%%%%%%%%%%%%%%%%%%%%%%%%%%%%%%%%%

\title[titrecourt]{Réunion flash}
\subtitle{Point hebdomadaire}
\date[14/05/2025]{date long}
\author[Duzes Florian]{Duzés Florian}

\usetheme{inria}

\begin{document}

%%%%%%%%%%%%%%%%%%%%%%%%%%%%%%%%%%%%%%%%%%%%%%%%%%%%%%%
%%
%%%%%%%%%%%%%%%%%%%%%%%%%%%%%%%%%%%%%%%%%%%%%%%%%%%%%%%

\frame{\titlepage}

%%%%%%%%%%%%%%%%%%%%%%%%%%%%%%%%%%%%%%%%%%%%%%%%%%%%%%%

% Le titre des planches de sommaire est \contentsname, sa valeur
% est fixée ici à "Sommaire" par défaut.
\renewcommand{\contentsname}{Sommaire}

\frame{\tocpage}


%%--%%--%%--%%--%%--%%--%%--%%--%%--%%--%%--%%--%%--%%--%%
 
\section{État des lieux}
\frame{\sectionpage}

\begin{frame}{Point actuel}
  \begin{tikzpicture}[
    node distance=2cm,
    box/.style={rectangle, draw=black, thick, minimum width=3cm, minimum height=1cm, text centered, rounded corners, font=\bfseries},
    arrow/.style={->, >=Stealth, thick, draw=arrowColor},
    contentBoxDone/.style={rectangle, draw=black, thick, fill=board_lightGray!40, rounded corners, minimum width=3cm, minimum height=2cm, text width=3cm, align=center},
    contentBox/.style={rectangle, draw=black, thick, rounded corners, minimum width=3cm, minimum height=2cm, text width=3cm, align=center}
]

    % DESSINS
    \node[box, fill=faitColor] (fait) {Fait};
    \node[box, fill=enCoursColor, right=of fait] (en_cours) {En cours};
    \node[box, fill=aFaireColor, right=of en_cours] (a_faire) {\parbox{3cm}{\centering Prévus \\ \tiny{Pour le 15/06}}};
    \draw[arrow] (fait) -- (en_cours);
    \draw[arrow] (en_cours) -- (a_faire);

    % Choses réalisées
    \node[contentBoxDone, below=0.5cm of fait] {
        \begin{itemize}
          \item Méthodologie x86\_64
          \item Schémas de compilation
        \end{itemize}
    };

    % En cours
    \node[contentBox, below=0.5cm of en_cours, fill=enCoursColor!15] {
        \begin{itemize}
          \item Méthodologie ARM
          \item Étendre au dossier \textit{/tests}
          \item Automatisation de x86\_64
        \end{itemize}
    };

    % Point fixer
    \node[contentBox, below=0.5cm of a_faire, fill=aFaireColor!15] {
        \begin{itemize}
            \item Automatisation des .ini
            \item Couvrir les tests
            \item Couvrir les primitives Hacl*
            \item RiskV - compil à la main de tests
        \end{itemize}
    };

      \end{tikzpicture}

\end{frame}

%  .  .  .  .  .  .  .  .  .  .  .  .  .  .  .  .  .  .  %

\begin{frame}{Réalisation}
  \begin{tikzpicture}[
    node distance=2cm,
    box/.style={rectangle, draw=black, thick, minimum width=3cm, minimum height=1cm, text centered, rounded corners, font=\bfseries},
    arrow/.style={->, >=Stealth, thick, draw=arrowColor},
    contentBoxDone/.style={rectangle, draw=black, thick, fill=board_lightGray!40, rounded corners, minimum width=3cm, minimum height=2cm, text width=3cm, align=center},
    contentBox/.style={rectangle, draw=black, thick, rounded corners, minimum width=3cm, minimum height=2cm, text width=3cm, align=center}
]

    % DESSINS
    \node[box, fill=faitColor] (fait) {Fait};
    \node[box, fill=enCoursColor, right=of fait] (en_cours) {En cours};
    \node[box, fill=aFaireColor, right=of en_cours] (a_faire) {\parbox{3cm}{\centering Prévus \\ \tiny{Pour le 15/06}}};
    \draw[arrow] (fait) -- (en_cours);
    \draw[arrow] (en_cours) -- (a_faire);

    % Choses réalisées
    \node[contentBoxDone, below=0.5cm of fait] {
        \begin{itemize}
          \item Méthodologie x86\_64
          \item Méthodologie ARM
          \item Schémas de compilation
        \end{itemize}
    };

    % En cours
    \node[contentBox, below=0.5cm of en_cours, fill=enCoursColor!15] {
        \begin{itemize}
          \item Automatisation de x86\_64
          \item Conception des fichiers tests
        \end{itemize}
    };

    % Point fixer
    \node[contentBox, below=0.5cm of a_faire, fill=aFaireColor!15] {
        \begin{itemize}
            \item Automatisation des .ini
            \item Couvrir les tests
            \item Couvrir les primitives Hacl*
            \item RiskV - compil à la main de tests
        \end{itemize}
    };

      \end{tikzpicture}

\end{frame}


%%--%%--%%--%%--%%--%%--%%--%%--%%--%%--%%--%%--%%--%%--%%

\section{Rappel des arbres}
\frame{\sectionpage}

\begin{frame}{Compilation x86\_64}
  \centering
  \begin{tikzpicture}[auto]

    % Styles
    \tikzstyle{startstop} = [rectangle, rounded corners, minimum width=2cm, minimum height=1cm, text centered, draw=black, fill=green!30]
    \tikzstyle{process} = [rectangle, minimum width=2cm, minimum height=1cm, text centered, draw=black, fill=orange!30]
    \tikzstyle{arrow} = [thick,->,>=stealth]
    \tikzset{zone1/.style={rectangle, rounded corners, draw=red, dashed, fill=red!10, inner sep=0.3cm}}
    \tikzset{zone2/.style={rectangle, rounded corners, draw=blue, dashed, fill=blue!10, inner sep=0.3cm}}
    \tikzset{zone22/.style={rectangle, rounded corners, draw=none, fill=blue!10, inner sep=0.3cm}}
    \tikzset{zone3/.style={rectangle, rounded corners, draw=green, dashed, fill=green!10, inner sep=0.3cm}}
    
    % Noeuds
    \node (hacl) [startstop] {Hacl*};
    \node (c) [below of=hacl] {.h};
    \node (ini) [below of=c, xshift=2cm] {.ini};
    \node (test) [below of=c, xshift=-2cm] {-test.c};
    \node (script) [below of=c] {.script};
    \node (compilateur) [process, below of=test] {Compilateur};
    \node (exe) [below of=compilateur] {-test.exe};
    \node (blanc1) [below of=script] {};
    \node (blanc2) [below of=blanc1] {};
    \node (gdb) [process, below of=blanc2] {GDB};
    \node (snap) [right of=gdb, xshift=2cm] {.snapshot};
    \node (binsec) [startstop, right of=snap, xshift=1.5cm] {Binsec};
    
    % Flèches
    \draw [arrow] (hacl) -- (c);
    \draw [arrow] (c) -- (ini);
    \draw [arrow] (c) -- (test);
    \draw [arrow] (c) -- (script);
    \draw [arrow] (test) -- (compilateur);
    \draw [arrow] (compilateur) -- (exe);
    \draw [arrow] (exe) -- (gdb);
    \draw [arrow] (script) -- (gdb);
    \draw [arrow] (gdb) -- (snap);
    \draw [arrow] (snap) -- (binsec);
    \draw [arrow] (ini) -- (binsec);

    % Zones
    \begin{scope}[on background layer]
        \node [zone1, fit=(c) (ini) (test) (script)] {};
        \node [zone2, fit=(script) (gdb)] {};
        \node [zone2, fit=(gdb) (snap) (binsec)] {};
        \draw [zone22]
        ([xshift=-9pt, yshift=10pt]gdb.north west) --
        ([xshift=9pt, yshift=10pt]gdb.north east) -- 
        ([xshift=1pt, yshift=-1pt]gdb.south east) -- 
        ([xshift=-1pt, yshift=-1pt]gdb.south west) --
        cycle; 
        \node [zone3, fit=(compilateur) (exe)] {};
    \end{scope}
    
    
    \end{tikzpicture}
\end{frame}

%  .  .  .  .  .  .  .  .  .  .  .  .  .  .  .  .  .  .  %

\begin{frame}{Compilation ARM}
  \centering
  \begin{tikzpicture}[auto]

    % Styles
    \tikzstyle{startstop} = [rectangle, rounded corners, minimum width=2cm, minimum height=1cm, text centered, draw=black, fill=green!30]
    \tikzstyle{process} = [rectangle, minimum width=2cm, minimum height=1cm, text centered, draw=black, fill=orange!30]
    \tikzstyle{arrow} = [thick,->,>=stealth]
    \tikzset{zone1/.style={rectangle, rounded corners, draw=red, dashed, fill=red!10, inner sep=0.3cm}}
    \tikzset{zone2/.style={rectangle, rounded corners, draw=blue, dashed, fill=blue!10, inner sep=0.3cm}}
    \tikzset{zone22/.style={rectangle, rounded corners, draw=none, fill=blue!10, inner sep=0.3cm}}
    \tikzset{zone3/.style={rectangle, rounded corners, draw=green, dashed, fill=green!10, inner sep=0.3cm}}
    
    % Noeuds
    \node (hacl) [startstop] {Hacl*};
    \node (c) [below of=hacl] {.h};
    \node (ini) [below of=c] {.ini};
    \node (test) [below of=c, xshift=-2cm] {-test.c};
    \node (compilateur) [process, below of=test] {Compilateur};
    \node (exe) [below of=compilateur] {-test.exe};
    \node (blanc) [below of=c] {};
    \node (blanc1) [below of=blanc] {};
    \node (blanc2) [below of=blanc1] {};
    \node (binsec) [startstop, below of=blanc2] {Binsec};
    
    % Flèches
    \draw [arrow] (hacl) -- (c);
    \draw [arrow] (c) -- (ini);
    \draw [arrow] (c) -- (test);
    \draw [arrow] (test) -- (compilateur);
    \draw [arrow] (compilateur) -- (exe);
    \draw [arrow] (exe) -- (binsec);
    \draw [arrow] (ini) -- (binsec);

    % Zones
    \begin{scope}[on background layer]
        \node [zone2, fit=(ini) (binsec)] {};
        \node [zone1, fit=(c) (ini) (test) ] {};
        \node [zone3, fit=(compilateur) (exe)] {};
    \end{scope}
    
    
    \end{tikzpicture}
\end{frame}

%%--%%--%%--%%--%%--%%--%%--%%--%%--%%--%%--%%--%%--%%--%%

\section{Construction de \textbf{Érysichthon}}
\frame{\sectionpage}


\begin{frame}{Structure générale}
  \hspace{3cm}
  \begin{tikzpicture}[auto, node distance=1.3cm]

    % Styles
    \tikzstyle{startstop} = [rectangle, minimum width=2cm, minimum height=1cm, text centered, draw=black, fill=green!60]
    \tikzstyle{process} = [rectangle, rounded corners,minimum width=1cm, minimum height=1cm, text centered, draw=black, fill=magenta!40]
    \tikzstyle{arrow} = [thick,->,>=stealth]
    
    % Noeuds
    \node (control) [startstop] {\parbox{3cm}{\centering Érysichthons \\ \tiny{Control panel}}};
    \node (x86) [process, below of=control, xshift= 2cm] {x86\_64};
    \node (arm) [process, below of=x86] {ARM};
    \node (test) [process, below of=arm] {Make\_test};

    % Flèches
    \draw [arrow] (control) -- ++(0,-1.3) -- (x86);
    \draw [arrow] (control) -- ++(0,-2.6) -- (arm);
    \draw [arrow] (control) -- ++(0,-3.9) -- (test);
    
    \end{tikzpicture}
\end{frame}

%  .  .  .  .  .  .  .  .  .  .  .  .  .  .  .  .  .  .  %

\begin{frame}{Module x86\_64}
  \hspace{3cm}
  \begin{tikzpicture}[auto]

    % Styles
    \tikzstyle{startstop} = [rectangle, minimum width=2cm, minimum height=1cm, text centered, draw=black, fill=green!60]
    \tikzstyle{process} = [rectangle, rounded corners,minimum width=1cm, minimum height=1cm, text centered, draw=black, fill=magenta!40]
    \tikzstyle{process2} = [rectangle, rounded corners, minimum width=0.8cm, minimum height=0.3cm, text centered, draw=black, fill=blue!30]
    \tikzstyle{arrow} = [thick,->,>=stealth]
    
    % Noeuds
    \node (control) [startstop] {\parbox{3cm}{\centering Érysichthons \\ \tiny{Control panel}}};
    \node (x86) [process, below of=control, xshift= 2cm] {x86\_64};
    \node (x86-ini) [process2, below of=x86, xshift = 2cm] {Make\_ini};
    \node (x86-dump) [process2, below of=x86-ini] {Make\_core\_dump};
    \node (x86-compil) [process2, below of=x86-dump] {Compil};
    \node (arm) [process, below of=x86-compil, xshift=-2cm] {ARM};
    \node (test) [process, below of=arm] {Make\_test};

    % Flèches
    \draw [arrow] (control) -- ++(0,-1) -- (x86);
    \draw [arrow] (control) -- ++(0,-5) -- (arm);
    \draw [arrow] (control) -- ++(0,-6) -- (test);
    \draw [arrow] (x86) -- ++(0,-1) -- (x86-ini);
    \draw [arrow] (x86) -- ++(0,-2) -- (x86-dump);
    \draw [arrow] (x86) -- ++(0,-3) -- (x86-compil);
    
    \end{tikzpicture}
\end{frame}
%  .  .  .  .  .  .  .  .  .  .  .  .  .  .  .  .  .  .  %

\begin{frame}{Module ARM}
  \hspace{3cm}
  \begin{tikzpicture}[auto]

    % Styles
    \tikzstyle{startstop} = [rectangle, minimum width=2cm, minimum height=1cm, text centered, draw=black, fill=green!60]
    \tikzstyle{process} = [rectangle, rounded corners,minimum width=1cm, minimum height=1cm, text centered, draw=black, fill=magenta!40]
    \tikzstyle{process2} = [rectangle, rounded corners, minimum width=0.8cm, minimum height=0.3cm, text centered, draw=black, fill=blue!30]
    \tikzstyle{arrow} = [thick,->,>=stealth]
    
    % Noeuds
    \node (control) [startstop] {\parbox{3cm}{\centering Érysichthons \\ \tiny{Control panel}}};
    \node (x86) [process, below of=control, xshift= 2cm] {x86\_64};
    \node (arm) [process, below of=x86] {ARM};
    \node (arm-ini) [process2, below of=arm, xshift=2cm] {Make\_ini};
    \node (arm-compil) [process2, below of=arm-ini] {Compil};
    \node (test) [process, below of=arm-compil, xshift=-2cm] {Make\_test};


    % Flèches
    \draw [arrow] (control) -- ++(0,-1) -- (x86);
    \draw [arrow] (control) -- ++(0,-2) -- (arm);
    \draw [arrow] (arm) -- ++(0,-1) -- (arm-ini);
    \draw [arrow] (arm) -- ++(0,-2) -- (arm-compil);
    \draw [arrow] (control) -- ++(0,-5) -- (test);

    \end{tikzpicture}
\end{frame}

%  .  .  .  .  .  .  .  .  .  .  .  .  .  .  .  .  .  .  %

\begin{frame}{Module Tests}
  \hspace{3cm}
  \begin{tikzpicture}[auto]

    % Styles
    \tikzstyle{startstop} = [rectangle, minimum width=2cm, minimum height=1cm, text centered, draw=black, fill=green!60]
    \tikzstyle{process} = [rectangle, rounded corners,minimum width=1cm, minimum height=1cm, text centered, draw=black, fill=magenta!40]
    \tikzstyle{process2} = [rectangle, rounded corners, minimum width=0.8cm, minimum height=0.3cm, text centered, draw=black, fill=blue!30]
    \tikzstyle{arrow} = [thick,->,>=stealth]
    
    % Noeuds
    \node (control) [startstop] {\parbox{3cm}{\centering Érysichthons \\ \tiny{Control panel}}};
    \node (x86) [process, below of=control, xshift= 2cm] {x86\_64};
    \node (arm) [process, below of=x86] {ARM};
    \node (test) [process, below of=arm] {Make\_test};
    \node (enum) [process2, below of=test, xshift=2cm] {enumerate};
    \node (parse) [process2, below of=enum] {parse};
    \node (exe) [process2, below of=parse] {Make\_tests};

    % Flèches
    \draw [arrow] (control) -- ++(0,-1) -- (x86);
    \draw [arrow] (control) -- ++(0,-2) -- (arm);
    \draw [arrow] (control) -- ++(0,-3) -- (test);
    \draw [arrow] (test) -- ++(0,-1) -- (enum);
    \draw [arrow] (test) -- ++(0,-2) -- (parse);
    \draw [arrow] (test) -- ++(0,-3) -- (exe);
    
    
    
    \end{tikzpicture}
\end{frame}

%%--%%--%%--%%--%%--%%--%%--%%--%%--%%--%%--%%--%%--%%--%%

\section{Focus sur le Module Tests}
\frame{\sectionpage}

%  .  .  .  .  .  .  .  .  .  .  .  .  .  .  .  .  .  .  %

\begin{frame}[fragile]{Makefile}
  \begin{lstlisting}[style=MakefileStyle, caption={Makefile}, gobble=4]
    DIRECTORY=$(HOME)/Documents/recoules-hacl-star/hacl-star/dist/gcc-compatible
    ENUM_FILE=liste
    GREP=Hacl

    all: enumerate target

    enumerate:
      @echo "Enumerate : $(DIRECTORY) - for library $(GREP)"
      @ls $(DIRECTORY)/*.h | grep $(GREP)| xargs -n 1 basename > $(ENUM_FILE)

    target:
      @echo "Targeted functions enumerated in : $(ENUM_FILE)"
      @rm -f tests/*
      @echo "Lancement du script\n----" && python3 enum_test.py $(DIRECTORY) $(ENUM_FILE)

    clean:
      @echo "Nettoyage des fichiers générés..."
      @rm -f $(ENUM_FILE) tests/*
  \end{lstlisting}
\end{frame}


%  .  .  .  .  .  .  .  .  .  .  .  .  .  .  .  .  .  .  %

\begin{frame}[fragile]{Résultat courant}

  \begin{center}
    \begin{lstlisting}[style=global, caption={liste}, gobble=6]
      Hacl_AEAD_Chacha20Poly1305.h
      Hacl_AEAD_Chacha20Poly1305_Simd128.h
      Hacl_AEAD_Chacha20Poly1305_Simd256.h
      Hacl_AES128.h
      Hacl_Bignum256_32.h
      ...
    \end{lstlisting}
  \end{center}
  \begin{columns}
  
    \begin{column}{0.5\textwidth}
        \begin{lstlisting}[style=global, caption={Hacl\_AEAD\_Chacha20Poly1305-Hacl\_AEAD\_Chacha20Poly1305\_decrypt}, gobble=8]
        uint32_t
        Hacl_AEAD_Chacha20Poly1305_decrypt
          uint8_t *output,   uint8_t *input,
          uint32_t input_len,   uint8_t *data,
          uint32_t data_len,   uint8_t *key,
          uint8_t *nonce,   uint8_t *tag
      \end{lstlisting}
      
    \end{column}

    \begin{column}{0.5\textwidth}
        \begin{lstlisting}[style=global, caption={Hacl\_AEAD\_Chacha20Poly1305-Hacl\_AEAD\_Chacha20Poly1305\_encrypt}, gobble=8]
        void
        Hacl_AEAD_Chacha20Poly1305_encrypt
          uint8_t *output,   uint8_t *tag,
          uint8_t *input,   uint32_t input_len,
          uint8_t *data,   uint32_t data_len,
          uint8_t *key,   uint8_t *nonce
      \end{lstlisting}
      
    \end{column}

  \end{columns}

\end{frame}

%  .  .  .  .  .  .  .  .  .  .  .  .  .  .  .  .  .  .  %

\begin{frame}[fragile]{Objectif}

  \begin{columns}
  
    \begin{column}{0.37\textwidth}
        \begin{lstlisting}[style=global, caption={Hacl\_AEAD\_Chacha20Poly1305\_decrypt}, gobble=8]
        uint32_t
        Hacl_AEAD_Chacha20Poly1305_decrypt
          uint8_t *output,   uint8_t *input,
          uint32_t input_len,   uint8_t *data,
          uint32_t data_len,   uint8_t *key,
          uint8_t *nonce,   uint8_t *tag
      \end{lstlisting}
      
    \end{column}

    \begin{column}{0.7\textwidth}
        \begin{lstlisting}[style=CStyle, caption={Hacl\_AEAD\_Chacha20Poly1305\_Simd128\_encrypt-test.c}, captionpos=t, gobble=8]
        #include <stdlib.h>
        #include "Hacl_AEAD_Chacha20Poly1305_Simd128.h"
        #define BUF_SIZE 16384
        #define KEY_SIZE 32
        #define NONCE_SIZE 12
        #define AAD_SIZE 12
        #define TAG_SIZE 16
        uint8_t plain[BUF_SIZE];uint8_t cipher[BUF_SIZE];
        uint8_t aead_key[KEY_SIZE];uint8_t aead_nonce[NONCE_SIZE];
        uint8_t aead_aad[AAD_SIZE];uint8_t tag[16];
        int main (int argc, char *argv[]){
          Hacl_AEAD_Chacha20Poly1305_Simd128_encrypt
            (cipher, tag, plain, BUF_SIZE, aead_aad, AAD_SIZE, aead_key, aead_nonce);
          exit(0);}
      \end{lstlisting}
    \end{column}

  \end{columns}

\end{frame}


%%--%%--%%--%%--%%--%%--%%--%%--%%--%%--%%--%%--%%--%%--%%

\section{Conclusion}
\frame{\sectionpage}

\begin{frame}{Conclusion}
  \begin{block}{Automatisation}
    \begin{itemize}
      \item Continuer la génération des fichiers \textit{-test.c}
      \item Activer la chaîne de compilation
      \pause
      \item \textit{Une interface graphique ?}
    \end{itemize}
  \end{block}

\end{frame}

%%%%%%%%%%%%%%%%%%%%%%%%%%%%%%%%%%%%%%%%%%%%%%%%%%%%%%%

%% Le texte est modifiable en changeant \thankyou
%% \renewcommand{\thankyou}{Thank You.}
\frame{\merci}


\end{document}

