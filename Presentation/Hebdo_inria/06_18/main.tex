%%%%%%%%%%%%%%%%%%%%%%%%%%%%%%%%%%%%%%%%%%%%%%%%%%%%%%%%%%%%%%%%%%%%%%
%
%         Copyright (c) 2023, gitlabci_gallery / latex
%         All rights reserved.
%
%%%%%%%%%%%%%%%%%%%%%%%%%%%%%%%%%%%%%%%%%%%%%%%%%%%%%%%%%%%%%%%%%%%%%%

\documentclass[A4,svgnames,9pt,aspectratio=169]{beamer}
%% document options:
%% - aspectratio = { 43, 169, 1610 }
%% - utf8
%%


\setlength{\footskip}{300pt}
\usepackage[french]{babel}

\hypersetup{
   allcolors   = rouge_inria,
   pdfauthor   = {Duzés Florian},
   pdftitle    = {\@title},
   pdfsubject  = {Point hebdomadaire, bi-mensuel du stage},
   pdfkeywords = {entretien, observation du travail}
}

%%%%%%%%%%%%%%%%%%%%%%%%%%%%%%%%%%%%%%%%%%%%%%%%%%%%%%%
%%
%%%%%%%%%%%%%%%%%%%%%%%%%%%%%%%%%%%%%%%%%%%%%%%%%%%%%%%

\title[titrecourt]{Réunion flash}
\subtitle{Point hebdomadaire}
\date[18/06/2025]{date long}
\author[Duzes Florian]{Duzés Florian}

\usetheme{inria}

\begin{document}

%%%%%%%%%%%%%%%%%%%%%%%%%%%%%%%%%%%%%%%%%%%%%%%%%%%%%%%
%%
%%%%%%%%%%%%%%%%%%%%%%%%%%%%%%%%%%%%%%%%%%%%%%%%%%%%%%%

\frame{\titlepage}

%%%%%%%%%%%%%%%%%%%%%%%%%%%%%%%%%%%%%%%%%%%%%%%%%%%%%%%

% Le titre des planches de sommaire est \contentsname, sa valeur
% est fixée ici à "Sommaire" par défaut.
\renewcommand{\contentsname}{Sommaire}

\frame{\tocpage}


%%--%%--%%--%%--%%--%%--%%--%%--%%--%%--%%--%%--%%--%%--%%
 
\section{État des lieux}
\frame{\sectionpage}

\begin{frame}{Général information}
  \begin{block}{Information exterieur}
    \begin{itemize}
      \item Entretien avec Maria Mishtaq
      \item Réunion Mercredi 2 juillet - 15h30
    \end{itemize}
    
  \end{block}
\end{frame}

%  .  .  .  .  .  .  .  .  .  .  .  .  .  .  .  .  .  .  %

\begin{frame}{Point actuel}
  \begin{tikzpicture}[
    node distance=2cm,
    box/.style={rectangle, draw=black, thick, minimum width=3cm, minimum height=1cm, text centered, rounded corners, font=\bfseries},
    arrow/.style={->, >=Stealth, thick, draw=arrowColor},
    contentBoxDone/.style={rectangle, draw=black, thick, fill=board_lightGray!40, rounded corners, minimum width=3cm, minimum height=2cm, text width=3cm, align=center},
    contentBox/.style={rectangle, draw=black, thick, rounded corners, minimum width=3cm, minimum height=2cm, text width=3cm, align=center}
]

    % DESSINS
    \node[box, fill=faitColor] (fait) {Fait};
    \node[box, fill=enCoursColor, right=of fait] (en_cours) {En cours};
    \node[box, fill=aFaireColor, right=of en_cours] (a_faire) {Prévus};
    \draw[arrow] (fait) -- (en_cours);
    \draw[arrow] (en_cours) -- (a_faire);

    % Choses réalisées
    \node[contentBoxDone, below=0.5cm of fait] {
        \begin{itemize}
          \item Toolchains RiscV
          \item Tests p256 avec gcc RiscV
        \end{itemize}
    };

    % En cours
    \node[contentBox, below=0.5cm of en_cours, fill=enCoursColor!15] {
        \begin{itemize}
          \item test p256 avec d'autres compilateurs
          \item Compiler Hacl* vers RiscV
          \item Test Binsec sur RiscV
          \item -test.json
        \end{itemize}
    };

    % Point fixer
    \node[contentBox, below=0.5cm of a_faire, fill=aFaireColor!15] {
        \begin{itemize}
            \item Chaîne de bout en bout 
            \item Couverture des architectures différentes
            \item Couverture des compilateurs
        \end{itemize}
    };
      \end{tikzpicture}

\end{frame}

%  .  .  .  .  .  .  .  .  .  .  .  .  .  .  .  .  .  .  %

\begin{frame}{Réalisation}
  \begin{tikzpicture}[
    node distance=2cm,
    box/.style={rectangle, draw=black, thick, minimum width=3cm, minimum height=1cm, text centered, rounded corners, font=\bfseries},
    arrow/.style={->, >=Stealth, thick, draw=arrowColor},
    contentBoxDone/.style={rectangle, draw=black, thick, fill=board_lightGray!40, rounded corners, minimum width=3cm, minimum height=2cm, text width=3cm, align=center},
    contentBox/.style={rectangle, draw=black, thick, rounded corners, minimum width=3cm, minimum height=2cm, text width=3cm, align=center}
]

    % DESSINS
    \node[box, fill=faitColor] (fait) {Fait};
    \node[box, fill=enCoursColor, right=of fait] (en_cours) {En cours};
    \node[box, fill=aFaireColor, right=of en_cours] (a_faire) {Prévus};
    \draw[arrow] (fait) -- (en_cours);
    \draw[arrow] (en_cours) -- (a_faire);

    % Choses réalisées
    \node[contentBoxDone, below=0.5cm of fait] {
        \begin{itemize}
          \item Étendu test de p256 avec LLVM / 32 / 64
          \item -test.json / generic json
          \item Hacl* -> RiscV
        \end{itemize}
    };

    % En cours
    \node[contentBox, below=0.5cm of en_cours, fill=enCoursColor!15] {
        \begin{itemize}
          \item Compiler Hacl* vers RiscV
          \item Test Binsec sur RiscV
          \item -test.json
        \end{itemize}
    };

    % Point fixer
    \node[contentBox, below=0.5cm of a_faire, fill=aFaireColor!15] {
        \begin{itemize}
            \item Chaîne de bout en bout 
            \item Couverture des architectures différentes
            \item Couverture des compilateurs
        \end{itemize}
    };
      \end{tikzpicture}

\end{frame}


%%--%%--%%--%%--%%--%%--%%--%%--%%--%%--%%--%%--%%--%%--%%

\section{Compilation de RiscV}
\frame{\sectionpage}

\begin{frame}{Préambule}

  \begin{block}{Informations générales}
    \begin{itemize}
      \item Construction de la toolchain LLVM
      \begin{itemize}
        \item \textit{GCC}
        \item \textit{LLVM}
        \begin{itemize}
          \item Architecure 64bits : \textbf{--with-arch=rv64gc --with-abi=lp64d}
          \item - 32bits : \textbf{--with-arch=rv32gc --with-abi=lp32d}
        \end{itemize}
      \end{itemize}
      \item Niveau d'optimisation testé
      \begin{itemize}
        \item \textit{-O0, -O1, -O2, -O3, -Oz, -Os}
      \end{itemize}
      \item Code analysé de \textit{cmovznz4}
    \end{itemize}
    
  \end{block}

\end{frame}

%  .  .  .  .  .  .  .  .  .  .  .  .  .  .  .  .  .  .  %

\begin{frame}{Compilation vers riscV-64}
\begin{columns}
  
  \begin{column}{0.4\textwidth}
    \begin{center}

  \begin{tabular}{|c|c|c|}
    \hline
     & \textbf{GCC} & \textbf{CLANG+LLVM} \\
    \hline
    -O0 & \~{} & \~{} \\
    \hline
    -O1 & \checkmark & \onslide<1,2,3>{X}\onslide<4>{\checkmark} \\
    \hline
    -O2 & \checkmark & \onslide<1,2,3>{X}\onslide<4>{\checkmark} \\
    \hline
    -O3 & \checkmark & \onslide<1,2,3>{X}\onslide<4>{\checkmark} \\
    \hline
    -Os & \checkmark & \onslide<1,2,3>{X}\onslide<4>{\checkmark} \\
    \hline
    -Oz & \checkmark & \onslide<1,2,3>{X}\onslide<4>{\checkmark} \\
    \hline
    \end{tabular}
  \end{center}
  \end{column}
  \begin{column}{0.6\textwidth}
    \begin{itemize}
      \pause
    \item \textbf{-O0} error - Binsec ISA definition
    \item Clang error - \textbf{beqz}
    \pause
    \item Passage \textit{InstCombinePass}
    \item patch : \textit{\# pragma clang optimise <off/on>}
  \end{itemize}
  
\end{column}

\end{columns}
  
\end{frame}

%  .  .  .  .  .  .  .  .  .  .  .  .  .  .  .  .  .  .  %

\begin{frame}{Compilation vers riscV-32}
\begin{columns}
  
  \begin{column}{0.4\textwidth}
    \begin{center}

  \begin{tabular}{|c|c|c|}
    \hline
     & \textbf{GCC} & \textbf{CLANG+LLVM} \\
    \hline
    -O0 & \~{} & \onslide<1>{\checkmark}\onslide<2>{\checkmark} \\
    \hline
    -O1 & \checkmark & \onslide<1>{X}\onslide<2>{\checkmark}  \\
    \hline
    -O2 & \checkmark & \onslide<1>{X}\onslide<2>{\checkmark} \\
    \hline
    -O3 & \checkmark & \onslide<1>{X}\onslide<2>{\checkmark} \\
    \hline
    -Os & \checkmark & \onslide<1>{X}\onslide<2>{\checkmark} \\
    \hline
    -Oz & \checkmark & \onslide<1>{X}\onslide<2>{\checkmark} \\
    \hline
    \end{tabular}
  \end{center}
  \end{column}
  \begin{column}{0.6\textwidth}
    \begin{itemize}
    \item \textbf{Gcc} error - Binsec ISA definition
    \item Clang error - \textbf{beqz}
  \end{itemize}
  
\end{column}

\end{columns}
\end{frame}

%%%%%%%%%%%%%%%%%%%%%%%%%%%%%%%%%%%%%%%%%%%%%%%%%%%%%%%

\section{Compilation croisé HACL*}
\frame{\sectionpage}

\begin{frame}{HACL* compilation}

  \begin{block}{Cible}
    \begin{columns}
      \begin{column}{0.5\textwidth}
        \begin{itemize}
          \item aarch64-none-linux-android
          \item aarch64-none-linux-gnu
          \item aarch64-apple-darwin
          \item aarch64-apple-ios
          \item x86\_64-apple-ios-simulator
        \end{itemize}
    \end{column}
    \begin{column}{0.5\textwidth}
      \begin{itemize}
        \item \textit{riscv64-unknown-linux-gnu}
      \end{itemize}
    \end{column}
  \end{columns}
  \end{block}

\end{frame}

%%%%%%%%%%%%%%%%%%%%%%%%%%%%%%%%%%%%%%%%%%%%%%%%%%%%%%%


\section{Érisychton v2}
\frame{\sectionpage}

\begin{frame}{Architecture reconstruite}
  \begin{block}{Amicalement débogable}
    \begin{columns}
      \begin{column}{0.5\textwidth}
        \begin{minipage}[t][4cm][t]{\linewidth}
          \large{
            \dirtree{%
              .1 Érysichton.
              .2 make\_test.
              .3 source.
              .4 test1{.}c.
              .4 test2{.}c.
              .4 \ldots.
              .3 make\_test{.}py.
              .3 utils{.}py.
              .3 config{.}json.
              .3 matching{.}json.
            }
          }
        \end{minipage}
      \end{column}

      \begin{column}{0.5\textwidth}
        \begin{minipage}[t][4cm][t]{\linewidth}
          \large{
            \dirtree{%
              .1 Érysichton.
              .2 andhrimnir (make\_test).
              .3 source.
              .4 test1{.}c.
              .4 test1{.}json.
              .4 \ldots.
              .3 config.
              .4 (header){.}json.
              .3 make\_test{.}py.
              .3 lib.
              .4 lexer{.}py.
              .4 backend{.}py.
              .4 utils{.}py.
              .4 \ldots.
            }
          }
        \end{minipage}
      \end{column}
    \end{columns}
  \end{block}
\end{frame}

%  .  .  .  .  .  .  .  .  .  .  .  .  .  .  .  .  .  .  %

\begin{frame}[fragile]{Fabrication des json}
  \begin{lstlisting}[style=CStyle, gobble=4, caption={Hacl\_Curve25519\_64.json}]
    {
    "Meta_data":{
        "build" : "17-06-2025",
        "version" : "0.2.0"
    }

    ,"Hacl_Curve25519_64_scalarmult": {
        "*out":""
      ,"*priv":""
      ,"*pub":""
    }

    ,"Hacl_Curve25519_64_secret_to_public": {
        "*pub":""
      ,"*priv":""
    }

    ,"Hacl_Curve25519_64_ecdh": {
        "*out":""
      ,"*priv":""
      ,"*pub":""
    }
    }
  \end{lstlisting}
\end{frame}

%  .  .  .  .  .  .  .  .  .  .  .  .  .  .  .  .  .  .  %

\begin{frame}[fragile]{Remplissage des json}
  \begin{columns}
    \begin{column}{0.5\textwidth}
      \begin{lstlisting}[style=CStyle, gobble=8, caption={Hacl\_AEAD\_Chacha20Poly1305\_Simd128.json (1)}]
        {
        "Meta_data":{
            "build" : "13-06-2025",
            "version" : "0.2.0"
        }
    
        ,"Hacl_AEAD_Chacha20Poly1305_Simd128_encrypt": {
            "*output":"BUF_SIZE"
          ,"*input":"BUF_SIZE"
          ,"input_len":"BUF_SIZE"
          ,"*data":"AAD_SIZE"
          ,"data_len":"AAD_SIZE"
          ,"*key":"KEY_SIZE"
          ,"*nonce":"NONCE_SIZE"
          ,"*tag":"TAG_SIZE"
          ,"BUF_SIZE":16384
          ,"TAG_SIZE":16
          ,"AAD_SIZE":12
          ,"KEY_SIZE":32
          ,"NONCE_SIZE":12
        }
      \end{lstlisting}
    \end{column}

    \begin{column}{0.5\textwidth}
      \begin{lstlisting}[style=CStyle, gobble=8, caption={Hacl\_AEAD\_Chacha20Poly1305\_Simd128.json (2)}]
        ,"Hacl_AEAD_Chacha20Poly1305_Simd128_decrypt": {
            "*output":"BUF_SIZE"
          ,"*input":"BUF_SIZE"
          ,"input_len":"BUF_SIZE"
          ,"*data":"AAD_SIZE"
          ,"data_len":"AAD_SIZE"
          ,"*key":"KEY_SIZE"
          ,"*nonce":"NONCE_SIZE"
          ,"*tag":"TAG_SIZE"
          ,"BUF_SIZE":16384
          ,"TAG_SIZE":16
          ,"AAD_SIZE":12
          ,"KEY_SIZE":32
          ,"NONCE_SIZE":12
          }
        }
      \end{lstlisting}
    \end{column}
  \end{columns}
\end{frame}

%  .  .  .  .  .  .  .  .  .  .  .  .  .  .  .  .  .  .  %

\begin{frame}[fragile]{Construction des tests}
  \begin{lstlisting}[style=CStyle, gobble=4, caption={Hacl\_AEAD\_Chacha20Poly1305\_Simd128.json}]
    //
    // Made by
    // ANDHRÍMNIR - 0.2.2
    // 17-06-2025
    //

    #include <stdlib.h>
    #include "Hacl_AEAD_Chacha20Poly1305.h"

    #define tag TAG_SIZE
    #define output BUF_SIZE
    #define data AAD_SIZE
    #define nonce NONCE_SIZE
    #define key KEY_SIZE
    #define input BUF_SIZE

    #define BUF_SIZE 16384
    #define AAD_SIZE 12
    #define TAG_SIZE 16
    #define NONCE_SIZE 12
    #define KEY_SIZE 32

    uint8_t output[BUF_SIZE];    uint8_t tag[TAG_SIZE];
    uint8_t input[BUF_SIZE];     uint8_t data[AAD_SIZE];
    uint8_t key[KEY_SIZE];       uint8_t nonce[NONCE_SIZE];

    int main (int argc, char *argv[]){
    Hacl_AEAD_Chacha20Poly1305_encrypt(output, tag, input, BUF_SIZE, data, AAD_SIZE, key, nonce);
      exit(0);
    }
  \end{lstlisting}
\end{frame}

\section{Conclusion}
\frame{\sectionpage}

\begin{frame}{Conclusion}
  \begin{block}{Objectif}
    \begin{center}
      Finir le module x86\_64.
    \end{center}
  \end{block}
\pause
  \begin{enumerate}
    \item Remplir les configurations
    \item Générer les tests
    \item Compiler les tests
    \item Analyser les tests
  \end{enumerate}
  

\end{frame}

%%%%%%%%%%%%%%%%%%%%%%%%%%%%%%%%%%%%%%%%%%%%%%%%%%%%%%%

\frame{\merci}


\end{document}

