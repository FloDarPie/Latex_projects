%%%%%%%%%%%%%%%%%%%%%%%%%%%%%%%%%%%%%%%%%%%%%%%%%%%%%%%%%%%%%%%%%%%%%%
%
%         Copyright (c) 2023, gitlabci_gallery / latex
%         All rights reserved.
%
%%%%%%%%%%%%%%%%%%%%%%%%%%%%%%%%%%%%%%%%%%%%%%%%%%%%%%%%%%%%%%%%%%%%%%

\documentclass[A4,svgnames,9pt,aspectratio=169]{beamer}
%% document options:
%% - aspectratio = { 43, 169, 1610 }
%% - utf8
%%


\setlength{\footskip}{300pt}
\usepackage[french]{babel}

\hypersetup{
   allcolors   = rouge_inria,
   pdfauthor   = {Duzes Florian},
   pdftitle    = {\@title},
   pdfsubject  = {Point hebdomadaire, bi-mensuel du stage},
   pdfkeywords = {entretien, observation du travail}
}

%%%%%%%%%%%%%%%%%%%%%%%%%%%%%%%%%%%%%%%%%%%%%%%%%%%%%%%
%%
%%%%%%%%%%%%%%%%%%%%%%%%%%%%%%%%%%%%%%%%%%%%%%%%%%%%%%%

\title[titrecourt]{Réunion flash}
\subtitle{Point hebdomadaire}
\date[22/05/2025]{date long}
\author[Duzes Florian]{Duzés Florian}

\usetheme{inria}

\begin{document}

%%%%%%%%%%%%%%%%%%%%%%%%%%%%%%%%%%%%%%%%%%%%%%%%%%%%%%%
%%
%%%%%%%%%%%%%%%%%%%%%%%%%%%%%%%%%%%%%%%%%%%%%%%%%%%%%%%

\frame{\titlepage}

%%%%%%%%%%%%%%%%%%%%%%%%%%%%%%%%%%%%%%%%%%%%%%%%%%%%%%%

% Le titre des planches de sommaire est \contentsname, sa valeur
% est fixée ici à "Sommaire" par défaut.
\renewcommand{\contentsname}{Sommaire}

\frame{\tocpage}


%%--%%--%%--%%--%%--%%--%%--%%--%%--%%--%%--%%--%%--%%--%%
 
\section{État des lieux}
\frame{\sectionpage}

\begin{frame}{Point actuel}
  \begin{tikzpicture}[
    node distance=2cm,
    box/.style={rectangle, draw=black, thick, minimum width=3cm, minimum height=1cm, text centered, rounded corners, font=\bfseries},
    arrow/.style={->, >=Stealth, thick, draw=arrowColor},
    contentBoxDone/.style={rectangle, draw=black, thick, fill=board_lightGray!40, rounded corners, minimum width=3cm, minimum height=2cm, text width=3cm, align=center},
    contentBox/.style={rectangle, draw=black, thick, rounded corners, minimum width=3cm, minimum height=2cm, text width=3cm, align=center}
]

    % DESSINS
    \node[box, fill=faitColor] (fait) {Fait};
    \node[box, fill=enCoursColor, right=of fait] (en_cours) {En cours};
    \node[box, fill=aFaireColor, right=of en_cours] (a_faire) {\parbox{3cm}{\centering Prévus \\ \tiny{Pour le 15/06}}};
    \draw[arrow] (fait) -- (en_cours);
    \draw[arrow] (en_cours) -- (a_faire);

    % Choses réalisées
    \node[contentBoxDone, below=0.5cm of fait] {
        \begin{itemize}
          \item Méthodologie x86\_64 
          \item Méthodologie ARM
          \item Schémas de compilation
        \end{itemize}
    };

    % En cours
    \node[contentBox, below=0.5cm of en_cours, fill=enCoursColor!15] {
        \begin{itemize}
          \item Automatisation des \textit{-test.c}
          \item Préparation d'une config pour la génération des tests
          \item Vérification de \textbf{p256} en ARM
        \end{itemize}
    };

    % Point fixer
    \node[contentBox, below=0.5cm of a_faire, fill=aFaireColor!15] {
        \begin{itemize}
            \item Automatisation des .ini
            \item Couvrir les tests
            \item Couvrir les primitives Hacl*
            \item RiskV - compil à la main de tests
        \end{itemize}
    };

      \end{tikzpicture}

\end{frame}

%  .  .  .  .  .  .  .  .  .  .  .  .  .  .  .  .  .  .  %

\begin{frame}{Réalisation}
  \begin{tikzpicture}[
    node distance=2cm,
    box/.style={rectangle, draw=black, thick, minimum width=3cm, minimum height=1cm, text centered, rounded corners, font=\bfseries},
    arrow/.style={->, >=Stealth, thick, draw=arrowColor},
    contentBoxDone/.style={rectangle, draw=black, thick, fill=board_lightGray!40, rounded corners, minimum width=3cm, minimum height=2cm, text width=3cm, align=center},
    contentBox/.style={rectangle, draw=black, thick, rounded corners, minimum width=3cm, minimum height=2cm, text width=3cm, align=center}
]

    % DESSINS
    \node[box, fill=faitColor] (fait) {Fait};
    \node[box, fill=enCoursColor, right=of fait] (en_cours) {En cours};
    \node[box, fill=aFaireColor, right=of en_cours] (a_faire) {\parbox{3cm}{\centering Prévus \\ \tiny{Pour le 15/06}}};
    \draw[arrow] (fait) -- (en_cours);
    \draw[arrow] (en_cours) -- (a_faire);

    % Choses réalisées
    \node[contentBoxDone, below=0.5cm of fait] {
        \begin{itemize}
          \item Chaîne de compilation
          \item Identifications d'erreur de \textbf{p256}
        \end{itemize}
    };

    % En cours
    \node[contentBox, below=0.5cm of en_cours, fill=enCoursColor!15] {
        \begin{itemize}
          \item Automatisation des \textit{-test.c}
          \item Préparation d'une config pour la génération des tests
        \end{itemize}
    };

    % Point fixer
    \node[contentBox, below=0.5cm of a_faire, fill=aFaireColor!15] {
        \begin{itemize}
            \item Automatisation des .ini
            \item Couvrir les tests
            \item Couvrir les primitives Hacl*
            \item RiskV - compil à la main de tests
        \end{itemize}
    };

      \end{tikzpicture}

\end{frame}


%%--%%--%%--%%--%%--%%--%%--%%--%%--%%--%%--%%--%%--%%--%%

\section{Couverture de Hacl*}
\frame{\sectionpage}

\begin{frame}[fragile]{Fabrication des tests}
  \begin{block}{Premiers scripts pondus}
    \tiny{La semaine dernière}
      \begin{lstlisting}[style=global, caption={Hacl\_AEAD\_Chacha20Poly1305\_decrypt}, gobble=8, label=lst:lastweek]
        uint32_t
          Hacl_AEAD_Chacha20Poly1305_decrypt
          uint8_t *output,   uint8_t *input,
          uint32_t input_len,   uint8_t *data,
          uint32_t data_len,   uint8_t *key,
          uint8_t *nonce,   uint8_t *tag
      \end{lstlisting}
  \end{block}
\end{frame}
%  .  .  .  .  .  .  .  .  .  .  .  .  .  .  .  .  .  .  %

\begin{frame}[fragile]{Fabrication des tests}
  \begin{block}{Premiers scripts pondus}
      \begin{lstlisting}[style=CStyle, caption={Hacl\_AEAD\_Chacha20Poly1305\_decrypt.c}, gobble=8]
        #include <stdlib.h>
        #include Hacl_AEAD_Chacha20Poly1305.h
        #define BUF_SIZE 16384
        #define KEY_SIZE 32
        #define TAG_SIZE 16
        #define AAD_SIZE 12
        #define NONCE_SIZE 12
        uint8_t output[BUF_SIZE];
        uint8_t input[BUF_SIZE];
        uint8_t data[AAD_SIZE];
        uint8_t key[KEY_SIZE];
        uint8_t nonce[NONCE_SIZE];
        uint8_t tag[TAG_SIZE];
        int main (int argc, char *argv[]){
        uint32_t a =    Hacl_AEAD_Chacha20Poly1305_decrypt
              (output, input, BUF_SIZE, data, AAD_SIZE, key, nonce, tag);
          exit(0);
        }
      \end{lstlisting}
  \end{block}
\end{frame}

%  .  .  .  .  .  .  .  .  .  .  .  .  .  .  .  .  .  .  %

\begin{frame}{Chaîne de fabrication}
  \centering
  \begin{tikzpicture}[auto]

    % Styles
    \tikzstyle{startstop} = [rectangle, minimum width=2cm, minimum height=1cm, text centered, draw=black, fill=green!60]
    \tikzstyle{process} = [rectangle, rounded corners,minimum width=1cm, minimum height=1cm, text centered, draw=black, fill=magenta!40]
    \tikzstyle{process2} = [rectangle, rounded corners, minimum width=0.8cm, minimum height=0.3cm, text centered, draw=black, fill=blue!30]
    \tikzstyle{arrow} = [thick,->,>=stealth]
    
    % Noeuds
    \node (control) [startstop] {\parbox{3cm}{\centering Érysichthons \\ \tiny{Control panel}}};
    \node (blank) [below of=control] {};
    \node (test) [process, below of=blank, xshift=2cm] {Make\_test};
    \node (enum) [process2, below of=test, xshift=2cm] {enumerate};
    \node (parse) [process2, below of=enum] {parse};
    \node (make) [process2, below of=parse] {Make\_tests};
    \onslide<2>{\node (config) [process2, fill=orange!30, below of=make] {matching.json};}

    % Flèches
    \draw [arrow] (control) -- ++(0,-2) -- (test);
    \draw [arrow] (test) -- ++(0,-1) -- (enum);
    \draw [arrow] (test) -- ++(0,-2) -- (parse);
    \draw [arrow] (test) -- ++(0,-3) -- (make);
    \onslide<2>{\draw [dashed, ->, thick, >=stealth] (test) -- ++(0,-4) -- (config);}

    \end{tikzpicture}

\end{frame}

%  .  .  .  .  .  .  .  .  .  .  .  .  .  .  .  .  .  .  %

\begin{frame}[fragile]{Phase de transformation}
 
  \begin{tikzpicture}[auto, remember picture, overlay]
    \tikzstyle{startstop} = [rectangle, minimum width=2cm, minimum height=1cm, text centered, draw=black, fill=green!60]
    \tikzstyle{arrow} = [thick,->,>=stealth, line width=1.5pt]
    
    % Noeuds
    \node (raw) at (3,-2.9)  {};
    \node (matching) at (6.8,-3.8) {};
    \node (matching2) at (6.8,-4.2) {};
    \node (final) at (3,-5.5) {};
    
    \onslide<2,3>{\draw [arrow, bend right, color=blue!40] (raw) to (matching);}
    \onslide<3>{\draw [arrow, bend right, color=blue!60] (matching2) to (final);}
    
  \end{tikzpicture}

  \begin{columns}
    \begin{column}{0.5\textwidth}
      \begin{lstlisting}[style=global, caption={Hacl\_AEAD\_Chacha20Poly1305\_decrypt}, gobble=8]
        uint32_t
          Hacl_AEAD_Chacha20Poly1305_decrypt
          uint8_t *output,   uint8_t *input,
          uint32_t input_len,   uint8_t *data,
          uint32_t data_len,   uint8_t *key,
          uint8_t *nonce,   uint8_t *tag
      \end{lstlisting}
      \vspace{2cm}
      \onslide<3>{Hacl\_AEAD\_Chacha20Poly1305\_decrypt.c}
    \end{column}


    \begin{column}{0.5\textwidth}
      \pause
      \begin{lstlisting}[style=CStyle, caption={matching.json}]
{
  "input":"BUF_SIZE",
  "input_len":"BUF_SIZE",
  "output":"BUF_SIZE",
  "output_len":"BUF_SIZE",
  "key":"KEY_SIZE",
  "nonce":"NONCE_SIZE",
  "tag":"TAG_SIZE",
  "data":"AAD_SIZE",
  "data_len":"AAD_SIZE",
  "BUF_SIZE":16384,
  "KEY_SIZE":32,
  "NONCE_SIZE":12,
  "AAD_SIZE":12,
  "TAG_SIZE":16
}
      \end{lstlisting}
  
    \end{column}
  \end{columns}

\end{frame}

%%--%%--%%--%%--%%--%%--%%--%%--%%--%%--%%--%%--%%--%%--%%

\section{Reproduction de bug}
\frame{\sectionpage}

\begin{frame}{Etude de ARM}
  \begin{block}{État de l'art}
    \begin{itemize}
      \item Compilation des tests \pause
      \item Analyse Binsec difficile sur \textit{p256-test.c}\pause
      \begin{itemize}
        \item Simplification des tests
        \item Remonté d'une erreur au sein même de la fonction
      \end{itemize}
    \end{itemize}
  \end{block}
\end{frame}

%  .  .  .  .  .  .  .  .  .  .  .  .  .  .  .  .  .  .  %

\begin{frame}[fragile]{À la poursuite de l'erreur}
  \begin{block}{Cible :}
    \begin{columns}
      \begin{column}{0.5\textwidth}
        \begin{lstlisting}[style=CStyle, caption={Hacl\_P256.h/cmovznz4}, gobble=10]
          static void cmovznz4(uint64_t cin, uint64_t *x, uint64_t *y, uint64_t *r)
          {
            uint64_t mask = ~FStar_UInt64_eq_mask(cin, (uint64_t)0U);
            uint64_t r0 = (y[0U] & mask) | (x[0U] & ~mask);
            uint64_t r1 = (y[1U] & mask) | (x[1U] & ~mask);
            uint64_t r2 = (y[2U] & mask) | (x[2U] & ~mask);
            uint64_t r3 = (y[3U] & mask) | (x[3U] & ~mask);
            r=[r0,r1,r2,r3];
          }
      \end{lstlisting}
      \end{column}
      \begin{column}{0.5\textwidth}
        \begin{lstlisting}[style=CStyle, caption={Hacl\_P256.h/bn\_cmovznz4}, gobble=10]
          static inline void bn_cmovznz4(uint64_t *res, uint64_t cin, uint64_t *x, uint64_t *y)
          {
            uint64_t mask = ~FStar_UInt64_eq_mask(cin, 0ULL);
            KRML_MAYBE_FOR4(i,
              0U,
              4U,
              1U,
              uint64_t *os = res;
              uint64_t uu____0 = x[i];
              uint64_t x1 = uu____0 ^ (mask & (y[i] ^ uu____0));
              os[i] = x1;);
          }
      \end{lstlisting}
      \end{column}
    \end{columns}
  \end{block}
  
\end{frame}

%  .  .  .  .  .  .  .  .  .  .  .  .  .  .  .  .  .  .  %

\begin{frame}[fragile]{Prise en main de Clang+LLVM}
  \begin{columns}
    \begin{column}{0.4\textwidth}
      \tiny{
        \dirtree{%
            .1 Documents.
            .2 cross\_compilation.
            .3 arm-gnu-toolchain-12.2.rel1-x86\_64-arm-none-linux-gnueabihf.
            .3  clang+llvm-14.0.6-x86\_64-linux-gnu-rhel-8.4.
            .3  clang+llvm-15.0.6-x86\_64-linux-gnu-ubuntu-18.04.
            .3  clang+llvm-16.0.4-x86\_64-linux-gnu-ubuntu-22.04.
            .3  clang+llvm-17.0.6-x86\_64-linux-gnu-ubuntu-22.04.
            .3  clang+llvm-18.1.8-x86\_64-linux-gnu-ubuntu-18.04.
          }
      }
    \end{column}
    \begin{column}{0.6\textwidth}
      \begin{lstlisting} [style=MakefileStyle, caption={Makefile}, gobble=8]
        define compile
        $(BUILD_DIR)/$(version)/%.exe: $(SRC_DIR)/%.c
          @mkdir -p $$(dir $$@)
          $(COMPIL)/$(version)/bin/clang $(ARCHI) $(CFLAGS) $(FORCE) -c $$< -o $$(patsubst %.exe,%.o,$$@)
          $(COMPIL)/$(version)/bin/clang $(ARCHI) $(CFLAGS) $(LDFLAGS) $(FORCE) $$(patsubst %.exe,%.o,$$@) -static -o $$@
          @rm $(BUILD_DIR)/$(version)/*.o
          @cp binsec_script/$(INI)/* $(BUILD_DIR)/$(version)/
        endef
      \end{lstlisting}
    \end{column}
  \end{columns}

\end{frame}

%  .  .  .  .  .  .  .  .  .  .  .  .  .  .  .  .  .  .  %

\begin{frame}{Résultats}
    \begin{center}
      \onslide<2,3>{--target=aarch64-none-linux-gnu}
      \renewcommand{\arraystretch}{1.5} 
      \begin{tabular}{|c|cc|cc|cc|cc|cc|}
        \hline
        \rowcolor{blue!30}
        \textbf{Clang+LLVM} & \multicolumn{2}{c|}{\textbf{14.0.6}} & \multicolumn{2}{c|}{\textbf{15.0.6}} & \multicolumn{2}{c|}{\textbf{16.0.4}} & \multicolumn{2}{c|}{\textbf{17.0.6}} & \multicolumn{2}{c|}{\textbf{18.1.8}} \\
        \hline
        \rowcolor{blue!10}
        \cellcolor{inria-2024-gris-bleu!20}\textbf{opt}\textbackslash\textbf{src} & cmovznz4 & bn\_cmovznz4 & cmov & bn & cmov & bn & cmov & bn & cmov & bn \\
        \hline
        \rowcolor{orange!30!red!50}
        \textbf{-O2} & \cellcolor{green!60}\checkmark & \cellcolor{green!60}\checkmark & \cellcolor{green!60}\checkmark & \cellcolor{green!60}\checkmark & \cellcolor{green!60}\checkmark & \cellcolor{green!60}\checkmark & \cellcolor{green!60}\checkmark & \cellcolor{green!60}\checkmark & \cellcolor{green!60}\checkmark & \cellcolor{green!60}\checkmark \\
        \hline
        \rowcolor{orange!30!red!50}
        \textbf{-O3} & \cellcolor{green!60}\checkmark & \cellcolor{green!60}\checkmark & \cellcolor{green!60}\checkmark & \cellcolor{green!60}\checkmark & \cellcolor{green!60}\checkmark & \cellcolor{green!60}\checkmark & \cellcolor{green!60}\checkmark & \cellcolor{green!60}\checkmark & \cellcolor{green!60}\checkmark & \cellcolor{green!60}\checkmark \\
        \hline
        \rowcolor{orange!30!red!50}
        \textbf{-Os} & \cellcolor{green!60}\checkmark & \cellcolor{green!60}\checkmark & \cellcolor{green!60}\checkmark & \cellcolor{green!60}\checkmark & \cellcolor{green!60}\checkmark & \cellcolor{green!60}\checkmark & \cellcolor{green!60}\checkmark & \cellcolor{green!60}\checkmark & \cellcolor{green!60}\checkmark & \cellcolor{green!60}\checkmark \\
        \hline
        \rowcolor{orange!30!red!50}
        \textbf{-Oz} & \cellcolor{green!60}\checkmark & \cellcolor{green!60}\checkmark & \cellcolor{green!60}\checkmark & \cellcolor{green!60}\checkmark & \cellcolor{green!60}\checkmark & \cellcolor{green!60}\checkmark & \cellcolor{green!60}\checkmark & \cellcolor{green!60}\checkmark & \cellcolor{green!60}\checkmark & \cellcolor{green!60}\checkmark \\
        \hline
      \end{tabular}
    \end{center}
    \pause \pause
    \begin{block}{Armv7}
      --target=gcc-arm-none-eabi -mcpu=cortex-a9 -marm
    \end{block}
\end{frame}

%  .  .  .  .  .  .  .  .  .  .  .  .  .  .  .  .  .  .  %

\begin{frame}{Résultats Armv7}
    \begin{center}
      --target=armv7-none-linux-gnueabihf \pause
      \renewcommand{\arraystretch}{1.5} 
      \begin{tabular}{|c|cc|cc|cc|cc|cc|}
        \hline
        \rowcolor{blue!30}
        \textbf{Clang+LLVM} & \multicolumn{2}{c|}{\textbf{14.0.6}} & \multicolumn{2}{c|}{\textbf{15.0.6}} & \multicolumn{2}{c|}{\textbf{16.0.4}} & \multicolumn{2}{c|}{\textbf{17.0.6}} & \multicolumn{2}{c|}{\textbf{18.1.8}} \\
        \hline
        \rowcolor{blue!10}
        \cellcolor{inria-2024-gris-bleu!20}\textbf{opt}\textbackslash\textbf{src} & cmovznz4 & bn\_cmovznz4 & cmov & bn & cmov & bn & cmov & bn & cmov & bn \\
        \hline
        \rowcolor{orange!30!red!50}
        \textbf{-O2} & \cellcolor{green!60}\checkmark & \cellcolor{green!60}\checkmark & \cellcolor{orange!60}\textasciitilde & \cellcolor{orange!60}\textasciitilde & \cellcolor{orange!60}\textasciitilde & \cellcolor{orange!60}\textasciitilde & \cellcolor{orange!60}\textasciitilde & \cellcolor{orange!60}\textasciitilde & \cellcolor{orange!60}\textasciitilde & \cellcolor{orange!60}\textasciitilde \\
        \hline
        \rowcolor{orange!30!red!50}
        \textbf{-O3} & \cellcolor{green!60}\checkmark & \cellcolor{green!60}\checkmark & \cellcolor{orange!60}\textasciitilde & \cellcolor{orange!60}\textasciitilde & \cellcolor{orange!60}\textasciitilde & \cellcolor{orange!60}\textasciitilde & \cellcolor{orange!60}\textasciitilde & \cellcolor{orange!60}\textasciitilde & \cellcolor{orange!60}\textasciitilde & \cellcolor{orange!60}\textasciitilde \\
        \hline
        \rowcolor{orange!30!red!50}
        \textbf{-Os} & \cellcolor{green!60}\checkmark & \cellcolor{green!60}\checkmark & \cellcolor{orange!60}\textasciitilde & \cellcolor{orange!60}\textasciitilde & \cellcolor{orange!60}\textasciitilde & \cellcolor{orange!60}\textasciitilde & \cellcolor{orange!60}\textasciitilde & \cellcolor{orange!60}\textasciitilde & \cellcolor{orange!60}\textasciitilde & \cellcolor{orange!60}\textasciitilde \\
        \hline
        \rowcolor{orange!30!red!50}
        \textbf{-Oz} & \cellcolor{orange!60}\textasciitilde & \cellcolor{orange!60}\textasciitilde & \cellcolor{orange!60}\textasciitilde & \cellcolor{orange!60}\textasciitilde & \cellcolor{orange!60}\textasciitilde & \cellcolor{orange!60}\textasciitilde & \cellcolor{orange!60}\textasciitilde & \cellcolor{orange!60}\textasciitilde & \cellcolor{orange!60}\textasciitilde & \cellcolor{orange!60}\textasciitilde \\
        \hline
      \end{tabular}
    \end{center}
\end{frame}

%  .  .  .  .  .  .  .  .  .  .  .  .  .  .  .  .  .  .  %

\begin{frame}[fragile]{Focus on 15.0.6}
  \begin{block}{Fast}
    \begin{columns}
      \begin{column}{0.5\textwidth}
        \begin{lstlisting} [style=CStyle, caption={FStar\_UInt64\_eq\_mask}, gobble=12]
            static uint64_t FStar_UInt64_eq_mask(uint64_t a, uint64_t b)
            {
              uint64_t x = a ^ b;
              uint64_t minus_x = ~x + 1ULL;
              uint64_t x_or_minus_x = x | minus_x;
              uint64_t xnx = x_or_minus_x >> 63U;
              return xnx - 1ULL;
            }
          \end{lstlisting}
      \end{column}


      \begin{column}{0.5\textwidth}
        \begin{lstlisting} [style=CStyle, caption={FStar\_UInt64\_eq\_mask}, gobble=12]
            #include "Hacl_P256.h"
          \end{lstlisting}
      \end{column}
    \end{columns}
  \end{block}
\end{frame}

%  .  .  .  .  .  .  .  .  .  .  .  .  .  .  .  .  .  .  %

\begin{frame}[fragile]{Focus on 15.0.6}
  \begin{block}{Fast}
    \begin{lstlisting} [style=INIStyle, caption={study.ini}, gobble=8]
        load sections .text, .rodata, .data, .got, .bss from file

        secret global  r, cin, y, x


        @[sp, 4] := 0x0000000000075570 as fin
        starting from <main>
        with concrete stack pointer
        halt at fin
        explore all
      \end{lstlisting}
    \end{block}
\end{frame}

%  .  .  .  .  .  .  .  .  .  .  .  .  .  .  .  .  .  .  %

\begin{frame}[fragile]{Binsec}
  \begin{lstlisting} [style=global2, caption={make binsec}, gobble=6]
      [sse:info] SMT queries
                  Preprocessing simplifications
                    total          1
                    true           0
                    false          0
                    constant enum  1
                  
                  Satisfiability queries
                    total          28
                    sat            2
                    unsat          26
                    unknown        0
                    time           0.57
                    average        0.02
                  
                Exploration
                  total paths                      4
                  completed/cut paths              0
                  pending paths                    0
                  stale paths                      4
                  failed assertions                0
                  branching points                 1
                  max path depth                   24
                  visited instructions (unrolled)  24
                  visited instructions (static)    24

      [checkct:result] Program status is : unknown (0.678)
      [checkct:info] 0 visited path covering 24 instructions
      [checkct:info] 1 / 1 control flow checks pass
      [checkct:info] 29 / 29 memory access checks pass
    \end{lstlisting}
\end{frame}

%  .  .  .  .  .  .  .  .  .  .  .  .  .  .  .  .  .  .  %

\begin{frame}[fragile]{Debug}
  \begin{lstlisting} [style=global, caption={FStar\_UInt64\_eq\_mask}, gobble=6]
      [sse:debug] 0x00075560 vst1.64	{d16, d17}, [r0 :128]	# <main> + 0x70
      [sse:debug] 0x00075564 mov	r0, #0               	# <main> + 0x74
      [sse:debug] 0x00075568 ldmia	sp!, {r4, pc}      	# <main> + 0x78
      [sse:info] Empty path worklist: halting ...
  \end{lstlisting}
  \pause
  \begin{lstlisting} [style=global, caption={disas}, gobble=6]
    75564:	e3a00000 	mov	r0, #0
    75568:	e8bd8010 	pop	{r4, pc}
    7556c:	000266c0 	andeq	r6, r2, r0, asr #13
    75570:	000266b4 			@ <UNDEFINED> instruction: 0x000266b4
  \end{lstlisting}
  \pause
  \begin{lstlisting} [style=INIStyle, caption={study.ini}, gobble=6]
    lr<32> := 0x7556c
  \end{lstlisting}
\end{frame}

%  .  .  .  .  .  .  .  .  .  .  .  .  .  .  .  .  .  .  %

\begin{frame}[fragile]{Solution finale}
  \begin{lstlisting} [style=global, caption={study.ini}, gobble=6]              
      [checkct:result] Program status is : secure (4.969)
      [checkct:info] 1 visited path covering 37 instructions
      [checkct:info] 2 / 2 control flow checks pass
      [checkct:info] 33 / 33 memory access checks pass
    \end{lstlisting}
\end{frame}

%  .  .  .  .  .  .  .  .  .  .  .  .  .  .  .  .  .  .  %

\begin{frame}[fragile]{Binsec - version include}
    \begin{lstlisting} [style=global2, caption={study.ini}, gobble=6]
      Preprocessing simplifications
          total          2
          true           0
          false          0
          constant enum  2
        
        Satisfiability queries
          total          31
          sat            4
          unsat          27
          unknown        0
          time           4.64
          average        0.15
        
      Exploration
        total paths                      4
        completed/cut paths              0
        pending paths                    0
        stale paths                      4
        failed assertions                0
        branching points                 2
        max path depth                   36
        visited instructions (unrolled)  36
        visited instructions (static)    36
                  
                
      [checkct:result] Program status is : unknown (4.707)
      [checkct:info] 0 visited path covering 36 instructions
      [checkct:info] 2 / 2 control flow checks pass
      [checkct:info] 33 / 33 memory access checks pass
    \end{lstlisting}
\end{frame}

%  .  .  .  .  .  .  .  .  .  .  .  .  .  .  .  .  .  .  %

\begin{frame}[fragile]{Debug - version include}
  \begin{lstlisting} [style=global, caption={debug.trace}, gobble=6]
      [sse:debug] 0x00075544 vst1.64	{d16, d17}, [r0 :128]	# <main> + 0x54
      [sse:debug] 0x00075548 mov	r0, #0               	# <main> + 0x58
      [sse:debug] 0x0007554c bx	lr                    	# <main> + 0x5c
      [sse:info] Empty path worklist: halting ...
  \end{lstlisting}
  \pause
  \begin{lstlisting} [style=global, caption={FStar\_UInt64\_eq\_mask}, gobble=6]
      75548:	e3a00000 	mov	r0, #0
      7554c:	e12fff1e 	bx	lr
      75550:	00026694 	muleq	r2, r4, r6
      75554:	00026688 	andeq	r6, r2, r8, lsl #13
  \end{lstlisting}
  \pause
  \begin{lstlisting} [style=INIStyle, caption={FStar\_UInt64\_eq\_mask}, gobble=4]
    lr<32> := 0x75550
  \end{lstlisting}
\end{frame}

%  .  .  .  .  .  .  .  .  .  .  .  .  .  .  .  .  .  .  %

\begin{frame}[fragile]{Solution finale - version include}
  \begin{lstlisting} [style=global, caption={study.ini}, gobble=6]              
      [arm:error] Probable parse error at line 5, column 1
                  Lexeme was: undefined
                  Entry was: (address . 0x00075550)
      (opcode . 0x00026694)
      (size . 4)
      (mnemonic . "; Unknown ARM instruction")
      (undefined)

                  Getting basic infos only ... 
      [sse:error] Cut path 1 (uninterpreted "; Unknown ARM instruction") @ 0x00075550
    \end{lstlisting}
\end{frame}

%  .  .  .  .  .  .  .  .  .  .  .  .  .  .  .  .  .  .  %

\begin{frame}{Résultats Armv7 - hypothétique}
    \begin{center}
      --target=armv7-none-linux-gnueabihf
      \renewcommand{\arraystretch}{1.5} 
      \begin{tabular}{|c|cc|cc|cc|cc|cc|}
        \hline
        \rowcolor{blue!30}
        \textbf{Clang+LLVM} & \multicolumn{2}{c|}{\textbf{14.0.6}} & \multicolumn{2}{c|}{\textbf{15.0.6}} & \multicolumn{2}{c|}{\textbf{16.0.4}} & \multicolumn{2}{c|}{\textbf{17.0.6}} & \multicolumn{2}{c|}{\textbf{18.1.8}} \\
        \hline
        \rowcolor{blue!10}
        \cellcolor{inria-2024-gris-bleu!20}\textbf{opt}\textbackslash\textbf{src} & cmovznz4 & bn\_cmovznz4 & cmov & bn & cmov & bn & cmov & bn & cmov & bn \\
        \hline
        \rowcolor{orange!30!red!50}
        \textbf{-O2} & \cellcolor{green!60}\checkmark & \cellcolor{green!60}\checkmark & \cellcolor{yellow!30!green!30}\checkmark & \cellcolor{yellow!30!green!30}\checkmark & \cellcolor{yellow!30!green!30}\checkmark & \cellcolor{yellow!30!green!30}\checkmark & \cellcolor{yellow!30!green!30}\checkmark & \cellcolor{yellow!30!green!30}\checkmark & \cellcolor{yellow!30!green!30}\checkmark & \cellcolor{yellow!30!green!30}\checkmark \\
        \hline
        \rowcolor{orange!30!red!50}
        \textbf{-O3} & \cellcolor{green!60}\checkmark & \cellcolor{green!60}\checkmark & \cellcolor{green!60}\checkmark & \cellcolor{yellow!30!green!30}\checkmark & \cellcolor{yellow!30!green!30}\checkmark & \cellcolor{yellow!30!green!30}\checkmark & \cellcolor{yellow!30!green!30}\checkmark & \cellcolor{yellow!30!green!30}\checkmark & \cellcolor{yellow!30!green!30}\checkmark & \cellcolor{yellow!30!green!30}\checkmark \\
        \hline
        \rowcolor{orange!30!red!50}
        \textbf{-Os} & \cellcolor{green!60}\checkmark & \cellcolor{green!60}\checkmark & \cellcolor{yellow!30!green!30}\checkmark & \cellcolor{yellow!30!green!30}\checkmark & \cellcolor{yellow!30!green!30}\checkmark & \cellcolor{yellow!30!green!30}\checkmark & \cellcolor{yellow!30!green!30}\checkmark & \cellcolor{yellow!30!green!30}\checkmark & \cellcolor{yellow!30!green!30}\checkmark & \cellcolor{yellow!30!green!30}\checkmark \\
        \hline
        \rowcolor{orange!30!red!50}
        \textbf{-Oz} & \cellcolor{orange!60}\textasciitilde & \cellcolor{orange!60}\textasciitilde & \cellcolor{orange!60}\textasciitilde & \cellcolor{orange!60}\textasciitilde & \cellcolor{orange!60}\textasciitilde & \cellcolor{orange!60}\textasciitilde & \cellcolor{orange!60}\textasciitilde & \cellcolor{orange!60}\textasciitilde & \cellcolor{orange!60}\textasciitilde & \cellcolor{orange!60}\textasciitilde \\
        \hline
      \end{tabular}
    \end{center}
\end{frame}

%  .  .  .  .  .  .  .  .  .  .  .  .  .  .  .  .  .  .  %

\begin{frame}{Résultats Armv7 - Frais du 22/05}
    \begin{center}
      --target=armv7-none-linux-gnueabihf
      \renewcommand{\arraystretch}{1.5} 
      \begin{tabular}{|c|cc|cc|cc|cc|cc|}
        \hline
        \rowcolor{blue!30}
        \textbf{Clang+LLVM} & \multicolumn{2}{c|}{\textbf{14.0.6}} & \multicolumn{2}{c|}{\textbf{15.0.6}} & \multicolumn{2}{c|}{\textbf{16.0.4}} & \multicolumn{2}{c|}{\textbf{17.0.6}} & \multicolumn{2}{c|}{\textbf{18.1.8}} \\
        \hline
        \rowcolor{blue!10}
        \cellcolor{inria-2024-gris-bleu!20}\textbf{opt}\textbackslash\textbf{src} & cmovznz4 & bn\_cmovznz4 & cmov & bn & cmov & bn & cmov & bn & cmov & bn \\
        \hline
        \rowcolor{orange!30!red!50}
        \textbf{-O2} & \cellcolor{green!60}\checkmark & \cellcolor{green!60}\checkmark & \cellcolor{green!60}\checkmark & \cellcolor{orange!60}\textasciitilde & \cellcolor{green!60}\checkmark & \cellcolor{orange!60}\textasciitilde & \cellcolor{green!60}\checkmark & \cellcolor{orange!60}\textasciitilde & \cellcolor{green!60}\checkmark & \cellcolor{orange!60}\textasciitilde \\
        \hline
        \rowcolor{orange!30!red!50}
        \textbf{-O3} & \cellcolor{green!60}\checkmark & \cellcolor{green!60}\checkmark & \cellcolor{green!60}\checkmark & \cellcolor{orange!60}\textasciitilde & \cellcolor{green!60}\checkmark & \cellcolor{orange!60}\textasciitilde & \cellcolor{green!60}\checkmark & \cellcolor{orange!60}\textasciitilde & \cellcolor{green!60}\checkmark & \cellcolor{orange!60}\textasciitilde \\
        \hline
        \rowcolor{orange!30!red!50}
        \textbf{-Os} & \cellcolor{green!60}\checkmark & \cellcolor{green!60}\checkmark & \cellcolor{green!60}\checkmark & \cellcolor{orange!60}\textasciitilde & \cellcolor{green!60}\checkmark & \cellcolor{orange!60}\textasciitilde & \cellcolor{green!60}\checkmark & \cellcolor{orange!60}\textasciitilde & \cellcolor{green!60}\checkmark & \cellcolor{orange!60}\textasciitilde \\
        \hline
        \rowcolor{orange!30!red!50}
        \textbf{-Oz} & \cellcolor{green!60}\checkmark & \cellcolor{orange!60}\textasciitilde & \cellcolor{green!60}\checkmark & \cellcolor{orange!60}\textasciitilde &  \cellcolor{green!60}\checkmark & \cellcolor{orange!60}\textasciitilde &  \cellcolor{green!60}\checkmark & \cellcolor{orange!60}\textasciitilde &  \cellcolor{green!60}\checkmark & \cellcolor{orange!60}\textasciitilde\\
        \hline
      \end{tabular}
    \end{center}
    \textit{Sous condition spéciale, voir démo.}
\end{frame}



%%--%%--%%--%%--%%--%%--%%--%%--%%--%%--%%--%%--%%--%%--%%

\section{Conclusion}
\frame{\sectionpage}

\begin{frame}{Conclusion}

  \begin{columns}
    \begin{column}{0.5\textwidth}
      \begin{block}{Automatisation}
        \begin{itemize}
          \item Continuer la génération des fichiers \textit{-test.c}
          \item Activer la chaîne de compilation
        \end{itemize}
      \end{block}
    \end{column}
    \pause
    \begin{column}{0.5\textwidth}
      \begin{block}{Reproduction de bug}
        \begin{itemize}
          \item Observation de nouveaux opcodes
          \item Sûrement \textit{insecure}
        \end{itemize}
      \end{block}
    \end{column}
  \end{columns}

\end{frame}

%%%%%%%%%%%%%%%%%%%%%%%%%%%%%%%%%%%%%%%%%%%%%%%%%%%%%%%

%% Le texte est modifiable en changeant \thankyou
%% \renewcommand{\thankyou}{Thank You.}
\frame{\merci}


\end{document}

