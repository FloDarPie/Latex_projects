% --- Ton document en A5 ---
\documentclass[10pt,a5paper]{article}

\usepackage[a5paper,margin=1.2cm]{geometry}
\usepackage[utf8]{inputenc}
\usepackage[T1]{fontenc}
\usepackage{lmodern}
\usepackage{multicol}
\setlength{\columnsep}{0.5cm}
\usepackage{enumitem}
\setlist{nosep}
\usepackage{xcolor}
\newcommand{\important}[1]{\textbf{\textcolor{red}{#1}}}
\pagestyle{empty}

\begin{document}

\section*{Plan de l'oral}

\begin{itemize}
  \item Introduction — contexte et objectifs
  \item Méthodes de protection et limitations
  \item Outils de vérifications
  \item Érysichthon
  \begin{itemize}
    \item Conception générale
    \item Andhrímnir
  \end{itemize}
  \item Résultats
  \item Conclusion
\end{itemize}

\subsection*{Introduction}
\begin{multicols}{2}
  \begin{itemize}
    \item Attaquer sur la sécurité et le besoin d'avoir des libs cryptographiques
    \item Présenter HACL*
    \item Historique timing attacks
    \item Introduction de la problématique
  \end{itemize}
\end{multicols}

\subsection*{Méthodes de protection et limitations}
- compilateurs
\begin{multicols}{2}
\begin{itemize}
  \item CompCert $=>$ garanties formelles $//$ retard sur les standards
  \item Jasmine $=>$ annotations de codes, execute toutes les branches $//$ pas employable sur un projet industriel, artefact de recherche
  \item Raccoon $=>$ annotations de codes $//$ pas le temps constants
  \item Constantine $=>$ linéarisation $//$ 16.36x taille binaire \& 27.1x temps
\end{itemize}
\end{multicols}
- assembleur\\
- programmation en temps constant 

\subsection*{Outils de vérifications}
\subsection*{Érysichthon}
\subsection*{Résultats}
\subsection*{Conclusion}
\end{document}
