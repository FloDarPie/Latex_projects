% --- Ton document en A5 ---
\documentclass[10pt,a5paper]{article}

\usepackage[a5paper,margin=1.2cm]{geometry}
\usepackage[utf8]{inputenc}
\usepackage[T1]{fontenc}
\usepackage{lmodern}
\usepackage{multicol}
\setlength{\columnsep}{0.5cm}
\usepackage{enumitem}
\setlist{nosep}
\usepackage{xcolor}
\newcommand{\important}[1]{\textbf{\textcolor{red}{#1}}}
\pagestyle{empty}

\begin{document}

\section*{Plan de l'oral}

\begin{itemize}
  \item Introduction — contexte et objectifs
  \item Méthodes de protection et limitations
  \item Outils de vérifications
  \begin{itemize}
    \item Étude sur cas simple
    \item Contraintes et identification des limitations
  \end{itemize}
  \item Érysichthon
  \begin{itemize}
    \item Conception générale
    \item Andhrímnir
  \end{itemize}
  \item Résultats
  \item Annexes
\end{itemize}

\subsection*{Introduction}
\begin{multicols}{2}
  \begin{itemize}
    \item Attaquer sur la sécurité et le besoin d'avoir des libs cryptographiques
    \item Présenter HACL*
    \item Historique timing attacks et mise à distance
    \item Exemple et instructions en temps constant
  \end{itemize}
\end{multicols}

\subsubsection*{Axes de défenses}
- programmation en temps constant \\
\important{motivation du stage avec l'article}\\
- compilateurs
\begin{itemize}
  \item CompCert $=>$ garanties formelles \dotfill retard sur les standards
  \item Jasmine $=>$ annotations de codes, execute toutes les branches \dotfill pas employable sur un projet industriel, artefact de recherche
  \item Raccoon $=>$ annotations de codes \dotfill pas le temps constants
  \item Constantine $=>$ linéarisation \dotfill 16.36x taille binaire \& 27.1x temps
\end{itemize}
- assembleur\\

\subsubsection*{Réalisation}
\textbf{vérification de binaire et continuité des Spécifications}\\
Présentation Érysichthon et résultats

\medbreak
\textit{Sommaire}
\newpage

\subsection*{Outils de vérifications}
- tableau \\
\textbf{vérification de binaire}\\
\textbf{vérification correcte}
\smallbreak
- \important{Binsec}
\subsubsection*{Binsec}
\begin{itemize}
  \item analyse au binaire
  \item couvrir de nombreuses architectures
  \item permet l'automatisation
\end{itemize}

Commande de choix de l'outil $\bigoplus$ script d'instruction\smallbreak

EXEMPLE

\subsection*{Automatismes}
\begin{itemize}
  \item Simplification
  \item Réduction de la taille des binaires
  \item Script binsec simple
\end{itemize}
\important{tableaux de résultats}\smallbreak

- identifications des points d'attention\\
- Variables secrètes et test complet

\subsubsection*{Cahier des charges}
\begin{itemize}
  \item identification des points clés
  \item graphes de fonctionnements
  \item spécialisation x86\_64
\end{itemize}



\subsection*{Érysichthon}

- construction en modules
- ensemble des scripts et de Makefile

\subsubsection*{Andhrímnir}
- Indépendant : automatique et corrects\\
- Adapté : facilité l'usage et avancer dans la conception de recherche\\
\smallbreak
Graphe de fonctionnement
\smallbreak
Exemple et standardisation
\newpage
\subsection*{Résultats}
- graphes\\
- discuter des \textbf{unknown}\\
\hspace{5pt}\textbf{ORDRE des arrêts}
  \begin{itemize}
    \item max-depth / timeout / killed
    \item killed
    \item syscall / KO / error
    \item KO / error
  \end{itemize}



\subsection*{Conclusion}
(afficher les références)



\subsection*{Annexes}

\begin{itemize}
  \item options de compilation
  \item construction en vue user
  \item pourquoi json
  \item fin de stage / ouvertures autres pb
\end{itemize}

\end{document}
