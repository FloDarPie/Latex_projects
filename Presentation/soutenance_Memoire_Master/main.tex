% !TEX root = ./main.tex


%-----------------------------------------------------------------------------
%    STARTER
%-----------------------------------------------------------------------------

\documentclass{backend/backend}
\usepackage{backend/font}
\usepackage{backend/colors}
\usepackage{backend/structure}
\usepackage{backend/informations}

%-----------------------------------------------------------------------------
%    PRESENTATION SLIDES
%-----------------------------------------------------------------------------

\begin{document}
\justifying

\begin{frame}
    \titlepage
\end{frame}

\comment{  %%%%%%%%%%%%%%%%%%%%%%%%%%%%%%%%%%%%% COMMENTARY


\section*{Introduction}
\showtoctrue % active l'affichage des slides de transition

\begin{frame}{Introduction - 1}
    \pause
    \begin{exampleblock}{HACL*}
        \textit{"\textbf{H}igh \textbf{A}ssurance \textbf{C}ryptography \textbf{L}ibrary"}\footnote{\url{https://hacl-star.github.io/}} est une bibliothèque cryptographique, écrite en F* ("F star"), implémentant tous les algorithmes de cryptographie modernes et est prouvée mathématiquement sûre. 
        \smallbreak
        HACL* est notamment utilisé dans plusieurs systèmes de production tels que Mozilla Firefox, le noyau Linux, le VPN WireGuard...
    \end{exampleblock}
\end{frame}

\begin{frame}{Introduction - 2}
    \begin{block}{1996 : Paul C. Kocher, \textit{Timing Attacks on Implementations of Diffie-Hellman, RSA, DSS, and Other Systems} }
        Une mesure précise du temps requis par des opérations sur les clés secrètes permettrait à un attaquant de casser le cryptosystème.
    \end{block}
    \pause
    2003 : \citeauthor{270176} \citetitle{270176}\\
    \pause
    2011 : \citeauthor{stillPractical} \citetitle{stillPractical}
\end{frame}


\begin{frame}{Introduction - 3}
    \begin{enumerate}
        \item[\textbf{QR1}] Est-il possible de propager les garanties de sécurité pendant la compilation ?
        \item[\textbf{QR2}] Est-il possible d'automatiser la détection de ces failles sur des fichiers compilés ?
        \item[\textbf{QR3}] Est-il possible d'appliquer ces mécanismes pour assurer la vérification d'une bibliothèque cryptographique ?
    \end{enumerate}
\end{frame}

%%%%%%%%%%%%%%%%%%%%%%%%%%%%%%%%%%%%%%%%%
%
%         SUMMARY
%
%%%%%%%%%%%%%%%%%%%%%%%%%%%%%%%%%%%%%%%%%
\begin{frame}{Sommaire}
        \small
        \tableofcontents
\end{frame}


%%%%%%%%%%%%%%%%%%%%%%%%%%%%%%%%%%%%%%%%%
%
%         PREMIERE PARTIE
%
%%%%%%%%%%%%%%%%%%%%%%%%%%%%%%%%%%%%%%%%%
\section{Méthodes de protection et limitations}

\begin{frame}{État des lieux}
    \begin{blockSimple}{Usage sécurisé}
        \centering
        \begin{tikzpicture}[>={Latex[length=3mm,width=2.5mm]}, link/.style={->, very thick}]

        % Nœuds avec apparition progressive
        \node[src] (src) {Source};
        \node[comp, right=18mm of src] (cmp) {Compilateur};
        \node[asm, right=18mm of cmp] (asm) {Assembleur};

        % Flèches
        \draw[link] (src) -- (cmp);
        \draw[link] (cmp) -- (asm);

        \end{tikzpicture}
    \end{blockSimple}
\end{frame}


\begin{frame}{Analyse en remontée - 1}

    \begin{blockSimple}{Écrire en assembleur}
        \begin{columns}
            \begin{column}{0.3\textwidth}
                + Efficace\\
                + Contrôle total
            \end{column}
            \begin{column}{0.5\textwidth}
            - Restreint l'architecture et les usages\\
            - Beaucoup de connaissance spécifique au processeur ciblé
            \end{column}
        \end{columns}        
    \end{blockSimple}

    \vspace{6em}
    \begin{blockSimple}{}
        \centering
        \begin{tikzpicture}[scale=0.5, transform shape,>={Latex[length=3mm,width=2.5mm]}, link/.style={->, very thick}]

        % Nœuds avec apparition progressive
        \node[gray] (src) {Source};
        \node[gray, right=18mm of src] (cmp) {Compilateur};
        \node[asm, right=18mm of cmp] (asm) {Assembleur};

        % Flèches
        \draw[link] (src) -- (cmp);
        \draw[link] (cmp) -- (asm);

        \end{tikzpicture}
    \end{blockSimple}
\end{frame}


\begin{frame}{Analyse en remontée - 2}

    \begin{blockSimple}{Utilisation des compilateurs}
        \begin{columns}
        \begin{column}{0.5\textwidth}
            \begin{itemize}
                \item Constantine - 2021
                \item Jasmin - 2017
                \item \alt<1>{Raccoon - 2015}{\sout{Raccoon} - 2015}
                \item CompCert - 2008 (2019)
            \end{itemize}
        \end{column}
        \begin{column}{0.5\textwidth}
          \onslide<2>{
            \begin{itemize}
            \item[-] Couverture des architectures supportée
            \item[-] Informations à transmettre
            \item[-] Spécifications ne sont plus respectées
            \end{itemize}
          }
        \end{column}
    \end{columns}
    \end{blockSimple}
    \vspace{3em}
    \begin{blockSimple}{}
        \centering
        \begin{tikzpicture}[scale=0.5, transform shape,>={Latex[length=3mm,width=2.5mm]}, link/.style={->, very thick}]

        % Nœuds avec apparition progressive
        \node[gray] (src) {Source};
        \node[comp, right=18mm of src] (cmp) {Compilateur};
        \node[gray, right=18mm of cmp] (asm) {Assembleur};

        % Flèches
        \draw[link] (src) -- (cmp);
        \draw[link] (cmp) -- (asm);

        \end{tikzpicture}
    \end{blockSimple}
\end{frame}

\begin{frame}{Analyse en remontée - 3}

    \begin{blockSimple}{Programmation en temps constant}
        \begin{columns}
            \begin{column}{0.3\textwidth}
                + Position haut niveau\\
                + Couverture d'architectures importantes
            \end{column}
            \begin{column}{0.5\textwidth}
            - Rigueur et conception particulière des actions\\
            - Identification des points de fuites
            \end{column}
        \end{columns}        
    \end{blockSimple}

    \vspace{4em}
    \begin{blockSimple}{}
        \centering
        \begin{tikzpicture}[scale=0.5, transform shape,>={Latex[length=3mm,width=2.5mm]}, link/.style={->, very thick}]

        % Nœuds avec apparition progressive
        \node[src] (src) {Source};
        \node[gray, right=18mm of src] (cmp) {Compilateur};
        \node[gray, right=18mm of cmp] (asm) {Assembleur};

        % Flèches
        \draw[link] (src) -- (cmp);
        \draw[link] (cmp) -- (asm);

        \end{tikzpicture}
    \end{blockSimple}
\end{frame}


} %%%%%%%%%%%%%%%%%%%%%%%%%%%%%%%%%%%%% COMMENTARY


\begin{frame}[fragile]{Opérations dangereuses}

    \begin{columns}
        \begin{column}{0.4\textwidth}
            \begin{blockSimple}{Opérations influantes :}
                \begin{itemize}
                    \item Accès mémoire
                    \item Décalage/rotation de valeurs \hfill\onslide<2>{(caché)}
                    \item Saut conditionnel
                    \item Division/multiplication
                \end{itemize}
            \end{blockSimple}
        \end{column}
        \pause
        \begin{column}{0.5\textwidth}
            \begin{minted}[frame=lines,framesep=2mm,baselinestretch=1.2,fontsize=\tiny,linenos, gobble=8]{C}
                bool check_pwd(msg, pwd){
                if (msg.length != pwd.length){
                    return False
                }
                for(int i = 0; i < msg.length; i++){
                    if(msg[i] != pwd[i]){
                        return False
                    }
                }
                return True
                }
            \end{minted}
            \begin{tikzpicture}[scale = 0.5, transform shape]
            % Noeuds
            \node (start) [startstop] {check\_pwd};
            \node (valid) [right of=start, xshift=9cm, green] {\huge{$\checkmark$}};
            \draw [arrow] (start) -- (valid);
            
            \node (inputs) [below of = start, yshift=0.2cm] {(msg, pwd)};
            \node (if) [process] [right of=start, xshift=2cm] {if};
            \node (not) [above of=if, red] {\huge{$\times$}};
            \node (for) [process] [right of=if,, xshift=2cm] {for};
            \node (for1) [above of=for,xshift=1cm, red] {\huge{$\times$}};
            \node (for2) [right of=for1, red] {\huge{$\times$}};
            \node (for3) [right of=for2, red] {\huge{$\times$}};

            \draw [arrow] (if) -- (not);
            \draw [arrow] (for) -- (for1);
            \draw [arrow] (for) -- (for2);
            \draw [arrow] (for) -- (for3);

            \node (t) [above of=start] {};
            \node (a) [above of=valid, xshift=-0.3cm] {};
            \draw [arrow] (t) -- node[above left] {Temps ($\mu$s)} (a.west);
            
            \end{tikzpicture}
        \end{column}
    \end{columns}

\end{frame}

\begin{frame}{Plus de problème ?}
    \pause
    \begin{blockSimple}{Mauvaises nouvelles \alt<2>{?}{!}}
        2019 : \citeauthor{binsecRel2019}, \citetitle{binsecRel2019}\\
        \pause
        2024 : \citeauthor{schneider2024breakingbadcompilersbreak}, \citetitle{schneider2024breakingbadcompilersbreak}
    \end{blockSimple}
\end{frame}

%%%%%%%%%%%%%%%%%%%%%%%%%%%%%%%%%%%%%%%%%
%
%         DEUXIEME PARTIE
%
%%%%%%%%%%%%%%%%%%%%%%%%%%%%%%%%%%%%%%%%%

\section{Outils de vérifications}
\begin{frame}{Spécialisations}
    \begin{columns}% t = alignement en haut
    % Colonne gauche : tableau
    \column{0.6\textwidth}
    \tiny
    \begin{tabular}{lccc}
        \toprule
        \textbf{Outil} & \textbf{Cible} & \textbf{Techn.} & \textbf{Garanties} \\
        \rowcolor{lightgray}
        ctgrind \cite{ctgrind} & Binaire & Dynamique & $\blacktriangle$ \\
        ABPV13 \cite{ABPV13} & C & Formel & $\bullet$ \\
        \rowcolor{lightgray}
        VirtualCert \cite{VirtualCert} & x86 & Formel & $\bullet$ \\
        ct-verif \cite{ctverif} & LLVM & Formel & $\bullet$ \\
        \rowcolor{lightgray}
        FlowTracker \cite{FlowTracker} & LLVM & Formel & $\bullet$ \\
        Blazer \cite{Blazer} & Java & Formel & $\bullet$ \\
        \rowcolor{lightgray}
        BPT17 \cite{BPT17} & C & Symbolique & $\blacktriangle$ \\
        MemSan \cite{MemSan} & LLVM & Dynamique & $\blacktriangle$ \\
        \rowcolor{lightgray}
        Themis \cite{Themis} & Java & Formel & $\bullet$ \\
        COCO-CHANNEL \cite{COCOCHANNEL} & Java & Symbolique & $\bullet$ \\
        \rowcolor{lightgray}
        DATA \cite{DATA1,DATA2} & Binaire & Dynamique & $\blacktriangle$ \\
        MicroWalk \cite{MicroWalk} & Binaire & Dynamique & $\blacktriangle$ \\
        \rowcolor{lightgray}
        timecop \cite{timecop} & Binaire & Dynamique & $\blacktriangle$ \\
        SC-Eliminator \cite{SCEliminator} & LLVM & Formel & $\bullet$ \\
        \rowcolor{lightgray}
        Binsec/Rel \cite{binsecRel2019} & Binaire & Symbolique & $\blacktriangle$ \\
        CT-WASM \cite{CTWASM} & WASM & Formel & $\bullet$ \\
        \rowcolor{lightgray}
        FaCT \cite{FaCT} & DSL & Formel & $\bullet$ \\
        haybale-pitchfork \cite{haybale-pitchfork} & LLVM & Symbolique & $\blacktriangle$ \\
        \bottomrule
    \end{tabular}

    % Colonne droite : légende et titre
    \column{0.5\textwidth}
    \textbf{Liste d’outils de vérification}\\[1ex]
    Source : \cite{notThatHardCT}\\[2ex]
    \begin{scriptsize}
        
        \textbf{Cible}
        \begin{itemize}
        \item[[C, Java]] Code source
        \item[Binaire] Binaire
        \item[DSL] Surcouche de langage
        \item[Trace] Trace d'exécution
        \item[WASM] Assembleur web
    \end{itemize}
    \textbf{Techn.}
    \begin{itemize}
        \item[Formel] Programmation formelle 
        \item[[*]] type d'analyse
    \end{itemize}
    \textbf{Garanties (attaques temporelles)}\\
        $\bullet$ = Analyse correcte,$\blacktriangle$ =  Limitée
    \end{scriptsize}
    \end{columns}
\end{frame}



\section{Automatismes}
\begin{frame}{empty}
    
\end{frame}
\section{Érysichthon}

\subsection{Conception générale}
\begin{frame}{empty}
    
\end{frame}
\subsection{Andhrímnir}
\begin{frame}{empty}
    
\end{frame}

\end{document}