% !TEX root = ./main.tex


%-----------------------------------------------------------------------------
%    STARTER
%-----------------------------------------------------------------------------

\documentclass{backend/backend}
\usepackage{backend/font}
\usepackage{backend/colors}
\usepackage{backend/structure}
\usepackage{backend/informations}

%-----------------------------------------------------------------------------
%    PRESENTATION SLIDES
%-----------------------------------------------------------------------------

\begin{document}


\begin{frame}
    \titlepage
\end{frame}

\section*{Introduction}
\showtoctrue % active l'affichage des slides de transition

\begin{frame}{Introduction}
    
    \pause
    \begin{block}{1996 : Paul C. Kocher, \textit{Timing Attacks on Implementations of Diffie-Hellman, RSA, DSS, and Other Systems} }
        Une mesure précise du temps requis par des opérations sur les clés secrètes permettrait à un attaquant de casser le cryptosystème.
    \end{block}

    \pause
    2003 : \citeauthor{270176} \citetitle{270176}\\
    \pause
    2011 : \citeauthor{stillPractical} \citetitle{stillPractical}
\end{frame}

\begin{frame}{Introduction}
    \begin{enumerate}
        \item[\textbf{QR1}] Est-il possible de propager les garanties de sécurité pendant la compilation ?
        \item[\textbf{QR2}] Est-il possible d'automatiser la détection de ces failles sur des fichiers compilés ?
        \item[\textbf{QR3}] Est-il possible d'appliquer ces mécanismes pour assurer la vérification d'une bibliothèque cryptographique ?
    \end{enumerate}
\end{frame}

\begin{frame}{Sommaire}

        \small
        \tableofcontents

\end{frame}



\section{sec A}
\section{sec B}
\subsection{petit B.1}
\subsection{petit B.2}

\section{sec C}
\subsection{petit C.1}
\section{sec D}
\subsection{petit D.1}
\subsection{petit D.2}

\end{document}