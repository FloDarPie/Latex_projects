% !TEX root = ./main.tex


%-----------------------------------------------------------------------------
%    STARTER
%-----------------------------------------------------------------------------

\documentclass{backend/backend}
\usepackage{backend/font}
\usepackage{backend/colors}
\usepackage{backend/structure}
\usepackage{backend/informations}

%-----------------------------------------------------------------------------
%    PRESENTATION SLIDES
%-----------------------------------------------------------------------------

\begin{document}
\justifying

\begin{frame}
    \titlepage
\end{frame}

\section*{Introduction}
\showtoctrue % active l'affichage des slides de transition

\begin{frame}{Introduction - 1}
    
    \pause
    \begin{block}{1996 : Paul C. Kocher, \textit{Timing Attacks on Implementations of Diffie-Hellman, RSA, DSS, and Other Systems} }
        Une mesure précise du temps requis par des opérations sur les clés secrètes permettrait à un attaquant de casser le cryptosystème.
    \end{block}

    \pause
    2003 : \citeauthor{270176} \citetitle{270176}\\
    \pause
    2011 : \citeauthor{stillPractical} \citetitle{stillPractical}
\end{frame}

\begin{frame}{Introduction - 2}
    \begin{exampleblock}{HACL*}
        \textit{"\textbf{H}igh \textbf{A}ssurance \textbf{C}ryptography \textbf{L}ibrary"}\footnote{\url{https://hacl-star.github.io/}} est une bibliothèque cryptographique, écrite en F* ("F star"), implémentant tous les algorithmes de cryptographie modernes et est prouvée mathématiquement sûre. 
        \smallbreak
        HACL* est notamment utilisé dans plusieurs systèmes de production tels que Mozilla Firefox, le noyau Linux, le VPN WireGuard...
    \end{exampleblock}
\end{frame}

\begin{frame}{Introduction - 3}
    \begin{enumerate}
        \item[\textbf{QR1}] Est-il possible de propager les garanties de sécurité pendant la compilation ?
        \item[\textbf{QR2}] Est-il possible d'automatiser la détection de ces failles sur des fichiers compilés ?
        \item[\textbf{QR3}] Est-il possible d'appliquer ces mécanismes pour assurer la vérification d'une bibliothèque cryptographique ?
    \end{enumerate}
\end{frame}



\begin{frame}{Sommaire}

        \small
        \tableofcontents

\end{frame}



\section{Méthodes de protection et limitations}

\begin{frame}{Usages de compilateurs spécialisés}
    \begin{columns}
        \begin{column}{0.5\textwidth}
            \begin{blockSimple}{Petit tour d'horizon}
            \begin{itemize}
                \item Constantine - 2021
                \item Jasmine - 2017
                \item \alt<1>{Raccoon - 2015}{\sout{Raccoon} - 2015}
                \item CompCert - 2008 (2019)
            \end{itemize}
        \end{blockSimple}
        \end{column}
        \pause
        \begin{column}{0.5\textwidth}
          \begin{blockSimple}{Freins}
            \pause
                \begin{itemize}
                \item Couverture des architectures supportée
                \item Informations à transmettre
                \item Spécifitées ne sont plus respectées
                \end{itemize}
                
            \end{blockSimple}
        \end{column}
    \end{columns}


\end{frame}

\begin{frame}{Bonnes pratiques de programmations}
    
\end{frame}

\section{Outils de vérifications}
\begin{frame}{Usages de compilateur}
    
\end{frame}
\section{Automatismes}
\begin{frame}{Usages de compilateur}
    
\end{frame}
\section{Érysichthon}

\subsection{Conception générale}
\begin{frame}{Usages de compilateur}
    
\end{frame}
\subsection{Andhrímnir}
\begin{frame}{Usages de compilateur}
    
\end{frame}

\end{document}