% !TEX root = ./main.tex


%-----------------------------------------------------------------------------
%    STARTER
%-----------------------------------------------------------------------------

\documentclass{backend/backend}
\usepackage{backend/font}
\usepackage{backend/colors}
\usepackage{backend/structure}
\usepackage{backend/informations}

%-----------------------------------------------------------------------------
%    PRESENTATION SLIDES
%-----------------------------------------------------------------------------

\begin{document}
\justifying

\begin{frame}
    \titlepage
\end{frame}

\section*{Introduction}
\showtoctrue % active l'affichage des slides de transition

\begin{frame}{Introduction - 1}
    \pause
    \begin{exampleblock}{HACL*}
        \textit{"\textbf{H}igh \textbf{A}ssurance \textbf{C}ryptography \textbf{L}ibrary"}\footnote{\url{https://hacl-star.github.io/}} est une bibliothèque cryptographique, écrite en F* ("F star"), implémentant tous les algorithmes de cryptographie modernes et est prouvée mathématiquement sûre. 
        \smallbreak
        HACL* est notamment utilisé dans plusieurs systèmes de production tels que Mozilla Firefox, le noyau Linux, le VPN WireGuard...
    \end{exampleblock}
\end{frame}

\begin{frame}{Introduction - 2}
    \begin{block}{1996 : Paul C. Kocher, \textit{Timing Attacks on Implementations of Diffie-Hellman, RSA, DSS, and Other Systems} }
        Une mesure précise du temps requis par des opérations sur les clés secrètes permettrait à un attaquant de casser le cryptosystème.
    \end{block}
    \pause
    2003 : \citeauthor{270176} \citetitle{270176}\\
    \pause
    2011 : \citeauthor{stillPractical} \citetitle{stillPractical}
\end{frame}


\begin{frame}{Introduction - 3}
    \begin{enumerate}
        \item[\textbf{QR1}] Est-il possible de propager les garanties de sécurité pendant la compilation ?
        \item[\textbf{QR2}] Est-il possible d'automatiser la détection de ces failles sur des fichiers compilés ?
        \item[\textbf{QR3}] Est-il possible d'appliquer ces mécanismes pour assurer la vérification d'une bibliothèque cryptographique ?
    \end{enumerate}
\end{frame}

\begin{frame}{Sommaire}
        \small
        \tableofcontents
\end{frame}



\section{Méthodes de protection et limitations}

\begin{frame}{État des lieux}
    \begin{blockSimple}{Usage sécurisé}
        \centering
        \begin{tikzpicture}[>={Latex[length=3mm,width=2.5mm]}, link/.style={->, very thick}]

        % Nœuds avec apparition progressive
        \node[src] (src) {Source};
        \node[comp, right=18mm of src] (cmp) {Compilateur};
        \node[asm, right=18mm of cmp] (asm) {Assembleur};

        % Flèches
        \draw[link] (src) -- (cmp);
        \draw[link] (cmp) -- (asm);

        \end{tikzpicture}
    \end{blockSimple}
\end{frame}


\begin{frame}{Analyse en remontée - 1}

    \begin{blockSimple}{Écrire en assembleur}
        \begin{columns}
            \begin{column}{0.3\textwidth}
                + Efficace\\
                + Contrôle total
            \end{column}
            \begin{column}{0.5\textwidth}
            - Restreint l'architecture et les usages\\
            - Beaucoup de connaissance spécifique au processeur ciblé
            \end{column}
        \end{columns}        
    \end{blockSimple}

    \vspace{6em}
    \begin{blockSimple}{}
        \centering
        \begin{tikzpicture}[scale=0.5, transform shape,>={Latex[length=3mm,width=2.5mm]}, link/.style={->, very thick}]

        % Nœuds avec apparition progressive
        \node[gray] (src) {Source};
        \node[gray, right=18mm of src] (cmp) {Compilateur};
        \node[asm, right=18mm of cmp] (asm) {Assembleur};

        % Flèches
        \draw[link] (src) -- (cmp);
        \draw[link] (cmp) -- (asm);

        \end{tikzpicture}
    \end{blockSimple}
\end{frame}


\begin{frame}{Analyse en remontée - 2}

    \begin{blockSimple}{Utilisation des compilateurs}
        \begin{columns}
        \begin{column}{0.5\textwidth}
            \begin{itemize}
                \item Constantine - 2021
                \item Jasmin - 2017
                \item \alt<1>{Raccoon - 2015}{\sout{Raccoon} - 2015}
                \item CompCert - 2008 (2019)
            \end{itemize}
        \end{column}
        \begin{column}{0.5\textwidth}
          \onslide<2>{
            \begin{itemize}
            \item[-] Couverture des architectures supportée
            \item[-] Informations à transmettre
            \item[-] Spécifications ne sont plus respectées
            \end{itemize}
          }
        \end{column}
    \end{columns}
    \end{blockSimple}
    \vspace{3em}
    \begin{blockSimple}{}
        \centering
        \begin{tikzpicture}[scale=0.5, transform shape,>={Latex[length=3mm,width=2.5mm]}, link/.style={->, very thick}]

        % Nœuds avec apparition progressive
        \node[gray] (src) {Source};
        \node[comp, right=18mm of src] (cmp) {Compilateur};
        \node[gray, right=18mm of cmp] (asm) {Assembleur};

        % Flèches
        \draw[link] (src) -- (cmp);
        \draw[link] (cmp) -- (asm);

        \end{tikzpicture}
    \end{blockSimple}
\end{frame}

\begin{frame}{Analyse en remontée - 3}

    \begin{blockSimple}{Programmation en temps constant}
        \begin{columns}
            \begin{column}{0.3\textwidth}
                + Position haut niveau\\
                + Couverture d'architectures importantes
            \end{column}
            \begin{column}{0.5\textwidth}
            - Rigueur et conception particulière des actions\\
            - Identification des points de fuites
            \end{column}
        \end{columns}        
    \end{blockSimple}

    \vspace{4em}
    \begin{blockSimple}{}
        \centering
        \begin{tikzpicture}[scale=0.5, transform shape,>={Latex[length=3mm,width=2.5mm]}, link/.style={->, very thick}]

        % Nœuds avec apparition progressive
        \node[src] (src) {Source};
        \node[gray, right=18mm of src] (cmp) {Compilateur};
        \node[gray, right=18mm of cmp] (asm) {Assembleur};

        % Flèches
        \draw[link] (src) -- (cmp);
        \draw[link] (cmp) -- (asm);

        \end{tikzpicture}
    \end{blockSimple}
\end{frame}



\begin{frame}{Exemple}

\end{frame}


\section{Outils de vérifications}
\begin{frame}{empty}
    
\end{frame}
\section{Automatismes}
\begin{frame}{empty}
    
\end{frame}
\section{Érysichthon}

\subsection{Conception générale}
\begin{frame}{empty}
    
\end{frame}
\subsection{Andhrímnir}
\begin{frame}{empty}
    
\end{frame}

\end{document}