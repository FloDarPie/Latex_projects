\begin{large}

%---------%
% Adresse %
%---------%

\hfill \entreprise, \\
DUZÉS Florian  \hfill \adresse,\\
63 avenue de la République\hfill \adresseVille \\
94800 Villejuif \\
\phonenumber \\
\mail

\medbreak\medbreak

 
   
\vspace{8pt}
{\color{accentLine} \hrule}
\begin{center}
    {\Large\color{accentTitle} \stagename}
\end{center}
{\color{accentLine} \hrule}
\vspace{8pt}

 %---------------%
 % Ma motivation %
 %---------------%

\begin{justify}
\hspace{15pt}Bonjour Monsieur Hiet,\\

Bonjour Messieurs Courousse, Jean et Wilke.\\

Je suis actuellement en stage au sein de l'équipe PROSECCO, au centre Inria
Paris, sous la supervision joins de M.Aymeric fromherz \& des Messieurs Sebastien
Bardin et Yanis Sellami du centre CEA Nano-Innov de Saclay. Ce poste d'ingénieur de recherche me permet
de terminer mon master "Cryptologie et Sécurité Informatique" de Bordeaux.\\

Mon plan à l'origine était d'entrer dans le privé à la fin de ce master, mais les échanges avec mes professeurs et collègues de bureaux m'ont ouvert à la
possiblité qu'il existe une autre voie. Cette voie, c'est la thèse. Dans cet objectif, je prend contact avec différents centre de recherche français pour
recueillir des informations et me préparer à cette possibilité.\\

Ce chemin de la thèse, je le vois comme une possiblité offerte à une jeune personne d'essayer pendant trois ans, puis, de présenter un retour sur son
parcours pour faire avancer la recherche. C'est une chance que je voudrais saisir. Pour moi, la thèse, c'est un travail personnel, où on cultive un 
savoir, qui est ensuite rendu impersonnel. C'est un travail où il y a moins de contrainte de retour sur investissement que dans une société privée.
Un espace de liberté proposé dans un but de recherche et développement, mais un travail quand même.\\


J'aime le travail de recherche que je fais actuellement. Il consiste à "industrialiser" le processus de détection de faille pour des attaques temporelles.
Je vérifie la sureté, selon ce prisme, d'une bibliothèque cryptographique. J'aimerai pouvoir élargir ce prisme. Dans cette optique, votre sujet propose de 
descendre encore d'une couche dans la sécurité. Je voulais poursuivre en cherchant un travail au niveau de la sécurité du code, je n'avais pas conscience
qu'il était possible d'augmenter en précision. Cette offre me permet de découvrir un nouveau aspect de la sécurité que je ne percevais pas. J'ai envie
de me lancer dans cette nouvelle aventure et je serais heureux de pouvoir le faire avec vous.\\

Je serais heureux de travailler avec vous car vos travaux me plaise. J'ai rapidement parcouru les productions réalisées par l'ancienne équipe CIDRE, les travaux de SUSHI et les sujets des
anciens thésards, et certains sont l'idée que je me suis faîte du travail futur que je voudrais réaliser. La détection automatique des failles par canal auxiliaire,
vérifier l'éxécution d'un programme; c'est que j'avais en tête en consultant la page de l'équipe SUSHI. En parallèle, mes travaux actuels m'ont sensiblisé à RiscV. Pour moi, il est clair
qu'il est possible d'améliorer/développer la sécurité de cette nouvelle architecture, cela laisse donc de la place pour de nouvelles personnes. Je voudrais faire partie de ces personnes.\\

\newpage
Cette offre de thèse est donc pour moi une possibilité que je souhaite explorer. Je réalise la quantité de travail à abattre, l'ensemble des articles et pages de documentation à lire
pour bien cerner le projet et planifier les trois années à venir. Ce travail ne m'effraie pas, il est clair et précis. Avec de la discipline et de la régularité tout peur être réalisé.
Je vais pouvoir côtoyer des personnes très savantes, développer la sécurité embarqué d'une nouvelle technologie et entrer dans la cours des grands.\\

J'ai hâte de pouvoir discuter avec vous, de vos attentes, de nos visions, durant notre entretien, concernant cette proposition de travail.
Je vous remercie pour le temps accordez à la lecture de cette lettre,\\

En vous souhaitant une agréable journée,\medbreak
Cordialement\\




%  Formé par les Pr.Gilles Zemor, Castagnos Guilhem, Maxime Bombar, Alice Pelet-Mary
% pour les bases en cryptographies et par les Pr.Abdou Guermouche et Emmanuel Fleury pour les bases en sécurité informatiques.
%  and I want to pursue in the academics.

% Thanks to this internship "Securing HACL* against side channels attacks
% by automatic detection with BINSEC", I have gained knowledge in
% assembly languages and most of all discover RISCV.

% That's why, at the end of summer, once the degree obtained, I want to
% continue working in the field of hardware security, secure programming
% and security against side-channels.

% For this objective, I begin contact other researcher in the field and
% gather information where I should ask for information. I started by
% asking parisian's labs, closer to my current location and thanks to the
% GDR and the communication about ARTMAN workshop I acknowledge former
% CIDRE team and current SUSHI.

% I first wanted to gain contact with members of the team to know if you
% could inform me about opportunities you could be aware of, but seeing
% you present yourself this PhD offer fills me with joy and motivation.

% By the present email, I inform you of my strong motivation for your
% offer and will be honored to have a meeting to discuss more in details
% with you and determine if I can fit your requirements.

% My master degree cover various subjects in cryptography and computer
% science, I know about fault injection, CPA and DPA. I know about
% complexity theory, hardness problem and SMT solvers (theoretically
% speaking). I have knowledge in computer architecture, design and
% software logic.

% I hope I could raise some interest in me. Thank you for the time given
% for reading my email.

% I wish you an excellent enf of the week,
% Best regards,
% Florian Duzes

% PS: I've done a lot of benevolent work with kids during the past years,
% with my BAFA, and work a lot with the educationnal aspect of animation.
% I would love to have an opportunitie to help in teaching and propagate
% knowledge.

% PPS: If your passing by Inria Paris, my room is 304. I would be honored
% to have a brief moment to talk with one of you.

% PPS: I'm form Albi, south of France, I would love to come back at more
% smaller cities than Paris. 

%-----------%
% Signature %
%-----------%

\raggedleft
\fullName\smallbreak
\includegraphics[scale=0.3]{sign.png}

\end{justify}
\end{large}