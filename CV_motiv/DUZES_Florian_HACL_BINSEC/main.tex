%%%%
% MTecknology's Resume
%%%%
% Author: Michael Lustfield
% License: CC-BY-4
% - https://creativecommons.org/licenses/by/4.0/legalcode.txt
%%%%

\documentclass[letterpaper,10pt]{article}
\usepackage{style}
\usepackage{informations}
%\justifying


%===================%
% CV DUZES FLORIAN %
%===================%
\begin{document}
\newgeometry{top=0.4cm, bottom=0.6cm, left=0.7cm, right=0.7cm}

  %---------%
  % Heading %
  %---------%

  \documentTitle{\fullName}{
    \href{tel:0750641522}{
      \raisebox{-0.05\height} \faPhone\ \phonenumber} ~ | ~
    \href{mailto:florian.duzes@gmail.com}{
      \raisebox{-0.15\height} \faEnvelope\ \mail} ~ | ~
    \href{https://www.linkedin.com/in/florian-duzes/}{
      \raisebox{-0.15\height} \faLinkedin\ \linkedin} ~ | ~
    \href{https://github.com/FloDarPie}{
      \raisebox{-0.15\height} \faGithub\ \github}
   % 29/11/2001 Castres Français
  }

  %---------%
  % Summary %
  %---------%

 \tinysection{Objectif}  Réaliser mon stage de fin d'étude de Master Cryptologie et Sécurité Numérique avec \entreprise\ pour l'offre :\\ \hspace{61pt}{ \large \stagename}.

  %--------%
  % Skills %
  %--------%

  \section{Compétences}

  \begin{multicols}{2}
    \begin{itemize}[itemsep=-2px, parsep=1pt, leftmargin=75pt]
      \item[\textbf{Cryptanalyse :}] LFSR, AES en CBC, Cryptanalyse linéaire et différentielle
      \item[\textbf{Cryptologie :}] McEliece, LWE, Kyber, Dilithium, AES, RSA, DES
      \item[\textbf{Algorithmique :}] Quantum Computing, algèbre, complexité
      
      \item[\textbf{Réseau :}] TCP exploit, Firewall, IPS/IPSec, LDAP, VLAN
      \item[\textbf{Langage :}] Python, SageMath, C, Rust, LaTeX
      \item[\textbf{Système :}] assembleur x86 et amd64, shellcodes 
      \item[\textbf{OS :}] Kernel exploitation, Kernel module, ROP, Rootkit, Windows, Debian, Ubuntu
     \item[\textbf{Réseau :}] TCP exploit, Firewall, IPS/IPSec, Auth. user Kerberos, LDAP, VLAN, VPN
      \item[\textbf{Langues :}] Anglais et Espagnol - C1
      \item[\textbf{Personnelles :}] Travail d'équipe, organisé, curieux, prise d'initiative sous supervision
    \end{itemize}
  \end{multicols}

  %-----------%
  % Education %
  %-----------%

  \section{Formations}

  \headingBf{Master Cryptologie et sécurité informatique}{2025}
  \heading{Option : Cryptographie Post-Quantique, Algorithmique arithmétique, Cryptanalyse, Cartes à puces, Sécurité des systèmes}{}
  \headingIt{Université de Bordeaux}{}

  \vspace{5pt}
  \headingBf{Licence Informatique, mention Intelligence Artificielle}{2022}
  \headingIt{Institut National Universitaire Jean-François Champollion}{}

  \vspace{5pt}
  \headingBf{Baccalauréat S-Scientifique, spécialité mathématique / Bachillerato}{2019}
  \headingIt{Lycée Général Lapérouse - diplôme binational franco-espagnol}{}

  \vspace{5pt}
  \headingBf{Certifications de langues}{2022}
  \headingIt{Certificat de langue anglaise, Cambridge Center - IHT  \hspace{20pt}-  C1}{}
  \headingIt{Bachillerato - Baccalauréat espagnol  \hspace{90pt}-  C1}{}

  
  \commentaire{
  \vspace{5pt}
  \headingBf{Diplômes et certifications}{}
  \vspace{3pt}
  \begin{multicols}{2}
  \begin{resume_list}
    \item Certificat de langue anglaise, Cambridge Center - IHT
    \item BAFA - Brevet d'aptitude aux fonctions d'animateur
    \item BSB \hspace{5pt}- Brevet de surveillant de baignade
  \end{resume_list}
  \end{multicols}
  }

  %----------------------------%
  % Extracurricular Activities %
  %----------------------------%

  \section{Projets informatiques}

  \headingBf{Travaux étudiants : stage, travaux de recherche}{depuis 2019}
  \begin{resume_list}
    \item Logiciel d'étude des comportements humains pour dérouter une IA, programmation Python
    \item Compilateur du langage C en OCaml
    \item Étude du cryptosystème de Paillier en application sur une BDD, implémentation en C
    \item Attaque par canal auxiliaire sur ECDSA et réduction de réseau, implémentation SageMath
  \end{resume_list}

  \headingBf{Global Game Jam}{depuis 2019}
  \begin{resume_list}
    \item Développement de jeu vidéo durant cet évènement mondial 
    \item Développement 2D, 3D, génération d'Intelligence Artificielle pour nos jeux
    \item Créativité, travail à distance et organisation avec les fuseaux horaires d'une équipe de huit personnes
  \end{resume_list}  

  \headingBf{Nuit de l'info}{2019 - 2021}
  \begin{resume_list}
    \item Programmation de site web sous forme de concours national avec des défis proposés par des entreprises
    \item Récompensé pour les solutions de l'équipe
    \item Travail d'équipe, brainstorming, développement web et gestion de projets en temps restreint  
  \end{resume_list}




  %-------------%
  % Experiences %
  %-------------%

  \section{Expériences}
  
  \headingBf{Présidences étudiantes}{2022 - 2023}
  \headingIt{Association Alea}{}
  \hspace{15pt}- Président d'association de jeux de société, en charge de l'organisation d'évènements ludiques\\
  \headingIt{Université Champollion}{}
  \hspace{15pt}- Vice-président étudiant du département Sciences et Technologie, animation du Conseil des Étudiants


  \headingBf{Animation jeunesse - bénévole}{depuis 2018}
  \headingIt{Éclaireuses et éclaireurs de France (EEDF, scoutisme laïque) - bénévole}{}
  \hspace{15pt}- Animateur pour des enfants de 6 à 15 ans. Activités durant l'après-midi, le week-end et un camp d'été. Je pratique une\\
  \hspace{20pt}éducation populaire, j'amène les enfants à se responsabiliser, à se sensibiliser à l'environnement...\\
  \hspace{15pt}- En charge de la gestion de séjours, d'équipes, de l'intendance, de la création d'activités pour les enfants\\
  \headingIt{Wakanga - direction en 2024}{}
  \hspace{15pt}- Responsable d'unité pendant deux ans puis directeur sur un séjour de seize jeunes de 11 à 14 ans, avec dérogation ministérielle
 

  %--------------------%
  % Passion et hobbies %
  %--------------------%

  \section{Loisirs}
    \hspace{10pt}\textbf{Gastronomie}
    \hfill\textbf{Cinéma}
    \hfill\textbf{Sports} \textit{Tennis, Taekwondo, VTT}
    \hfill\textbf{Lectures} \textit{Hypérion, Belgariade, Les chevaliers d'émeraudes}

\commentaire{
 \newpage
 \newgeometry{top=2.5cm, bottom=0.5cm, left=2cm, right=2cm}
  %------------------------------%
  % Section lettre de motivation %
  %------------------------------%

\begin{large}

%---------%
% Adresse %
%---------%

\hfill \entreprise \\
DUZÉS Florian  \hfill \adresse\\
68 rue Denfert Rocherau\hfill \adresseVille \\
33130 Bègles \hfill Yannis Sellami \\
\phonenumber \\
\mail

\medbreak\medbreak

 
   
\vspace{8pt}
{\color{accentLine} \hrule}
\begin{center}
    {\Large\color{accentTitle} \stagename}
\end{center}
{\color{accentLine} \hrule}
\vspace{8pt}

 %---------------%
 % Ma motivation %
 %---------------%

\begin{justify}
Bonjour Madame, Monsieur, le recruteur,\medbreak

À la suite de votre offre "\stagename", je vous soumets ma candidature. Actuellement étudiant en Master 2 Cryptologie et Sécurité Informatique à l’Université de Bordeaux, je suis passionné par les domaines de la cybersécurité, et plus particulièrement par les mécanismes de défense et d’investigation des systèmes d’information.\medbreak 

Mes études et projets universitaires m’ont permis d’explorer divers aspects de la sécurité : cryptanalyse avancée, sécurité des systèmes embarqués, tests d’intrusion et développement de solutions cryptographiques. Ces expériences, enrichies par des collaborations avec des  professionnels de Serma Technologies, STMicroelectronics et Synacktiv, m’ont apporté une approche pratique et rigoureuse des défis de la cybersécurité.\medbreak 

En parallèle, mes expériences professionnelles précédentes, notamment en tant que développeur et directeur de centre de vacances, m’ont permis de développer autonomie, organisation et travail en équipe – des compétences qui me permettront de rapidement répondre aux exigences de votre formation pour travailler au sein d'une équipe d'analystes. Actuellement, je travaille sur une étude d'attaque par canal auxiliaire de ECDSA.\medbreak 

Je suis également enthousiaste à l’idée de contribuer à vos projets qui allient innovation technologique et expertise métier. Je tenais à ajouter que j'ai engagé les démarches pour être réserviste cyber au sein de l'armée de l'air. J'ai pris cette avance pour pouvoir améliorer mes compétences et pour pouvoir accéder à des postes nécessitant une accréditation au secret. J'ai pu réaliser ma PMI au sein de la base 120, et terminer premier du classement.\medbreak 

Bien que je ne dispose pas encore de l’expérience d'un professionnel, je suis convaincu que ma motivation, ma capacité d’apprentissage rapide et mes bases solides en cybersécurité feront de moi un atout pour votre équipe. Je suis prêt à m’investir pleinement pour acquérir les compétences requises et progresser aux côtés de vos experts. \medbreak 

Disponible dès le 3 mars 2025, je serais ravi de discuter avec vous de la manière dont je pourrais m’intégrer dans vos équipes et évoluer dans un environnement aussi dynamique que celui de la \entreprise. \medbreak

Dans l’attente de votre retour, \medbreak

Je vous prie d’agréer,\medbreak Madame, Monsieur, l’expression de mes salutations distinguées.

%-----------%
% Signature %
%-----------%

\commentaire{Madame, Monsieur, le recruteur,

Actuellement étudiant en Master 2 Cryptologie et Sécurité Informatique à l’Université de Bordeaux, je vous présente ma candidature pour l'offre "Lutte Informatique Défensive Aéronautique".

Dans l’attente de votre retour,
Je vous prie d’agréer,
Madame, Monsieur, l’expression de mes salutations distinguées.}

\raggedleft
\fullName\smallbreak
\includegraphics[scale=0.3]{sign.png}

\end{justify}
\end{large}
}
% FIN
\end{document}